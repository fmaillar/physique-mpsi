\chapter{Changement de référentiel, dynamique dans un référentiel non galiléen}
\section{Transformation de la vitesse et de l'accélération d'un point par
changement de référentiel}
\subsection{Point coïncidant}
Soit un référentiel \(R\) défini par ses axes de coordonnées cartésiennes \((O;
X, Y, Z)\), que l'on appellera le \emph{référentiel fixe} ou \emph{référentiel
absolu} et un autre référentiel \(r\) en mouvement par rapport à \(R\) noté
\((o; x, y, z)\) appelé \emph{référentiel mobile} ou \emph{référentiel 
relatif}. Ces dénomination n'étant bien sûr que des commodités de langage, sans
signification physique valable.
