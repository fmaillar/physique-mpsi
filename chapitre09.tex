\chapter{Charges électriques, intensité, tension et lois de Kirchhoff}%
\minitoc{}
\minilof{}
\minilot{}

\section{Charge électrique}%
\label{chap9-sec:chargeelectrique}%

\subsection{Électrisation}%
\label{chap9-subsec:electrisation}%

Elle se fait par frottement, contact ou influence. Elle permet de distinguer 
isolants et conducteurs : à la surface d'un conducteur les charges sont mobiles 
(présence d'électrons libres). I1 y a deux types de charges électriques. Une 
charge négative est due à un excès d'électrons par rapport aux protons; une 
charge positive est due à un défaut d'électrons. Les atomes concernés par ces 
excès ou défauts d'électrons sont toujours extrêmement minoritaires.

\subsection{Définition}%
\label{chap9-subsec:definition}%

Une grandeur physique est dite ``mesurable'' si et seulement si l'on peut 
définir l'égalité et la somme ou le rapport de deux valeurs de cette grandeur. 
Soient deux charges \(q\) et \(q'\) ponctuelles, fixes, placées successivement 
au même point, dans le même environnement électrique (1) et subissant les 
forces électrostatiques \(\vv{F_1}\) et \(\vv{F'_1}\), puis dans un autre 
environnement (2), \(\vv{F_2}\) et \(\vv{F'_2}\) etc.

On peut constater par la mesure que \(\frac{F'_1}{F_1} = \frac{F'_2}{F_2} = 
\dotsb\) Par définition, on pose \(\abs{\frac{q'}{q}} = \frac{F'_1}{F_1}\).
L'unité internationale de la charge électrique est le coulomb, c'est la charge 
qui traverse toute section d'un circuit parcourue par un courant de 
\(\SI{1}{A}\) pendant une durée de \(\SI{1}{s}\), donc \(\SI{1}{C} = 
\SI{1}{A.s}\).

\subsection{Quantification de la charge électrique}%
\label{chap9-subsec:quantificationdelachargeelectrique}%

Historiquement ceci a été vérifié par l'expérience de Millikan. La charge de 
tout objet (isolable) est un multiple de la charge élémentaire : \(q = z e\), 
avec \(z\) un entier relatif. Ceci ne s'applique donc pas aux quarks que l'on 
ne sait pas séparer les uns des autres. La charge élémentaire vaut \(e = 
\SI{1,6022e-19}{C}\).
Nombres de charge des particules fondamentales : électron \(z = -1\), proton 
\(z = 1\), neutron \(z = 0\), positron \(z = 1\), \dots,  mais pour les quarks: 
up (u.) \(z = 2/3\), down (d) \(z = -1/3\).

\subsection{Propriétés des charges électriques}%
\label{chap9-subsec:proprietesdeschargeselectriques}%

Elles sont invariantes dans un changement de référentiel. La charge électrique 
est une grandeur extensive conservative. Elle est donc invariable pour un 
système fermé. Les charges de chaque signe se conservent dans un système fermé 
en l'absence de toute réaction de création ou d'annihilation de particules.

\section{Densités de charge électrique}%
\label{chap9-sec:densitesdechargeelectrique}%

\subsection{Densité volumique de charge}%
\label{chap9-subsec:densitevolumiquedecharge}%

Soit un volume élémentaire \(\D\tau\) autour du point \(M\), portant la charge 
\(\D q\), la densité volumique de charge en \(M\) est: \(\rho = 
\derived{q}{\tau}\) (\(\D\tau\) est de l'ordre de \(\SI{1e-24}{m^3}\), \(\D q\) 
est grand devant la charge élémentaire \(e\)). Son unité internationale est le 
\(\si{C/m^3}\). Si \(n_i\) est la concentration volumique (nombre par unité de 
volume en métre cube) des particules de charge \(q_i\), la densité volumique de 
charge s'écrit encore : \(\rho = \sum_i n_i q_i\).

À l'intérieur d'un conducteur en équilibre électrique, \(\rho = 0\). Les 
charges éventuelles sont réparties sur la surface du conducteur (voir le cours 
de deuxième année). Dans un volume \(V\), la charge est \(q=\iiint_V \rho \D 
V\). Cette écriture signifie simplement que la charge \(q\) est la somme d'un 
nombre infini de charges infiniment petites. Le signe \(\iiint\) est utilisé 
ici parce que le volume élémentaire \(\D\tau = \D x \D y \D z\) est le produit 
de trois différentielles, ce qui implique qu'il faut effectuer trois 
intégrations, sur les valeurs de \(x\), sur celles de \(y\) et sur celles de 
\(z\) pour obtenir \(q\).

\subsection{Densité surfacique de charge, densité linéique de charge}%
\label{chap9-subsec:densitesurfacique}%

Soit un surface élémentaire d'aire \(\D S\) autour du point \(M\), portant la 
charge \(\D q\), la densité surfacique de charge en \(M\) est : \(\sigma = 
\derived{q}{S}\). (\(\D S\) de l'ordre de \(\si{1e-16}{m^2}\), \(\D q\) grand 
devant la charge élémentaire \(e\)). Son unité internationale est le 
\(\si{C.m^{-2}}\). Sur une surface \(\Sigma\), la charge est \(\iint_\Sigma 
\sigma \D S\).

Soit une ligne élémentaire de longueur \(\D L\) autour du point \(M\), portant 
la charge \(\D q\), la densité linéique de charge en \(M\) est : \(\lambda = 
\derived{q}{L}\). (\(\D L\) de l'ordre de \(\si{1e-8}{m}\), \(\D q\) grand 
devant la charge élémentaire \(e\)). Son unité internationale est le 
\(\si{C.m^{-1}}\). Sur une ligne \(\Gamma\), la charge est \(\int_\Gamma 
\lambda \D L\).

\section{Densités de courant}%
\label{chap9-sec:densitedecourant}%

\subsection{Densité volumique de courant}%
\label{chap9-subsec:densitevolumique}%

Soit, pour le \(i\)\ieme type de porteur de charge, au point \(M\):
\begin{itemize}%
\item \(q_i\): charge de ce type de porteur de charge,
\item \(\rho_i\): densité volumique de charge mobile (en \(\si{C.m^{-3}}\)),
\item \(\vvv_i\): vecteur vitesse moyen (en \(\si{m/s}\)),
\item \(n_i\): concentration volumique (en \(\si{m^{-3}}\)).
\end{itemize}%
La densité volumique de courant en \(M\) est \(\vj = \sum_i \rho_i \vvv_i = 
\sum_i n_i q_i \vvv_i\). Elle s'exprime en \(\si{A.m^{-2}}\).

\subsection{Densité surfacique de courant}%
\label{chap9-subsec:Densité_surfacique_de_courant}%
Avec des notations semblables, pour des courants circulant sur une surface, la 
densité surfacique de courant en \(M\) est~:\(\vj_s = \sum_i \sigma_i \vvv_i\).
\(\vj_s\) s'exprime en \(\si{A.m^{-1}}\).

\section{Intensité d'un courant électrique}%
\label{chap9-sec:intensiteduncourantelectrique}%

Soit une surface \(\Sigma\) à travers laquelle circule un courant d'intensité 
algébrique \(i\). \(\si{A}=\si{C.s^{-1}}\) donc ceci signifie que \(i\) est le 
nombre d'ampères, c'est-à-dire le nombre de coulombs qui traverse \(\Sigma\) 
par seconde dans le sens de la flèche qui donne la convention de signe pour 
\(i\).
% Inserer figure

\(\vv{\D S}\) représente un vecteur surface élémentaire de norme \(\D S\) 
autour du point \(M\) (aire élémentaire), normal à la surface \(\Sigma\), 
orienté dans le sens de la flèche qui définit la convention de signe pour 
\(i\). Si la densité volumique de charges mobiles, au voisinage de \(M\) est 
\(\rho\), et si leur vitesse moyenne est \(\vvv\), la densité volumique de 
courant est \(\vj = \rho \vvv\).
Les charges mobiles contenues dans le cylindre de base , de génératrice \(\vvv 
\D t\), sont celles qui auront traversé la surface \(\D S\) au bout du temps 
\(\D t\). Le volume de ce cylindre est \(\D \vv{S} \cdot \vvv\D t\).
La charge qui traverse \(\D S\) pendant le temps \(\D t\) a donc pour 
expression
\begin{equation}%
  \D q = \rho \D \vv{S} \cdot \vvv \D t = \vj \cdot \D S \D t.
\end{equation}%

L'intensité du courant traversant la surface \(\D S\) dans le sens défini par 
la flèche est \(\D i = \derived{q}{t} = \vj \cdot \D \vv{S}\).
Pour obtenir l'intensité totale du courant traversant la surface \(\Sigma\), il 
faut additionner les intensités élémentaires qui traversent toutes les surfaces 
élémentaires dont la réunion constitue \(\Sigma\).
L'intensité du courant électrique traversant \(\Sigma\) dans le sens défini par 
la flèche est donc le ``flux'' du vecteur densité volumique de courant à 
travers \(\Sigma\)~: \(i = \iint_\Sigma \vj \cdot \D \vv{S}\).

Dans le cas d'un courant surfacique (nappe de courant), pour évaluer le courant  
traversant une ligne \(\Gamma\), on fait correspondre à un élément de longueur 
\(\D L\) de cette ligne un vecteur, normal à la ligne, dans le plan tangent à 
la surface où circule le courant et orienté dans le sens de la flèche qui 
définit la convention de signe pour l'intensité. L'élément \(\D L\) est 
traversé parle courant d'intensité \(\D i = \vj_S \cdot \D \vv{L}\) et 
l'intensité totale qui traverse la ligne \(\Gamma\) est \(i = \int_\Gamma \vj_S 
\cdot \D \vv{L}\).

\section{Lignes de courant, tubes de courant}%
\label{chap9-sec:lignesdecourant}%

En chaque point d'une ligne de courant \(\vj\) est tangent à  cette ligne. Les 
lignes de courant sont orientées dans le même sens que \(\vj\). Ce sont donc 
les lignes de champ de la densité volumique de courant. Pour une nappe de 
courant, ce sont les lignes de champ de la densité surfacique de courant 
\(\vj_S\).
Les équations d'une ligne de courant sont telles que~: soit un point 
\(M(x,y,z)\) sur une ligne de courant et un vecteur \(\D M\) joignant \(M\) à 
un point \(M'\) infiniment voisins de \(M\) sur la même ligne de courant.
\(\D \vv{M} = \D x \ux + \D y \uy + \D z \uz\) est parallèle à \(\vj = j_x \ux 
+ j_y \uy + j_z \uz\) donc \(\frac{\D x}{j_x} = \frac{\D y}{j_y} = \frac{\D 
z}{j_z}\). L'intégration de ces équations différentielles donne l'équation des 
lignes de courant.

\emph{Un tube de courant s'appuie sur une courbe fermée et est limité par des 
lignes de courant.}%

Pour une portion de tube de courant limitée par les sections \(\Sigma_1\) et 
\(\Sigma_2\), avec \(I_1\) courant entrant par \(\Sigma_1\) et \(I_2\) courant 
sortant par \(\Sigma_2\). Le courant sortant par la paroi \(T\) est nul car en 
chaque point de \(T\), \(\vj\) est normal à \(\D \vv{S}\) et \(\vj \cdot \D 
\vv{S} = 0\). Le courant \(i\) sortant de cette portion de tube de courant est 
donc \(i = I_2 - I_1\).

\section{Régime stationnaire, approximation des régimes quasi-stationnaires 
(ARQS)}%
\label{chap9-sec:ARQS}%

\subsection{Définition}%
\label{chap9-sec:def}%

En régime stationnaire, toutes les grandeurs physiques sont indépendantes du 
temps (sauf bien entendu les coordonnées des porteurs de charge). Les dérivées 
partielles par rapport au temps de En régime stationnaire, toutes les grandeurs 
physiques sont indépendantes du temps (sauf bien entendu les coordonnées des 
porteurs de charge).
Les dérivées partielles par rapport au temps de \(n_i\), \(\rho_i\), \(\vj\), 
\(\vvv_i\), T, etc.\ sont nulles en chaque point. L'approximation des régimes 
quasi stationnaires (ARQS) concerne les régimes ``lentement'' variables.

Concrètement, en régime sinusoïdal permanent (courant alternatif sinusoïdal), 
on peut appliquer cette approximation tant que la fréquence est suffisamment 
basse pour que la longueur d'onde de la propagation du courant soit grande 
devant les dimensions du circuit. En notant \(f\) la fréquence et \(v\) la 
vitesse de propagation du courant (\(v\approx c = \SI{3e8}{m.s^{-1}}\) : 
vitesse de la lumière dans le vide), cette longueur d'onde est \(\lambda = 
\frac{c}{f}\).

Pour une fréquence de \(\SI{1}{MHz}\), \(\lambda \approx \SI{300}{m}\), il 
suffit donc que les dimensions du circuit ne dépassent pas \(\SI{1}{m}\) pour 
qu'on puisse appliquer l'ARQS.

Pour \(f=\SI{50}{Hz}\) (courants EDF) \(\lambda \approx \SI{6000}{km}\), pour 
des lignes de transport de plusieurs centaines de kilomètres, l'ARQS ne serait 
pas très bonne.

\subsection{Tube de courant dans le cadre de l'ARQS}%
\label{chap9-subsec:TubedeCourantARQS}%

En régime stationnaire, si on note \(q\) la charge électrique contenue dans un 
volume déterminé, on a \(\derived{q}{t}=0\). Par exemple, pour le tube de 
courant \(T\), limité par les sections \(\Sigma_1\) et \(\Sigma_2\). Le courant 
sortant par \(T\) est nul. Le courant sortant est donc \(i = I_2 - I_1\) or 
\(i=-\derived{q}{t}=0\), donc \(I_1 = I_2\). L'intensité du courant a la même 
valeur à travers toutes les sections d'une branche du circuit. Pour un tube de 
courant se partageant en deux branches : On a pour la même raison : \(I = I_1 + 
I_2\).

En ARQS, ces lois restent valables approximativement, mais les intensités 
varient (lentement) au cours du temps.

\section{Potentiel électrostatique, tension électrostatique ou différence de 
potentiel électrostatique}%
\label{chap9-sec:potentiel}%

\subsection{Potentiel électrostatique}%
\label{chap9-subsec:potentiel}%

Une répartition de charges (ponctuelles, volumiques, surfaciques ou linéiques) 
fixe et indépendante du temps crée un champ électrostatique. Le champ 
électrostatique en un point où une charge \(q\) subit (ou subirait) la force 
électrostatique \(\vF\)  est \(\vv{E} = \frac{\vF}{q}\), il ne dépend pas de 
\(q\) mais de la disposition et des valeurs des charges qui créent ce champ.

\emph{Les forces électrostatiques sont conservatives donc elles dérivent d'un 
potentiel : l'énergie potentielle électrostatique}. Pour un déplacement \(\D 
\vv{M}\) on a \(\vF \cdot \D \vv{M}=\delta W = -\D E_p\) soit \( q \vv{E} \cdot 
\D \vv{M}=-\D E_p\) avec \(q\) constante donc \(\D \left(\frac{E_p}{q}\right) = 
-\vv{E} \cdot \D \vv{M}\).%
\(V = \frac{E_p}{q}\) est le potentiel électrostatique au point \(M\), il ne 
dépend pas de la charge \(q\), ni même de la présence d'une charge électrique 
en \(M\) puisque sa différentielle est \(\D V = -\vv{E} \cdot \D \vv{M}\) et 
que \(\vv{E}\) dépend des charges qui le créent et non de celle qui 
éventuellement le subit.

Son unité internationale est le volt \(\si{V} = \si{J/C} =\si{W/A} = 
\si{kg.m^2.s^{-3}.A^{-1}}\). On peut donc définir le potentiel électrostatique 
par la formule
\begin{equation}%
 \D V = -\vv{E} \cdot \D \vv{M}.
\end{equation}%
Il en résulte que l'unité internationale du champ électrostatique est le volt 
par mètre : \(\si{V/m}\).

Étant défini par sa différentielle, (comme une énergie potentielle ou un 
avancement de réaction\ldots) le potentiel électrostatique n'est ainsi défini qu'à 
une constante prés que l'on peut choisir arbitrairement.

\subsection{Surfaces équipotentielles}%
\label{chap9-subsec:surfacesequipotentielles}%

\emph{Une surface équipotentielle est une surface sur laquelle le potentiel 
électrique \(V\) est uniforme}. Soient deux point \(M_1\) et \(M_2\) très 
voisins sur une surface équipotentielle. On note \(\D \vv{M}=\vv{M_1M_2}\) et 
\(dV= V_1 - V_2 = 0\), on a donc \(\D V = -\vv{E}\cdot\D \vv{M} = 0\)  donc 
\(\vv{E} \perp \D \vv{M}\).%
\emph{En un point d'une surface équipotentielle, le champ électrostatique est 
normal à la surface équipotentielle}.%

\subsection{Différence de potentiel électrostatique ou tension 
électrostatique}%
\label{chap9-subsec:differencedepotentielelectrostatique}%

\emph{La différence de potentiel électrostatique entre deux points} \(M_1\) et 
\(M_2\) est obtenue par intégration de la relation précédente : \(V_2 - V_1 = 
\int_{M_1}^{M_2} -\vv{E} \cdot \D \vv{M}\), cette intégrale est bien sûr 
indépendante du chemin suivi de \(M_1\) à \(M_2\) puisqu'on intègre une 
différentielle totale exacte : celle de V.%

La diminution du potentiel électrostatique entre \(M_1\) et \(M_2\) est donc 
égale à la circulation du champ électrostatique de \(M_1\) à \(M_2\): \(V_1-V_2 
= \int_{M_1}^{M_2} \vv{E} \cdot \D \vv{M}\).
Dans le cas d'un champ électrostatique uniforme : \(V_1-V_2 = \vv{E} \cdot 
\vv{M_1M_2}\).
La différence de potentiel électrostatique (d.d.p.) entre deux points est aussi 
nommée \emph{tension électrostatique entre ces deux points}. On peut la définir 
par une flèche :
%mettre figure
La tension \(u\) définie par la flèche ci-dessus est \(u=V_1-V_2\) (On peut 
aussi la noter \(u_{12}\)). Le travail de la force électrostatique subie par 
une charge ponctuelle \(q\) se déplaçant de \(M_1\) à \(M_2\) peut donc 
s'écrire : \(W_{1,2} = E_{p_1} - E_{p_2}\) avec \(E_p = q V\) donc \(W_{1,2} = 
q(V_1-V_2)\) ou \(W_{1,2} = q u\) si la flèche définissant \(u\) va de \(M_2\) 
vers \(M_1\).

\section{Cas de l'ARQS, différence de potentiel électrique ou tension 
électrique}%
\label{chap9-sec:casdelARQS}%

Dans le cas d'un régime variable, les charges sont soumises à un champ 
électromagnétique, formé d'un champ électrique et d'un champ magnétique qui 
sont couplés (chacun dépend des variations temporelles de l'autre). Il existe 
toujours un potentiel \(V\), nommé alors potentiel électrique, mais la relation 
\(\D V = -\vv{E} \cdot \D \vv{M}\) n'est plus valable (voir cours de deuxième 
année).
On admettra que si le régime est lentement variable, c'est-à-dire 
quasi-stationnaire, les relations obtenues en électrostatique restent valables 
avec une bonne approximation.
En ARQS : \(V\) représente alors le potentiel électrique, \(E_p\) l'énergie 
potentielle électrique (ou électrocinétique) et \(W\) le travail de la force 
électrique agissant sur la charge ponctuelle \(q\).

\section{Travail et puissance électrocinétiques reçus par un dipôle 
électrocinétique en ARQS}%
\label{chap9-sec:travailetpuissance}%

Une portion de circuit (tube de courant) comprise entre deux surfaces 
équipotentielles \(A\) et \(B\), constitue un dipôle électrocinétique \(D\). Si 
le régime est quasi stationnaire, l'intensité \(i\) (ou \(i_{AB}\))  du courant 
circulant de \(A\) vers \(B\) est la même à travers toutes les sections du tube 
de courant, et en particulier, à travers les deux équipotentielles \(A\) et 
\(B\). Soit \(u\) la différence de potentiel ou tension électrique définie sur 
le schéma (\(u = u_{AB} = V_A - V_B\)).

L'énergie électrocinétique que reçoit une charge \(q\) qui le traverse de la 
part du reste du circuit est : \(W = E_{p_A} - E_{p_B} = q(V_A - V_B)\). En un 
temps \(\D t\) :
\begin{itemize}%
\item il entre par \(A\) dans le dipôle, la charge \(\D q = i \D t\) avec 
  l'énergie potentielle \(\D Ep_A = i \D t V_A\),
\item il sort par \(B\) du dipôle la même charge \(\D q = i \D t\) (en ARQS la 
  charge de D est quasi-constante), avec l'énergie potentielle \(\D Ep_B = i \D 
    t V_B\),
\item il ne sort ni ne rentre aucune charge par la surface latérale du tube de 
  courant,
\item à l'intérieur du tube, les charges se sont déplacées mais en ARQS chaque 
  petit volume contient une charge électrique constante à un potentiel 
    constant. L'énergie potentielle électrique des charges intérieures au tube 
    n'a donc pas changé pendant \(\D t\).
\end{itemize}%
L'énergie électrocinétique reçue par le dipôle de la part du reste du circuit 
pendant \(\D t\), c'est-à-dire le travail total des forces électriques qui 
s'exercent sur les porteurs de charge du dipôle est donc
\begin{equation}%
\delta W = (V_A - V_B) i \D t = u i \D t = u \D q.
\end{equation}%

La puissance électrocinétique reçue par le dipôle est \(\P = \frac{\delta W}{\D 
t}=ui\). Si \(u\) et \(i\) sont fléchés dans le même sens on a bien sûr \(P = 
-ui\).

Si \(P  > 0\) le dipôle reçoit de l'énergie électrocinétique de la part du 
reste du circuit. Si \(P < 0\), il fournit de l'énergie électrocinétique au 
reste du circuit, c'est un générateur.

\section{Lois de Kirchhoff}%
\label{chap9-sec:loisdeKirchhoff}%

\subsection{Définitions}%
\label{chap9-subsec:kirchhoffdef}%

Ces lois concernent les circuits électriques filiformes, c'est-à-dire formés de 
composants connectés entre eux par des fils conducteurs fins. Un n\oe{}ud est 
un point d'où partent plus de deux branches du circuit. Une maille est une 
suite de branches du circuit partant d'un point pour aboutir au même point.

\subsection{Loi des n\oe{}uds}%
\label{chap9-subsec:loidesnoeuds}%

On a déjà démontré, en ARQS, que l'intensité totale du courant sortant d'un 
conducteur est nulle donc, si \(n\) courants d'intensités \(i_1\), \(i_2\), 
\ldots, \(i_n\) arrivent en un nœud par les différentes branches, leur somme
est nulle:
\begin{equation}%
  \sum_{k=1}^n i_k = 0.
\end{equation}%
Il en est de même pour les courants arrivant en un n\oe{}ud.

\subsection{Loi des mailles}%
\label{chap9-subsec:loidesmailles}%

Le potentiel électrique dépend du point considéré, la différence de potentiel 
ne dépend pas du chemin suivi par les porteurs de charge, d'où la loi des 
mailles :
\emph{La somme des tensions représentées par des flèches, correspondant toutes 
au même sens de parcours le long d'une maille, est nulle}. En notant \(u_1\), 
\(u_2\), \ldots, \(u_n\) les \(n\) tensions correspondantes :%
\begin{equation}%
  \sum_{k=1}^n u_k = 0.
\end{equation}%

\subsection{Utilisation des lois de Kirchhoff}%
\label{chap9-subsec:utilisationdeslois}%

On étudie ici le cas d'un circuit pour lequel on connaît les relations \(u = 
f(i)\) pour chacun des dipôles qui le constituent (\(u = R i\) pour un 
conducteur ohmique, \(u = E - R i\) pour un générateur, \(u = R i + E'\) pour 
un récepteur). On cherche à calculer les intensités des courants dans toutes 
les branches de ce circuit, (on pourra en déduire ensuite les tensions).

L'utilisation directe des lois de Kirchhoff se fait en écrivant autant 
d'équations qu'il y a d'intensités inconnues : on écrit d'abord la loi des 
nœuds pour tous les nœuds moins un. (le dernier nœud fournirait une équation 
qui serait une combinaison linéaire des précédentes).

On écrit ensuite la loi des mailles autant de fois que nécessaire pour 
compléter le système d'équations, après avoir choisi sur chaque maille utilisée 
un sens de parcours déterminé. Il suffit pour ceci de repérer un ensemble de 
mailles indépendantes.

Chaque tension intervenant dans une équation est exprimée en fonction de 
l'intensité dans la branche correspondante en utilisant l'expression de la 
caractéristique \(u = f(i)\) du dipôle contenu dans cette branche.

On résout ensuite le système d'équations par le moyen le plus adapté.

Cette méthode est souvent longue, d'autres méthodes plus rapides seront 
étudiées ultérieurement.

\section{Exercices}%
\label{chap9-sec:exercices}%

\begin{exercice}[Conducteur métallique]%
  \begin{enumerate}
  \item Sachant que chaque atome de cuivre fournit un électron libre, calculer 
    la densité volumique de charge mobile \(\rho\) dans un conducteur en 
      cuivre. On donne la masse volumique du cuivre : \(\mu = 
      \SI{8,96e3}{kg/m^3}\), son numéro atomique : \(Z = 29\) et sa masse 
      molaire atomique : \(M = \SI{63,5e-3}{kg/mol}\). La charge élémentaire 
      est \(e = \SI{1,602e-19}{C}\), la constante d'Avogadro est \(N  = 
      \SI{6,02e23}{mol^{-1}}\).
  \item Un fil de cuivre cylindrique de rayon \(r = \SI{0,500}{mm}\) est 
    parcouru par un courant d'intensité \(I = \SI{1,00}{A}\). En admettant que 
      le vecteur densité de courant soit uniforme dans le fil et parallèle à 
      l'axe du cylindre, calculer la norme \(j\) de ce vecteur ainsi que la 
      norme \(v\) du vecteur vitesse moyen des électrons libres.
  \item Ce fil est enroulé à spires jointives sur un cylindre de diamètre \(D = 
    \SI{10}{cm}\). On considérera que les spires sont pratiquement circulaires 
      et qu'elles forment une nappe de courant. Quels sont la direction, et la 
      norme \(j_s\) du vecteur densité surfacique de courant?
  \end{enumerate}
\end{exercice}%
\begin{exercice}[Pont de Wheatstone]%
  %TODO Mettre la figure
  \begin{enumerate}
  \item Écrire les équations d'inconnues \(i, i_{1 \ldots 5}\), données par les 
    lois de Kirchhoff, qui permettent le calcul de ces intensités.
  \item Si \(i_5 = 0\), quelle relation y a-t-il entre \(R_1, R_2, R_3\) et 
    \(R_4\)? Exprimer alors toutes les intensités avec les résistances et 
      \(E\).
  \end{enumerate}
\end{exercice}%
\begin{exercice}[Lois de Kirchhoff]%
  Pour le circuit suivant~:
  %METTRE LA FIGURE
  \begin{enumerate}
  \item Écrire la relation entre l'intensité et la tension pour chaque branche 
    du circuit (On appellera les tensions \(u\), \(u'\), \(u_1\), \(u_2\) 
      \ldots \(u_5\) et on les fléchera sur le circuit).
  \item \(E, E', r, r', R_{1 \ldots 5}\) étant des constantes supposées connues, 
    écrire les équations données par les lois de Kirchhoff qui permettent de 
      calculer les intensités.
  \item Réécrire en le simplifiant le système d'équations précédent pour \(E' = 
    2 E\), \(r' = r\) et \(R_1 = R_2 = R_3 = R_4 = R_5 = R\), avec \(E\), \(r\) 
      et \(R\) pour seuls paramètres.
  \item Résoudre ce système d'équations pour \(E = \SI{6}{V}\) , \(R = 
    \SI{10}{\ohm}\) et \(r = \SI{2}{\ohm}\).
  \end{enumerate}
\end{exercice}%
\begin{exercice}[Courant volumique]%
  Une tôle métallique est définie dans le repère cartésien Oxyz, de vecteurs 
  unitaire \(\ux, \uy, \uz\), par \(y \in \intervalleff{0}{\ell}\), \(x \in 
  \intervalleff{0}{L}\) et \(z \in \intervalleff{0}{H}\). Elle est parcourue 
  par un courant de densité volumique \(\vj = j_0 \exp\left(-\frac{z}{a}\right) 
  \ux\). Dans cette expression la constante \(a\) est une constante très petite 
  devant l'épaisseur \(H\) de la tôle.
  \begin{enumerate}
  \item Exprimer l'intensité \(I\) du courant qui traverse une section de la 
    tôle normale à l'axe Ox.
  \item H étant petite devant \(\ell\) et L, on assimile la tôle à une surface. 
    Exprimer le vecteur densité surfacique de courant \(\vj_s\).
  \end{enumerate}
\end{exercice}%
\begin{exercice}[Sphère chargée en volume]%
  Soit une sphère de rayon \(R\), de centre \(O\), chargée électriquement. La 
  densité volumique de charge en un point \(M\), à la distance \(r\) de \(O\) 
  est \(\rho = \frac{q \exp\left(-\frac{r}{a}\right)}{4 \pi r a^2}\). Calculer 
  sa charge électrique totale \(Q\) pour \(R = a\) ainsi que \(\lim\limits_{R 
  \to \infty} Q\).
\end{exercice}%

% Local Variables:
% mode: latex
% TeX-master: "physique"
% End:
