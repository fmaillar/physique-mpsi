\chapter{Quadripôles, filtres du premier ordre}
\minitoc
\minilof
\minilot

\section{Quadripôle, fonction de transfert}
\subsection{Quadripôle, générateur, charge}
Un quadripôle est une portion de circuit présentant quatre bornes~: deux bornes d'entrée et deux bornes de sortie. Les tension et les courants sont en général algébrisés comme sur le schéma ci-dessous. $u_E$ tension d'entrée, $i_E$ courant d'entrée, $u_S$ tension de sortie, $i_S$, courant de sortie (pour des raisons de symétrie, il peut être utile de flécher $i_S$ dans l'autre sens). Un quadripôle est en général associé à deux dipôles électrocinétiques~:
\begin{itemize}
\item Un dipôle générateur relié aux bornes d'entré ; Si ce dipôle est linéaire, au moins dans le domaine où on l'utilise, il pourra être modélisé par un générateur de Thévenin ou de Norton ; Si la résistance du générateur (ou, en régime sinusoïdal permanent, son impédance) est négligeable, on le modélisera comme une source de tension ; De la même manière, si sa conductance (ou, en régime sinusoïdal permanent, son admittance) est négligeable, on le modélisera comme une source de courant ;
\item Un dipôle récepteur relié aux bornes de sortie, que l'on nommera ``charge''. Ce sera souvent un dipôle passif, par exemple une résistance pure.
\end{itemize}
Un quadripôle peut comporter un dipôle générateur, le plus souvent une source liée dont le fonctionnement nécessite une alimentation (voir les montages à amplificateurs opérationnels), celle-ci n'est en général pas représentée sur les schémas. Lorsque le quadripôle ne comporte pas de source, il est dit ``passif'', dans le cas contraire, il est dit ``actif''.
\subsection{Fonctions de transfert --- Transmittances}
En régime sinusoïdal permanent, pour un quadripôle linéaire, les grandeurs d'entrée et de sortie sont de même pulsation, ou \emph{fréquence angulaire}, $\omega$. Les grandeurs complexes associées à une grandeur d'entrée $\xc_E$ et à une grandeur de sortie $\yc_S$ sont respectivement $\xc_E = \Xc_E \exp(\ju \omega t)$ et $\yc_S = \Yc_S \exp(\ju \omega t)$. Le rapport $\frac{\yc_S}{\xc_E} = \frac{\Yc_S}{\Xc_E}$ est indépendant du temps, c'est une fonction de $\ju \omega$, que l'on note $\Hc(\ju \omega)$.

La \emph{fonction de transfert, ou ``transmittance complexe'', du quadripôle} est  $\Hc(\ju \omega) = \frac{\yc_S}{\xc_E} = \frac{\Yc_S}{\Xc_E}$.

Il est très important de note dès maintenant, que la fonction de transfert d'un quadripôle dépend, \emph{a priori}, non seulement du quadripôle considéré, mais aussi de la charge.

On peut distinguer quatre transmittance pour un quadripôle suivant la nature, intensité ou tension, des grandeurs $\yc_S$ et $\xc_E$ auxquelles on s'intéresse~:
\begin{itemize}
	\item Amplification en tension : $\frac{\Uc_S}{\Uc_E}$ sans dimension ;
	\item Amplification en courant : $\frac{\Ic_S}{\Ic_E}$ sans dimension ;
	\item Transimpédance complexe : $\frac{\Uc_S}{\Ic_E}$ en \si{\ohm};
	\item Transadmittance complexe : $\frac{\Ic_S}{\Uc_E}$ en \si{\siemens}.
\end{itemize}
Le module $H$ et l'argument $\Phi$ d'une fonction de transfert sont des fonctions de la fréquence angulaire $\Hc(\ju \omega) = H(\omega) \exp(\ju \Phi(\omega))$.
\subsection{Amplification, Gain, Décibels}
Si les grandeurs d'entrée et de sortie considérées sont de même nature, le module $H$ de la fonction de trnasfert est nommé ``gain'', il peut varier énormément avec la pulsation, c'est pourquoi on utilise un échelle logarithmique.

Le logarithme décimal du rapport de deux puissance (électrocinétiques, acoustiques, $\dots$) s'exprime en décibels (\si{\decibel).