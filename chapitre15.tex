\chapter{Quadripôles, filtres du premier ordre}
\minitoc
\minilof
\minilot

\section{Quadripôle, fonction de transfert}
\subsection{Quadripôle, générateur, charge}
Un quadripôle est une portion de circuit présentant quatre bornes~: deux bornes d'entrée et deux bornes de sortie. Les tension et les courants sont en général algébrisés comme sur le schéma ci-dessous. $u_E$ tension d'entrée, $i_E$ courant d'entrée, $u_S$ tension de sortie, $i_S$, courant de sortie (pour des raisons de symétrie, il peut être utile de flécher $i_S$ dans l'autre sens). Un quadripôle est en général associé à deux dipôles électrocinétiques.

Un dipôle générateur relié aux bornes d'entré. Si ce dipôle est linéaire, au moins dans le domaine où on l'utilise, il pourra être modélisé par un générateur de Thévenin ou de Norton.