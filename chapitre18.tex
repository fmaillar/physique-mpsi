\chapter{Mouvement dans un champ de forces centrales conservatives}%
\minitoc%
\minilof%
\minilot%

\section{Mouvement à accélération centrale}%
\subsection{Définition, planéité du mouvement, loi des aires}%
Un point matériel a un mouvement à accélération centrale de centre \(O\), fixe
dans le référentiel considéré, si et seulement si son accélération est
parallèle à son vecteur position \(\vv{OM}\) à chaque instant.

%TODO:Mettre un schéma

Alors, pour ce mouvement, on a \(\vv{OM} \wedge \vv{a} = \vv{0}\). Soit le
vecteur \(\vv{C} = \vv{OM} \wedge \vvv\). En le dérivant, on obtient
\(\derived{\vv{C}}{t} = \vvv \wedge \vvv + \vv{OM} \wedge \vv{a} = \vv{0}\),
donc \(\vv{C}\) est constant (Le moment cinétique en \(O\) \(m\vv{C}\)%
est constant). Le vecteur \(\vv{OM}\) reste donc perpendiculaire à un vecteur
constant et par conséquent \emph{le mouvement est plan}.

Pendant un intervalle de temps du durée \(\D{t}\), \(\vv{OM}\) balaye une
surface élémentaire qui est la norme du vecteur \(\D\vv{S} =
\dfrac{\vv{OM}\wedge\D\vv{M}}{2}\), donc la ``vitesse aréolaire'' est %
\(\derived{\vv{S}}{t} = \dfrac{\vv{OM}\wedge\vvv}{2} = \dfrac{\vv{C}}{2}\).

Dans un mouvement à accélération centrale, \emph{la vitesse aréolaire est
constante}. C'est la loi des aires. Le vecteur \(\vv{C}\) est \emph{la
constante de la loi des aires}.

\subsection{Utilisation du théorème du moment cinétique}%
On arrive aux mêmes conclusions par le raisonnement suivant~: si le mouvement
de \(M\) est à accélération centrale, de centre \(O\), alors il est soumis à la
force totale \(\vv{F} = m \vv{a} \parallel \vv{OM}\), c'est-à-dire à une
\emph{force centrale (exercée par \(O\))}. Le moment de la force en \(O\) est %
\(\Gamma_O = \vv{OM} \wedge \vv{F} = 0\). Le théorème du moment cinétique
appliqué en \(O\) qui est fixe donne \(\derived{\vv{L_O}}{t} = \Gamma_O =
\vv{0}\), donc le moment cinétique en \(O\) est constant, \(L_O = m\vv{C}\).

\subsection{Formule de Binet}%
On utilisera les coordonnées cylindriques de \(M(r, \theta, z=0)\) avec l'axe
\(z\) dans la direction et le sens de \(\vv{C}\). On a donc~:
\[\vv{C} = C \uz = \vv{OM} \wedge \vvv = r \ur \wedge (\dot{r}\ur +
r\dot{\theta} \utheta) = r^2\dot{\theta} \uz\]
et donc \(\dot{\theta} = \dfrac{C}{r^2} = \dfrac{L_O}{mr^2}\)

Pour simplifier les expression suivantes, on pose~:
\[u = \dfrac{1}{r} ; u'=\derived{u}{\theta} ; u''=\deriveds{u}{\theta}.\]
On a donc \(\dot{\theta} = Cu^2\). On a aussi \(\dot{r} = -\dfrac{\dot{u}}{u^2}
= -\dfrac{u' \dot{\theta}}{u^2} = -Cu'\) et ainsi \(\vvv = -Cu' \ur + Cu
\utheta\). D'où la \emph{première loi de Binet}~:
\begin{equation}%
  v^2 = C^2(u^2 + u'^2)
\end{equation}%
En dérivant la vitesse, on obtient \(\vv{a} = -Cu''\dot{\theta}\ur
-Cu\dot{\theta}\ur\), les termes en \(\utheta\) s'annule puisque l'accélération
est colinéaire à \(\vv{OM}\). D'où la \emph{deuxième loi de Binet}~:
\begin{equation}%
  \vv{a} = -C^2u^2(u+u'')\ur
\end{equation}%
\section{Cas d'une force centrale conservative en \(F_r(r)\)}%
\subsection{Énergie potentielle, conservation de l'énergie mécanique}%
Dans ce case, le travail élémentaire de la force s'écrit \(\delta{}W =
\vv{F}\cdot\vv{\D{M}} = F_r(r)\ur\cdot(\D{r}\ur + r\D{\theta}\utheta) = %
F_r(r)\D{r}\). Donc \(-\D{E_p} = F_r(r)\D{r}\). L'énergie potentielle n'est
fonction que de \(r\). La constante d'intégration est choisie telle que la
limite en l'infini soit nulle. L'énergie mécanique (qui est constante puisque
la force est conservative) vaut donc~:
\[E = E_p(r) + \dfrac{1}{2} mv^2\]

\subsection{Énergie potentielle effective, énergie cinétique radiale}%
La vitesse de \(M\) est \(\vvv = \dot{r}\ur + r\dot{\theta}\utheta\) et son
carré vaut \(v^2 = \dot{r}^2 + r^2\dot{\theta}^2\) avec \(\dot{\theta} =
C/r^2\), donc \(v^2 = \dot{r}^2 + \dfrac{C^2}{r^2}\) donc l'énergie mécanique
de \(M\) s'écrit~:
\[E = \dfrac{1}{2}m\dot{r}^2 + \dfrac{mC^2}{2r^2} + E_p(r).\]
Elle est la somme de deux termes, l'énergie cinétique radiale (qui serait
l'énergie cinétique si la vitesse était radiale) \(Ec_r =
\dfrac{1}{2}m\dot{r}^2\) et l'énergie potentielle %
effective \(Ep_{\textmd{eff}} = \dfrac{mC^2}{2r^2} + E_p(r)\).

On a \(Ep_{\textmd{eff}} \leq E\).

\subsection{Différents types de mouvements}%
La condition \(Ep_{\textmd{eff}} \leq E\), si l'on connait l'expression de
\(Ep_{\textmd{eff}}\) en fonction de \(r\) permet de déterminer le type de
mouvement de \(M\) suivant la valeur de l'énergie mécanique (constante), donc
suivant la valeur initiale de \(E\) et suivant la valeur initiale de \(r\). De
toute façon, la limite de l'énergie potentielle effective en l'infini est nulle
et en zéro est infini (en général). Supposons par exemple que l'allure de la
courbe ait l'allure de la Fig.~\ref{fig:puits_potentiel}.

\begin{figure}%
  \centering
  \includegraphics[scale=0.7]{./puits_potentiel.png}
  \caption{Puits de potentiel}\label{fig:puits_potentiel}
\end{figure}%

Pour \(E = E_1\), le point \(M\) peut s'éloigner indéfiniment de \(O\), car il
n'est pas lié à \(O\). Pour \(E=E_2\), deux cas sont possibles suivant la
valeur initiale de \(r\), si le rayon initial est dans le puits, alors \(M\)
reste à une distance finie de \(O\), il est donc lié; sinon il n'est pas lié.
Pour \(E=E_3\) il est lié et pour \(E=E_0\), le point \(M\) est lié et a un
mouvement circulaire autour de \(O\).\ \emph{Si \(M\) peut s'éloigner
indéfiniment de \(O\) on parle d'état de diffusion, sinon on parle d'état lié.}

Bien entendu, un état lié n'est possible que si la force exercée par \(O\) sur
\(M\) est attractive.

\subsection{Obtention de l'équation de la trajectoire en coordonnées polaires}%
On peut utiliser la deuxième loi de Newton et la deuxième formule de Binet.
\(F_r\) étant fonction de \(r\) est aussi une fonction de \(u = 1/r\), donc
\(u\) en fonction de \(\theta\) est solution de l'équation différentielle~:
\begin{equation}%
  F_r(u) = -mC^2u^2(u+u'') \iff F_r(u) + \frac{L_0^2}{m}u^2(u+u'') = 0.
\end{equation}%
La constante \(C\) (ou la constante \(L_0\)) est déterminée avec les conditions
initiales.
\section{Cas des forces centrales en \(1/r^2\)~: interaction gravitationnelle,
interaction électrostatique}
\subsection{Expression de l'énergie potentielle, potentiel électrostatique,
potentiel gravitationnel}
Si \(M\) de masse \(m\) est attiré par la masse \(m_O\) placé en \(O\) fixe,
\(M\) subit la force d'attraction gravitationnelle~:
\begin{equation}%
  \vv{F} = -\Gb \frac{m_O m}{r^2} \er.
\end{equation}%
avec \(\er = \frac{\vv{OM}}{OM}\). Son travail élémentaire vaut \(\delta W =
-\D E_p = F_r \D r\) donc \(\D E_p = \Gb\frac{m_O m}{r^2}\) en intégrant avec
une constante d'intégration nulle (\(\lim_{r\to\infty} E_p = 0 = C\)) il vient
\begin{equation}%
  E_p = -\Gb \frac{m_O m}{r} \qquad V = \frac{E_p}{m} = -\Gb \frac{m_O}{r}
\end{equation}%
avec \(V\) le potentiel gravitationnel.

Si \(M\) de charge électrique \(q\) est attiré (ou repoussé) par la charge
\(q_O\) placée en \(O\) fixe, \(M\) subit la force électrostatique
\begin{equation}%
  \vv{F} = \frac{q_O q}{4\pi\epsilon_0r^2} \er.
\end{equation}%
avec \(\er = \frac{\vv{OM}}{OM}\). Son travail élémentaire vaut \(\delta W =
-\D E_p = F_r \D r\) donc \(\D E_p = - \frac{q_O q}{4\pi\epsilon_0r^2}\)
en intégrant avec une constante d'intégration nulle (\(\lim_{r\to\infty}
E_p = 0 = C\)) il vient
\begin{equation}%
  E_p = \frac{q_O q}{4\pi\epsilon_0 r} \qquad V = \frac{E_p}{q} =
  \frac{q_O}{4\pi\epsilon_0 r}
\end{equation}%
avec \(V\) le potentiel électrostatique.

Dans les deux cas, la force est centrale et inversement proportionnel au carré
de la distance au centre, l'énergie potentielle est de la forme \(E_p =
\frac{K}{r}\) et on a \(F_r = \frac{K}{r^2}\) avec~:%
\begin{itemize}%
  \item \(K = -\Gb m_O m < 0\) pour une interaction gravitationnelle~;
  \item \(K = \frac{q_O q}{4\pi\epsilon_0}\) pour une interaction
    électrostatique qui est négatif (attraction) si les charges sont de signes
    opposés et positif (répulsion) si les charges sont de même signe.
\end{itemize}%
\subsection{Énergie potentielle effective, différents types de mouvements}%
Dans ces deux cas, on a donc~:
\begin{equation}%
  E_{\text{p,eff}} = \frac{K}{r} + \frac{L_O^2}{2mr^2} = \frac{K}{r} +
  \frac{mC^2}{2r^2}.
\end{equation}%
Si \(K >0\), \(E_{\text{p,eff}}\) décroît de l'infini vers 0 lorsque le rayon
va de 0 à l'infini.\emph{Il n'y a que des états de diffusion.}

Si \(K < 0\), l'énergie potentielle effective passe par un minimum négatif en
\(r_0 = -\frac{mC^2}{K}\) qui vaut \(E_0 = -\frac{K^2}{2mC^2}\). On remarque
aussi que l'énergie potentielle effective s'annule pour \(r = \frac{r_0}{2}\).
Il y a donc différents cas~:
\begin{itemize}%
  \item Si \(E>0\) c'est-à-dire \(r<r_0/2\) alors il n'y a que des états de
    diffusion
  \item Si \(E<0\) il s'agit d'états liés, particulièrement en \(E = E_0\) pour
    lequel le mouvement de \(M\) est circulaire, de rayon \(r_0\).
\end{itemize}%
\subsection{Équation de la trajectoire en coordonnées polaires}%
Elle est donnée par l'équation différentielle \(F_r(u) = -mC^2u^2(u+u'')\).
Avec \(F_r = Ku^2\), l'équation différentielle s'écrit~:
\begin{equation}%
  u'' + u = \frac{-K}{mC^2}
\end{equation}%
et sa solution générale est de la forme \(u = \frac{-K}{mC^2} +
A\cos(\theta-\theta_0)\), avec le choix de \(\theta_0\) tel que \(A\) demeure
positif, d'où il vient~:
\begin{equation}%
  r = \frac{1}{u} = \frac{1}{\frac{-K}{mC^2} + A\cos(\theta - \theta_0)} =
  \frac{-\frac{mC^2}{K}}{1 - \frac{AmC^2}{K}\cos(\theta - \theta_0)}
\end{equation}%
\paragraph{Cas d'une force attractive}%
Pour une force attractive, \(K < 0\), on a affaire à une conique d'équation \(r
= \frac{p}{1 + e\cos(\theta - \theta_0)}\), avec~: le paramètre \(p =
-\frac{mC}{K}\) et l'excentricité \(e = -\frac{AmC^2}{K} = Ap\) (\(p\) et \(e\)
sont positifs). Si \(e=0\), la trajectoire est un cercle, si \(e=1\) il s'agit
d'une parabole, si \(e>1\) il s'agit d'une branche d'hyperbole. Le cercle et
les ellipses correspondent aux états liés., la parabole et la branche
d'hyperbole correspondent aux états de diffusion. La parabole correspond au cas
limite. Dans le cas de la branche d'hyperbole, il s'agit de celle pour laquelle
\(O\) est dans la concavité (on le démontre en tenant compte de \(r>0\)).
\paragraph{Cas d'une force répulsive}%
Pour une force répulsive, \(K > 0\), il ne peut y avoir que des états de
diffusion et la trajectoire ne peut être qu'une branche d'hyperbole puisque le
cas limite de la parabole n'existe pas. On a donc le paramètre \(p =
\frac{mC}{K}\) et l'excentricité \(e = \frac{AmC^2}{K} = Ap\) \(e = Ap\). %
L'équation de la trajectoire s'écrit~\(r = \frac{-p}{1 - e\cos(\theta -
\theta_0)}\). La branche d'hyperbole décrite par \(M\) est celle pour laquelle
\(O\) est du côté de sa convexité.

\emph{Dans tous les cas~: le paramètre est \(p=\frac{mC^2}{\abs{K}}\) et %
l'excentricité est \(e = Ap\)}.

\subsection{Énergie mécanique}%
L'énergie mécanique s'exprime comme la somme de l'énergie potentielle et de
l'énergie cinétique~: \[E = Ku + \frac{mC^2}{2}(u^2 + {u'}^2).\]

Si \(K<0\) alors \(mC^2 = -kP\) et donc \(u = \frac{1+e\cos(\theta -
\theta_0)}{p}\) et \(u' = \frac{-e\sin(\theta - \theta_0)}{p}\). Ainsi~:
\[E = \frac{K}{2p}\left(2+2e\cos(\theta-\theta_0) -
{(1+e\cos(\theta-\theta_0))}^2 - {(-e\sin(\theta-\theta_0))}^2\right) =
\frac{K}{2p}(1-e^2)\]%

Si \(K>0\) alors \(mC^2 = kP\) et donc \(u = \frac{1-e\cos(\theta -
\theta_0)}{-p}\) et \(u' = \frac{e\sin(\theta - \theta_0)}{-p}\). Ainsi~:
\[E = \frac{K}{2p}\left(-2+2e\cos(\theta-\theta_0) -
{(1-e\cos(\theta-\theta_0))}^2 - {(e\sin(\theta-\theta_0))}^2\right) =
\frac{K}{2p}(-1+e^2) = -\frac{K}{2p}(1-e^2)\]%

Dans les deux cas, \(E = \frac{\abs{K}}{2p}(e^2-1)\). Pour une trajectoire
circulaire ou elliptique (état lié), \(E<0\); pour une trajectoire
parabolique, \(E=0\) et pour une trajectoire hyperbolique, \(E>0\).

\section{Attraction gravitationnelle par un astre}%
\subsection{Cas général}%

Si l'astre a sa masse \(m_0\) répartie suivant une symétrie sphérique de centre
\(O\), on démontre avec le théorème de Gauss (cf.\ cours d'électrostatique) que
la force d'attraction qu'il exerce sur le point matériel \(M\) est la même que
si toute la masse \(m_O\) était concentrée en \(O\).

On supposera que les forces exercées par les autres astres sur \(M\) sont
négligeables devant celle qu'exerce \(O\) et que le repère d'origine \(O\) dont
les axes sont orientés vers des étoiles lointaines constitue un référentiel
galiléen.

On a donc~:\(K = -\Gb m_O m\). Le point \(M\) décrit alors une conique
d'équation \(r = \frac{p}{1+e\cos(\theta - \theta_0)}\) avec \(p =
\frac{C^2}{\Gb m_O}\) et \(e = \frac{AC^2}{\Gb m_O}\). Les constantes \(A\) et%
\(C\) sont déterminées par les conditions initiales sur la position et sur la
vitesse. L'énergie mécanique du point \(M\) vaut \(E = \frac{\Gb m_O
m}{2p}(e^2-1) = \frac{\Gb^2{m_O}^2m}{2C^2}(e^2-1)\). L'énergie potentielle du
point \(M\) vaut \(Ep = -\frac{\Gb m m_O (1+e\cos(\theta-\theta_0))}{p}\).
L'énergie cinétique vaut donc la différence entre l'énergie mécanique et
l'énergie potentielle~: \(Ec = \frac{\Gb m_O m}{2p}(e^2 +
2e\cos(\theta-\theta_0)+1)\)

\subsection{Trajectoire circulaire}%
Dans ce cas particulier, l'excentricité \(e\) est nulle, \(A=0\), et
\(r=p=r_0\) est constant. Si l'on note \(\vvv_0\) la vitesse initiale, on a
\(\vvv \perp \vv{OM}\)
et donc \(C = r_0 v_0 = r_0 v\) donc \(v\) est constante et vaut \(v_0\), le
mouvement est donc uniforme.
Ainsi, \(r_0 = p = \frac{C^2}{\Gb m_O} = \frac{r_0^2v_0^2}{\Gb m_O}\).

Donc pour que le mouvement soit circulaire, il faut que \(\vvv_0 \perp
\vv{OM_0}\) et que \(v_0 = \sqrt{\frac{m_O \Gb}{r_0}}\).%

Par exemple, pour un satellite artificiel d'orbite basse, \(r_O =
\qty{6.37e6}{m}\) qui est à peu près le rayon terrestre, avec \(\Gb = %
\qty{6.67e{-11}}{N.m^2.kg^{-2}}\) et \(m_O=m_T=\qty{5.97e24}{kg}\), on obtient
%
\(v_0 = \qty{7.9}{km/s}\). C'est la première vitesse cosmique~:
\begin{equation}%
  v_1 = \sqrt{\frac{m_T \Gb}{r_T}} = \qty{7.9}{km/s}
\end{equation}%

L'énergie mécanique vaut \(E = -\frac{\Gb m_O m}{2r_O}\), on a \(Ec=-E\) et
\(Ep = 2E\). Il en résulte que pour un satellite, l'atmosphère raréfié dans
laquelle il est plongé diminuant par fortement sont énergie mécanique, sa
distance à la Terre diminue, il perd de l'altitude, mais sa vitesse croît.

La période de révolution est \(T = \frac{2\pi r_0}{v_0} = 2\pi r_0
\sqrt{\frac{r_0}{m_O \Gb}}\). On a bien la troisième loi de Kepler, la loi des
%
périodes~:
\begin{equation}%
  \frac{r_0^3}{T^2} = \frac{m_O \Gb}{4\pi^2}
\end{equation}%

\subsection{Trajectoires elliptiques}%
%Voir le dessin de l'ellipse en 18.3.3
Le point \(P\) est le péri centre (périgée, périhélie, péri astre suivant
l'attracteur) et le point \(A\) est l'apocentre (apogée, aphélie ou apoastre
suivant l'attracteur). Les points \(O\) et \(O'\) sont les foyers, \(B\) et
\(M\) sont des points de l'ellipse.

Le grand axe d'une ellipse vaut \(2a\). D'après les propriétés des ellipses~:
\begin{equation}\label{eq:trajell}%
\begin{split}%
  2a & = PO + OA \\
  & = PO + PO' \\
  & = BO + BO' \\
  & = MO + MO'
\end{split}%
\end{equation}%
car l'ellipse est l'ensemble des points du plan dont la somme des distances aux
deux foyers est constante.
Pour \(\theta = \theta_0\), on a \(OP=r\), alors \(OP = \frac{p}{1+e}\) et \(OA
= r\) pour \(\theta = \theta_0 + \pi\), donc \(OA = \frac{p}{1-e}\).
D'après l'équation~\eqref{eq:trajell}, on a~:
\[2a = p\left(\frac{1}{1+e} + \frac{1}{1-e}\right) = \frac{2p}{1-e^2},\]
donc \(a = \frac{p}{1-e^2}\).
On sait aussi aussi que \(c = \Omega O = \frac{O'O}{2} = \frac{O'P - OP}{2}\),
de plus \(O'P = OA\), donc d'après les équations précédentes \(c=
\frac{p}{2}\left(\frac{1}{1-e} - \frac{1}{1+e}\right) = \frac{pe}{1-e^2} = %
ae\). Donc \(e = \frac{c}{a}\).

On a aussi \(b = \sqrt{a^2 - c^2} = a\sqrt{1-e^2} = \frac{p}{\sqrt{1-e^2}}\),
donc \(b^2 = pa\) aussi. L'aire de l'ellipse est \(S = \pi ab = \pi
a^2\sqrt{1-e^2}\).

L'énergie mécanique de \(M\) sur une trajectoire elliptique est \(E = \frac{\Gb
m_O m}{2p}(e^2-1) = -\frac{\Gb m_O m}{2a}\). La vitesse aréolaire est constante
et vaut (en norme) \(\derived{S}{t} = \frac{S}{T} = \frac{C}{2}\) avec \(p =
\frac{C^2}{\Gb m_O} = a(1-e^2)\). En réinjectant dans l'équation précédente il
%
vient donc que \(\frac{\pi a^2\sqrt{1-e^2}}{T} = \frac{\sqrt{\Gb m_O
a(1-e^2)}}{2}\). Après simplification, en élevant au carré, on obtient la
troisième loi de Kepler~:
\begin{equation}\label{eq:troisieme-loi-kepler}%
  \frac{a^3}{T^2} = \frac{m_O \Gb}{4\pi^2}
\end{equation}%
\emph{Le cube du demi grand axe de l'orbite d'une planète est proportionnel au
carré de sa période de révolution}

La première loi de Kepler affirme que les orbites sont des ellipses dont le
Soleil est un foyer et la deuxième est la loi des aires.

\subsection{Trajectoire parabolique, vitesse de libération}%

Si la trajectoire est parabolique, \(e=1\) et \(r =
\frac{p}{1+\cos(\theta-\theta_0)}\). La distance minimale d'approche est %
atteinte au péri centre~: \(OP = \frac{p}{2}\). L'énergie mécanique est \(E=0\)
donc \(Ec = -Ep = -\frac{\Gb m_O m(1+\cos(\theta - \theta_0))}{p}\). Lorsque le
point tend vers l'infini, \(Ep\) tend vers zéro et comme l'énergie mécanique
est nulle, \(Ec\) tend aussi vers zéro, et donc aussi la vitesse!
Alors que pour une trajectoire hyperbolique, la vitesse ne tend pas vers zéro.

En reprenant l'équation, on a \(\frac{mv^2}{2} = -\frac{K}{r} = \frac{\Gb m
m_O}{r}\). Pour un vaisseau spatial se trouvant près de la Terre, la vitesse
nécessaire pour avoir une trajectoire parabolique est la \emph{vitesse de
libération} ou \emph{seconde vitesse cosmique}~: \(v_{II} = \sqrt{\frac{2\Gb
m_T}{R_T}} = v_{I} \sqrt{2} = \qty{11.2}{km/s}\)

\clearpage
\section{Exercices}%
\begin{exercice}[Comète de Halley]%
  La période de la comète de Halley dans sa révolution autour du Soleil est \(T
  = \SI{76}{ans}\), la distance du périhélie au soleil est de
  \(\SI{0.59}{UA}\), une unité astronomique vaut \(SI{1}{UA} =
  \SI{1.5e11}{m}\), la masse du Soleil est d'environ \(\SI{2e30}{kg}\) et la
  constante de gravitation universelle vaut \(\Gb =
  \SI{6.67e-11}{N.m^2kg^{-2}}\).

  Calculer l'excentricité \(e\) de l'orbite de cette comète, le demi grand axe
  \(a\), les vitesses maximale \(v_M\) et minimale \(v_m\) de la comète au
  cours de son mouvement.
\end{exercice}

\begin{exercice}[Satellite artificiel]%
  Un satellite artificiel est mis en orbite en un point \(P\) à une altitude
  \(z = \SI{230}{km}\) au dessus du niveau de la mer, avec une vitesse
  \(\vvv_P\) perpendiculaire à \(OP\), \(O\) étant le centre de la Terre, de
  rayon \(R\). Avec ces conditions initiales, le satellite décrit une orbite
  elliptique caractérisée par son périgée \(OP = \SI{6630}{km}\) et son apogée
  \(OA = \SI{8250}{km}\).
  \begin{enumerate}
    \item Déterminer les caractéristiques géométriques de l'ellipse décrite par
      le satellite
    \item Calculer la constante \(C\) de la loi des aires ainsi que la période
      \(T\) de révolution du satellite.
    \item Calculer la vitesse du satellite à son apogée \(v_A\).
  \end{enumerate}
  On donne: \(\Gb = \SI{6.67e-11}{N.m^2.kg^{-2}}\), \(R = \SI{6400}{km}\),
  \(m_T = \SI{6.0e24}{kg}\) et \(v_P = \SI{8.19}{km/s}\).
\end{exercice}
\begin{exercice}[Modèle de Bohr de l'atome d'hydrogène]%
  On considère l'interaction électrostatique entre l'électron (de masse \(m\)
  et de charge \(q = -e\)) et le proton (de masse \(m' > m\) et de charge \(q =
  e\)) de l'atome d'hydrogène.
  On se place dans le modèle de Bohr où l'électron décrit une trajectoire
  circulaire de rayon \(r\) autour du proton supposé fixe et centré en \(O\).
  L'hypothèse de Bohr consiste à poser \(L_0 = n\frac{h}{2\pi}\), \(n\) étant
  entier, \(h\) la constante de Planck et \(\vv{L_0}\) le moment cinétique de
  l'électron calculé en \(O\).
  \begin{itemize}
    \item Montrer que l'énergie mécanique de l'électron peut s'écrire \(E =
      -\frac{E_0}{n^2}\). Exprimer \(E_0\) en fonction de \(m, e, h\) et
      \(\epsilon_0\).
    \item Calculer \(E_0\) en électron-volt.
  \end{itemize}
  Données~: la constante de Planck \(h = \SI{6,6262e-34}{J.s}\); la masse de
  l'électron \(m = \SI{9,10950e-31}{kg}\); la charge élémentaire
  \(e = \SI{1,6022e-19}{C}\); la permittivité du vide
  \(\epsilon_0 = \SI{8,8542e-12}{F/m}\).
\end{exercice}%
\begin{exercice}[Action des frottements sur un satelitteen orbite circulaire]%
  Soit un satellite de masse \(m\), en orbite circulaire de rayon \(R+h\). On
  assimile le champ de pesanteur au champ de gravitation dû à la Terre.
  \begin{itemize}%
    \item Déterminer la vitesse \(v\) et la période \(T\) du mouvement.
    \item Les forces de frottement (sur les couches raréfiées de l'atmosphère),
      de résultante \(\vv{f} = -\frac{m\alpha v}{h} \vvv\), produisent une très
      petite variation \(\Delta h\) de l'altitude. La trajectoire étant
      pratiquement circulaire, trouver la relation entre \(\alpha\) et
      \(\delta h\) et calculer le travail des forces de frottements \(W_f\)
      pour un tour
  \end{itemize}%
  Données~: le rayon de la Terre \(R = \frac{1}{\pi}\SI{20e3}{km}\), l'altitude
  du satellite \(h=\SI{300}{km}\), la pesanteur pour une altitude nulle
  \(g_0 = \SI{9.81}{m.s}\), la masse du satellite \(m = \SI{80}{kg}\) et la
  variation d'altitude \(\Delta h = \SI{200}{m}\).
\end{exercice}%
\begin{exercice}[Mouvement d'une planète]%
  On considère le mouvement d'une planète, assimilée à un point matériel \(P\)
  de masse \(m\), dans le champ gravitationnel du Soleil, de centre \(O\), et
  de masse \(m'\). On pose \(K = \Gb m'\) et on utilise la variable \(u=1/r\)
  avec \(r = OP\). On note \(L\) la norme du moment cinétique de \(P\) par
  rapport au point \(O\).
  \begin{enumerate}%
    \item Exprimer l'énergie mécanique \(E_m\) du point matériel \(P\). Montrer
      que le mouvement de \(P\) satisfait une équation différentielle
      \begin{equation}
        {\left(\derived{u}{\theta}\right)}^2 + u^2 -\alpha u = \beta
      \end{equation}
      où les constantes \(\alpha\) et \(\beta\) seront exprimées en fonction de
      \(K, m, L\) et \(E_m\);
    \item Résoudre cette équation différentielle. Montrer que la trajectoire de
      \(P\) est une conique dont on calculera l'excentricité \(e\) d'abord en
      fonction de \(\alpha\) et \(\beta\), puis en fonction de \(K, m, L\) et
      \(E_m\). En déduire la nature de la conique suivant le signe de \(E_m\).
  \end{enumerate}%
\end{exercice}%
\begin{exercice}[Satellite terrestre, erreur de satelisation]%
  On note \(xOy\) le plan équatorial de la Terre (centre \(O\), masse \(m_T\)).
  Un satellite \(M\) de masse \(m \ll m_T\) est lancé du point \(M_0(r=r_0, 
  \theta=0)\) de l'axe \(Ox\) à la vitesse \(\vvv_0\) faisant l'angle 
  \(\SI{90}{\degree}\) avec \((OM_0)\) et située dans le plan \(xOy\).

  On utilise les coordonnées polaires \(r=OM, \theta = (\vv{Ox}, \vv{OM})\). 
  les vecteurs de la base polaire étant \(\ex\) et \(\etheta\). On pose \(K = 
  \Gb m_T\).
  \begin{enumerate}%
    \item Calculer la constante des aires \(C\) en fonction de \(r_0\) et de 
      \(v_0\). En posant \(\eta = \frac{r_0 v_0^{2}}{K}\), montrer que 
      l'équation polaire de la conique peut se mettre sous la forme \[r = 
      \frac{\eta r_0}{1 + (\eta-1)\cos\theta}.\] Quelle est la nature de 
      conique suivant la valeur de \(\eta\)?
    \item Lorsque cette trajectoire est une ellipse, calculer les distances 
      \(r_A\) et \(r_P\) relatives à l'apogée et au périgée, puis le demi grand
      axe \(a\), en fonction de \(r_0\) et \(\eta\).
    \item On veut lancer un satellite d'orbite circulaire de rayon \(r\), sur 
      laquelle la vitesse doit être \(v\). On supposera que \(r_0 = r\) et que
      \(v_0 = v + \delta v\) avec \(\delta v \ll v\), à la suite d'une faible 
      erreur de trajectoire.

      Calculer \(\eta\), l'excentricité \(e\) de la trajectoire ainsi que 
      l'écart \(\frac{\delta T}{T}\) à la période prévue en introduisant la 
      quantité \(\epsilon = \frac{\delta v}{v}\).
  \end{enumerate}%
\end{exercice}%
\begin{exercice}[Vecteur excentricité]%
  Soit un point matériel \(M (m)\) qui subit la force gravitationnel d'un astre
  \(A\) située en \(O\) et de masse \(m_1\) du type \(\vv{f} = 
  -k\frac{\ur}{r^2}\) avec \(k > 0\).

  La position du point \(M\), dans le référentiel d'étude galiléen \(\Rep\), 
  est repérée par les coordonnées polaires \(OM = r\) et \(\theta = (\vv{Ox},
  \vv{OM})\), les vecteurs de base étant \(\ux\) et \(\utheta\).

  \begin{enumerate}%
    \item Le vecteur excentricité est défini par \[\ve = \frac{mC}{k} \vvv - 
      \vu_{\theta}\] dans laquelle \(\vvv\) est la vitesse du point \(M\) et 
      \(C\) la constante des aires.

      Montrer que le vecteur excentricité est une constante du mouvement.
    \item L'axe de symétrie de la trajectoire de \(M\) est pris comme axe 
      polaire \(Ox\). Justifier que le vecteur excentricité est colinéaire à 
      l'axe \(Oy\), orthogonal à \(Ox\).

      En effectuant le produit scalaire \(\ve \cdot \utheta\), établir 
      l'équation polaire de la trajectoire de \(M\). En déduire que l'on 
      obtient une conique d'excentricité \(e = \norme{\ve}\) et de paramètre 
      \(p = \frac{mC^2}{k}\).
    \item Montrer que \(e\) est liée à l'énergie mécanique \(E_m\) par \[e^2-1 
      = \frac{2mC^2 E_m}{k^2}.\] En déduire la nature de la conique suivant le 
      signe de \(E_m\).
  \end{enumerate}%
\end{exercice}%
% Local Variables:
% mode: latex
% TeX-master: "physique"
% End:
