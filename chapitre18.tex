\chapter{Mouvement dans un champ de forces centrales conservatives}
\minitoc{}
\minilof{}
\minilot{}

\section{Mouvement à accélération centrale}
\subsection{Définition, planéité du mouvement, loi des aires}
Un point matériel a un mouvement à accélération centrale de centre \(O\), fixe 
dans le référentiel considéré, si et seulement si son accélération est parallèle 
à son vecteur position \(\vv{OM}\) à chaque instant.

%TODO:Mettre un schéma

Alors, pour ce mouvement, on a \(\vv{OM} \wedge \vv{a} = \vv{0}\). Soit le 
vecteur \(\vv{C} = \vv{OM} \wedge \vvv\). En le dérivant, on obtient 
\(\derived{\vv{C}}{t} = \vvv \wedge \vvv + \vv{OM} \wedge \vv{a} = 
\vv{0}\), donc \(\vv{C}\) est constant (Le moment cinétique en \(O\) \(m\vv{C}\)
est constant). Le vecteur \(\vv{OM}\) reste donc perpendiculaire à un vecteur 
constant et par conséquent \emph{le mouvement est plan}.

Pendant un intervalle de temps du durée \(\D{t}\), \(\vv{OM}\) balaye une 
surface élémentaire qui est la norme du vecteur \(\D\vv{S} = 
\dfrac{\vv{OM}\wedge\D\vv{M}}{2}\), donc la ``vitesse aréolaire'' est 
\(\derived{\vv{S}}{t} = \dfrac{\vv{OM}\wedge\vvv}{2} = \dfrac{\vv{C}}{2}\).

Dans un mouvement à accélération centrale, \emph{la vitesse aréolaire est 
constante}. C'est la loi des aires. Le vecteur \(\vv{C}\) est \emph{la 
constante de la loi des aires}.

\subsection{Utilisation du théorème du moment cinétique}
On arrive aux mêmes conclusions par le raisonnement suivant~: si le mouvement 
de \(M\) est à accélération centrale, de centre \(O\), alors il est soumis à la 
force totale \(\vv{F} = m \vv{a} \parallel \vv{OM}\), c'est-à-dire à une 
\emph{force centrale (exercée par \(O\))}. Le moment de la force en \(O\) est 
\(\Gamma_O = \vv{OM} \wedge \vv{F} = 0\). Le théorème du moment cinétique 
appliqué en \(O\) qui est fixe donne \(\derived{\vv{L_O}}{t} = \Gamma_O = \vv{0}\), 
donc le moment cinétique en \(O\) est constant, \(L_O = m\vv{C}\).

\subsection{Formule de Binet}
On utilisera les coordonnées cylindriques de \(M(r, \theta, z=0)\) avec l'axe 
\(z\) dans la direction et le sens de \(\vv{C}\). On a donc~:
\[\vv{C} = C \uz = \vv{OM} \wedge \vvv = r \ur \wedge (\dot{r}\ur + 
r\dot{\theta} \utheta) = r^2\dot{\theta} \uz\]
et donc \(\dot{\theta} = \dfrac{C}{r^2} = \dfrac{L_O}{mr^2}\)

Pour simplifier les expression suivantes, on pose~:
\[u = \dfrac{1}{r} ; u'=\derived{u}{\theta} ; u''=\deriveds{u}{\theta}.\]
On a donc \(\dot{\theta} = Cu^2\). On a aussi \(\dot{r} = -\dfrac{\dot{u}}{u^2} 
= -\dfrac{u' \dot{\theta}}{u^2} = -Cu'\) et ainsi \(\vvv = -Cu' \ur + Cu 
\utheta\). D'où la \emph{première loi de Binet}~:
\begin{equation}
  v^2 = C^2(u^2 + u'^2)
\end{equation}
En dérivant la vitesse, on obtient \(\vv{a} = -Cu''\dot{\theta}\ur 
-Cu\dot{\theta}\ur\), les termes en \(\utheta\) s'annule puisque l'accélération 
est colinéaire à \(\vv{OM}\). D'où la \emph{deuxième loi de Binet}~:
\begin{equation}
  \vv{a} = -C^2u^2(u+u'')\ur
\end{equation}
\section{Cas d'une force centrale conservative en \(F_r(r)\)}
\subsection{Énergie potentielle, conservation de l'énergie mécanique}
Dans ce case, le travail élémentaire de la force s'écrit \(\delta{}W = 
\vv{F}\cdot\vv{\D{M}} = F_r(r)\ur\cdot(\D{r}\ur + r\D{\theta}\utheta) = 
F_r(r)\D{r}\). Donc \(-\D{E_p} = F_r(r)\D{r}\). L'énergie potentielle n'est 
fonction que de \(r\). La constante d'intégration est choisie telle que la 
limite en l'infini soit nulle. L'énergie mécanique (qui est constante puisque 
la force est conservative) vaut donc~:
\[E = E_p(r) + \dfrac{1}{2} mv^2\]

\subsection{Energie potentielle effective, énergie cinétique radiale}
La vitesse de \(M\) est \(\vvv = \dot{r}\ur + r\dot{\theta}\utheta\) et son 
carré vaut \(v^2 = \dot{r}^2 + r^2\dot{\theta}^2\) avec \(\dot{\theta} = 
C/r^2\), donc \(v^2 = \dot{r}^2 + \dfrac{C^2}{r^2}\) donc l'énergie mécanique 
de \(M\) s'écrit~:
\[E = \dfrac{1}{2}m\dot{r}^2 + \dfrac{mC^2}{2r^2} + E_p(r).\]
Elle est la somme de deux termes, l'énergie cinétique radiale (qui serait 
l'énergie cinétique si la vitesse était radiale) \(Ec_r = 
\dfrac{1}{2}m\dot{r}^2\) et l'énergie potentielle 
effective \(Ep_{\textmd{eff}} = \dfrac{mC^2}{2r^2} + E_p(r)\).

On a \(Ep_{\textmd{eff}} \leq E\).

\subsection{Différents types de mouvements}
La condition \(Ep_{\textmd{eff}} \leq E\), si l'on connait l'expression de 
\(Ep_{\textmd{eff}}\) en fonction de \(r\) permet de déterminer le type de 
mouvement de \(M\) suivant la valeur de l'énergie mécanique (constante), donc 
suivant la valeur initiale de \(E\) et suivant la valeur initiale de \(r\). De 
toute façon, la limite de l'énergie potentielle effective en l'infini est nulle 
et en zéro est infini (en général). Supposons par exemple que l'allure de la 
courbe ait l'allure de la Fig.~\ref{fig:puits_potentiel}.

\begin{figure}
  \centering
  \includegraphics[scale=0.7]{./puits_potentiel.png}
  \caption{Puits de potentiel}\label{fig:puits_potentiel}
\end{figure}

Pour \(E = E_1\), le point \(M\) peut s'éloigner indéfiniment de \(O\), car il 
n'est pas lié à \(O\). Pour \(E=E_2\), deux cas sont possibles suivant la 
valeur initiale de \(r\), si le rayon initial est dans le puits, alors \(M\) 
reste à une distance finie de \(O\), il est donc lié; sinon il n'est pas lié.
Pour \(E=E_3\) il est lié et pour \(E=E_0\), le point \(M\) est lié et a un
mouvement circulaire autour de \(O\). \emph{Si \(M\) peut s'éloigner 
indéfiniment de \(O\) on parle d'état de diffusion, sinon on parle d'état lié.}

Bien entendu, un état lié n'est possible que si la force exercée par \(O\) sur 
\(M\) est attractive.

\subsection{Obtention de l'équation de la trajectoire en coordonnées polaires}
On peut utliser la deuxième loi de Newton et la deuxième formule de Binet. 
\(F_r\) étant fonction de \(r\) est aussi une fonction de \(u = 1/r\), donc 
\(u\) en fonction de \(\theta\) est solution de l'équation différentielle~:
\begin{equation}
  F_r(u) = -mC^2u^2(u+u'') \iff F_r(u) + \frac{L_0^2}{m}u^2(u+u'') = 0.
\end{equation}
La constante \(C\) (ou la constante \(L_0\)) est déterminée avec les conditions 
initiales.
\section{Cas des forces centrales en \(1/r^2\)~: interaction gravitationnelle,
interaction électrostatique}
\subsection{Expression de l'énergie potentielle, potentiel électrostatique, 
potentiel gravitationnel}
Si \(M\) de masse \(m\) est attiré par la masse \(m_O\) placé en \(O\) fixe, 
\(M\) subit la force d'attraction gravitationnelle~:
\begin{equation}
  \vv{F} = -\Gb \frac{m_O m}{r^2} \er.
\end{equation}
avec \(\er = \frac{\vv{OM}}{OM}\). Son travail élémentaire vaut \(\delta W = 
-\D E_p = F_r \D r\) donc \(\D E_p = \Gb\frac{m_O m}{r^2}\) en intégrant avec 
une constante d'intégration nulle (\(\lim_{r\to\infty} E_p = 0 = C\)) il vient
\begin{equation}
  E_p = -\Gb \frac{m_O m}{r} \qquad V = \frac{E_p}{m} = -\Gb \frac{m_O}{r}
\end{equation}
avec \(V\) le potentiel gravitationnel.

Si \(M\) de charge électrique \(q\) est attiré (ou repoussé) par la charge 
\(q_O\) placée en \(O\) fixe, \(M\) subit la force électrostatique
\begin{equation}
  \vv{F} = \frac{q_O q}{4\pi\epsilon_0r^2} \er.
\end{equation}
avec \(\er = \frac{\vv{OM}}{OM}\). Son travail élémentaire vaut \(\delta W = 
-\D E_p = F_r \D r\) donc \(\D E_p = - \frac{q_O q}{4\pi\epsilon_0r^2}\)
en intégrant avec une constante d'intégration nulle (\(\lim_{r\to\infty}
E_p = 0 = C\)) il vient
\begin{equation}
  E_p = \frac{q_O q}{4\pi\epsilon_0 r} \qquad V = \frac{E_p}{q} = 
  \frac{q_O}{4\pi\epsilon_0 r}
\end{equation}
avec \(V\) le potentiel électrostatique.

Dans les deux cas, la force est centrale et inversement proportionnel au carré 
de la distance au centre, l'énergie potentielle est de la forme \(E_p = 
\frac{K}{r}\) et on a \(F_r = \frac{K}{r^2}\) avec~:
\begin{itemize}
  \item \(K = -\Gb m_O m < 0\) pour une interaction gravitationnelle~;
  \item \(K = \frac{q_O q}{4\pi\epsilon_0}\) pour une interaction 
    électrostatique qui est négatif (attraction) si les charges sont de signes
    opposés et positif (répulsion) si les charges sont de même signe.
\end{itemize}
\subsection{Énergie potentielle effective, différents types de mouvements}
Dans ces deux cas, on a donc~:
\begin{equation}
  E_{\text{p,eff}} = \frac{K}{r} + \frac{L_O^2}{2mr^2} = \frac{K}{r} + 
  \frac{mC^2}{2r^2}.
\end{equation}
Si \(K >0\), \(E_{\text{p,eff}}\) décroît de l'infini vers 0 lorsque le rayon va de 0 
à l'infini.\emph{Il n'y a que des états de diffusion.}
Si \(K < 0\), l'énergie potentielle effective passe par un minimum négatif en 
\(r_0 = -\frac{mC^2}{K}\) qui vaut \(E_0 = -\frac{K^2}{2mC^2}\). On remarque 
aussi que l'énergie potentielle effective s'annule pour \(r = \frac{r_0}{2}\).
Il y a donc différents cas~:
\begin{itemize}
  \item Si \(E>0\) c'est-à-dire \(r<r_0/2\) alors il n'y a que des états de 
    diffusion
  \item Si \(E<0\) il s'agit d'états liés, particulièrement en \(E = E_0\) pour 
    lequel le mouvement de \(M\) est circulaire, de rayon \(r_0\).
\end{itemize}
\subsection{Équation de la trajectoire en coordonnées polaires}
Elle est donnée par l'équation différentielle \(F_r(u) = -mC^2u^2(u+u'')\). 
Avec \(F_r = Ku^2\), l'équation différentielle s'écrit~:
\begin{equation}
  u'' + u = \frac{-K}{mC^2}
\end{equation}
et sa solution générale est de la forme \(u = \frac{-K}{mC^2} + 
A\cos(\theta-\theta_0)\), avec le choix de \(\theta_0\) tel que \(A\) demeure 
positif, d'où il vient~:
\begin{equation}
  r = \frac{1}{u} = \frac{1}{\frac{-K}{mC^2} + A\cos(\theta - \theta_0)} = 
  \frac{-\frac{mC^2}{K}}{1 - \frac{AmC^2}{K}\cos(\theta - \theta_0)}
\end{equation}
\paragraph{Cas d'une force attractive}
Pour une force attractive, \(K < 0\), on a affaire à une conique d'équation \(r 
= \frac{p}{1 + e\cos(\theta - \theta_0)}\), avec~: le paramètre \(p = 
-\frac{mC}{K}\) et l'excentricité \(e = -\frac{AmC^2}{K} = Ap\) (\(p\) et \(e\) 
sont positifs). Si \(e=0\), la trajectoire est un cercle, si \(e=1\) il s'agit 
d'une parabole, si \(e>1\) il s'agit d'une branche d'hyperbole. Le cercle et 
les ellipses correspondent aux états liés., la parabole et la branche 
d'hyperbole correspondent aux états de diffusion. La parabole correspond au cas 
limite. Dans le cas de la branche d'hyperbole, il s'agit de celle pour laquelle 
\(O\) est dans la concavité (on le démontre en tenant compte de \(r>0\)).
\paragraph{Cas d'une force répulsive}
Pour une force répulsive, \(K > 0\), il ne peut y avoir que des états de 
diffusion et la trajectoire ne peut être qu'une branche d'hyperbole puisque le 
cas limite de la parabole n'existe pas. On a donc le paramètre \(p = 
\frac{mC}{K}\) et l'excentricité \(e = \frac{AmC^2}{K} = Ap\) \(e = Ap\). 
L'équation de la trajectoire s'écrit~\(r = \frac{-p}{1 - e\cos(\theta - 
\theta_0)}\). La branche d'hyperbole décrite par \(M\) est celle pour laquelle 
\(O\) est du côté de sa convexité.

\emph{Dans tous les cas~: le paramètre est \(p=\frac{mC^2}{\abs{K}}\) et 
l'excentricité est \(e = Ap\)}.
\subsection{Énergie mécanique}
%%% Local Variables:
%%% mode: latex
%%% TeX-master: "physique"
%%% End:
