\chapter{Oscillateurs et portraits de phase}%
\label{chap:oscillateursetportraitdephase}%
\minitoc{}
\minilof{}
\minilot{}

\section{Oscillateur harmonique}%
\label{chap5-sec:oscillateur}%

\subsection{Mouvement d'un point matériel sur une courbe au voisinage d'une
position d'équilibre}%
\label{chap5-subsec:mvtauvoisinagedunepositiondequilibre}%

Soit \(s\) l'abscisse curviligne d'un point matériel \(M\), de masse \(m\),
mobile sur une courbe et soumis à des forces conservatives et à des forces qui
ne travaillent pas (de puissances nulles).

Soit \({E_{p}}_m\) son énergie pour une position d'équilibre stable et \(s_m\)
l'abscisse curviligne correspondante. De plus \(\derived{E_p}{s}_m=0\) et
\(\deriveds{E_p}{s}_m>0\). La formule de Taylor-Young permet d'écrire~:
\begin{equation}%
  E_p(s) = {E_p}_m + \derived{E_p}{s}_m (s-s_m) + \deriveds{E_p}{s}_m
  \frac{(s-s_m)^2}{2} + \ldots
\end{equation}%
pour \(s\) au voisinage de \(s_m\). Alors en posant \(K=\deriveds{E_p}{s}_m>0\)
on obtient
\begin{equation}%
  E_p(s) = {E_p}_m + K \frac{(s-s_m)^2}{2}.
\end{equation}%
En dérivant cette expression, on a au voisinage de \(s_m\)~:
\begin{equation}%
  \derived{E_p}{t} = K \dot{s} (s-s_m).
\end{equation}%

Mais la puissance développée par les forces conservatives est aussi la
puissance totale puisque les autres forces ne travaillent pas~:
\(\P=-\derived{E_p}{t}\). En notant \(\vF\) la résultante des forces que subit
le point matériel, on a
\begin{equation}%
  \P = \vF \cdot \vV = m \va \cdot \vV = m\ddot{s} \dot{s}.
\end{equation}%
Ainsi
\begin{equation}%
  m\ddot{s}\dot{s} = -K \dot{s}(s-s_m).
\end{equation}%
On suppose que \(\dot{s}\) n'est pas identiquement nulle, donc on obtient
l'équation différentielle suivante~:
\begin{equation}%
  \ddot{s} + \frac{K}{m}(s-s_m) = 0.
\end{equation}%
En posant \(\omega_0 = \sqrt{\frac{K}{m}}\) et en prenant l'origine des
abscisses curvilignes en \(s_m\), \emph{l'équation différentielle du mouvement
au voisinage d'une position d'équilibre stable est~:}
\begin{equation}%
  \ddot{s} + \omega_0^2 s = 0.
\end{equation}%

S'il se déplace sur un cercle de rayon \(R\), avec \(\theta=\frac{s}{R}\), elle
s'écrit aussi \(\ddot{\theta}+\omega_0^2\theta=0\). Si le point se déplace sur
une droite que l'on prend alors comme axe des \(x\), l'équation s'écrit
\(\ddot{x}+\omega_0^2x = 0\).

\subsection{Oscillateur harmonique}%
\label{chap5-subsec:oscillateur}%

Tout point matériel dont la loi horaire \(X=f(t)\) est solution de l'équation
différentielle \(\ddot{X}+\omega_0^2x=0\) est un oscillateur harmonique.

\emph{Un point matériel au voisinage d'une position d'équilibre stable est un
oscillateur harmonique.} On a déjà vu qu'un point matériel lié à un ressort
parfait dont l'autre extrémité est fixe et astreint à se déplacer sans frotter
sur la droite formée par l'axe du ressort est un oscillateur harmonique de
pulsation \(\omega_0=\sqrt{\frac{k}{m}}\) avec \(k\) la raideur du ressort et
\(m\) la masse du point. On a aussi vu qu'un pendule pesant simple est un
oscillateur harmonique, de pulsation \(\omega_0=\sqrt{\frac{g}{L}}\) avec \(g\)
l'accélération de la pesanteur et \(L\) la longueur du pendule.%

La loi horaire de l'oscillateur harmonique avec \(X=0\) à la position
d'équilibre stable s'écrit donc, avec des constantes \(A\) et \(\varphi\)~:
\(X=A\cos(\omega_0 t +\varphi)\).

\subsection{Aspect énergétique}%
\label{chap5-subsec:aspectenergetique}%

Soit un point mobile sur une courbe et dont l'abscisse curviligne est solution
de l'équation différentielle \(\ddot{s} +\omega_0^2(s-s_m)=0\) (un oscillateur!)

Sa loi horaire s'écrit donc~: \(s=s_m + A\cos(\omega_0 t+\varphi)\).On a donc,
dans le domaine où cette équation est valable,
\begin{equation}%
 \dot{s}\ddot{s} +\omega_0^2\dot{s}(s-s_m)=0.
\end{equation}%
En multipliant par \(m\) et en intégrant, on obtient
\begin{equation}%
  m \frac{\dot{s}^2}{2} + m\omega_0^2\frac{(s-s_m)^2}{2} = C^{st},
\end{equation}%
c'est-à-dire
\begin{equation}%
  E_c +f(s) = C^{st}.
\end{equation}%
La quantité \(f(s)=m\omega_0^2\frac{(s-s_m)^2}{2}+B\) (\(B\) constante) est
l'énergie potentielle du point matériel, et \(E_0=C+B\) est son énergie
mécanique (constante). En \(s_m\), \(E_p={E_p}_m=B\). Finalement
\begin{equation}%
  E_p = \frac{K}{2}(s-s_m)^2+{E_p}_m ,
\end{equation}%
avec \(K=m\omega_0^2=\deriveds{E_p}{s}\).

Si le mouvement est rectiligne, on notera \(x\) à la place de \(s\) ; et s'il
est circulaire de rayon \(R\), on notera \(s=R\theta\) et \(E_p =
\frac{KR^2}{2}(\theta-\theta_m)^2\).

L'allure de la courbe représentant \(E_p\) en fonction de \(s\), \(x\) ou
\(\theta\) est une parabole. Le point se déplace dans une cuvette de potentiel.
La condition \(E_p < E_0\) donne l'amplitude \(A\) du mouvement.

Bien entendu, un choix plus adéquat de l'origine des abscisses curvilignes
(\(s_m=0\)) et de l'origine des énergie potentielles (\({E_p}_m=0\)) ne fait
que décaler le sommet de la parabole au point \(O\) et du coup l'énergie
potentielle est transformée en \(E_p = \frac{K}{2}s^2\) et la loi horaire
\(s=A\cos(\omega_0 t +\varphi)\).

\section{Portrait de phase}%
\label{chap5-sec:portratitdephase}%
%
\subsection{Définitions, propriétés des trajectoires de phase}%
\label{chap5-subsec:trajectoiresdephase}%
%
On considère un point matériel \emph{à un seul degré de liberté}. On notera ici
\(X=s\) ou \(x\), ou \(\theta\) et \(\dot{X}\) la vitesse algébrique ou la
vitesse angulaire suivant les cas. L'état de ce point est caractérisé à un
instant donné par le \emph{point de phase} de coordonnées \(X\) et \(\dot{X}\).
Le point de phase se déplace au cours du temps dans le \emph{plan de phase} sur
sa \emph{trajectoire de phase}. L'ensemble des trajectoires de phase pour les
différentes conditions initiales est le \emph{portrait de phase} du point
matériel. Pour \(\dot{X}>0\), \(X\) est croissant et pour \(\dot{X}<0\), \(X\)
est décroissant. Ceci donne le sens de parcours du point de phase sur la
trajectoire de phase (sens horaire). D'autre part, si \(\dot{X}=0\), \(X\) est
un extrémal donc la tangente à la trajectoire de phase pour \(\dot{X}=0\) est
verticale. Un tel point de phase est appelé \emph{point de rebroussement}. De
plus
\begin{equation}%
  \derived{\dot{X}}{X} = \derived{X}{t} \derived{t}{X} =
  \frac{\ddot{X}}{\dot{X}}.
\end{equation}%
Donc la tangente à la trajectoire de phase pour \(\dot{X}=0\) est horizontale.
Un tel point de phase correspond au passage du point matériel par une position
d'équilibre. Si la trajectoire de phase est une courbe fermée, le point
matériel repasse aux mêmes positions avec la même vitesse. Il en résulte que le
mouvement est périodique et la trajectoire de phase est parcourue dans le sens
inverse du sens trigonométrique (sens horaire).
%
\subsection{Portraits de phase de l'oscillateur harmonique}%
\label{chap5-subsec:portraitdephaseoscillateurharmonique}%
%
On obtient la trajectoire de phase de l'oscillateur d'équation différentielle
\(\ddot{X}+\omega_0 X=0\), avec les conditions initiales \(X_0\) et
\(\dot{X}_0\) et de trois façons différentes~:
\begin{itemize}%
\item \emph{En résolvant l'équation différentielle}~: On a \(X=A\cos(\omega_0
  t+\varphi)\) et \(\dot{X}=-A\sin(\omega_0 t+\varphi)\). De plus
    \(X_0=A\cos\varphi\) et \(\dot{X}_0=-A\omega_0\sin\varphi\). Alors \(A =
    \sqrt{X_0^2+\frac{\dot{X}_0^2}{\omega_0^2}}\) et \(\begin{cases}
    \varphi=-\arctan\left(\frac{\dot{X}_0}{\omega_0X_0}\right) & X_0>0 \\
    \varphi=\pi-\arctan\left(\frac{\dot{X}_0}{\omega_0X_0}\right) &
    X_0<0\end{cases}\). Grâce à \(\sin^2+\cos^2=1\), on obtient l'équation de
    la trajectoire de phase~:
\begin{equation}%
  \left( \frac{\dot{X}}{\omega_0 A} \right)^2 +  \left(\frac{X}{A}\right)^2 =1.
\end{equation}%
\item \emph{Avec l'équation différentielle}~: En multipliant par \(2\dot{X}\D
  t\) les deux membres de l'équation différentielle et en intégrant entre \(0\)
    et \(t\), on obtient~:
  \begin{equation}
    \int_0^t 2\dot{X} \ddot{X}\D t + \omega_0^2 \int_0^t 2X\dot{X}\D t = 0,
  \end{equation}
  soit
  \begin{equation}
    \int_{\dot{X}_0}^{\dot{X}} 2\dot{X} \D\dot{X} + \omega_0^2 \int_{X_0}^X
    2X\D X = 0,
  \end{equation}
  c'est-à-dire
  \begin{equation}
    \dot{X}^2 + \omega_0^2 X^2 =\omega_0^2 A^2, \quad \dot{X}_0^2 + \omega_0^2
    X_0^2 = \omega_0^2 A^2.
  \end{equation}
  Finalement
  \begin{equation}
    \left( \frac{\dot{X}}{\omega_0 A} \right)^2 +  \left(\frac{X}{A}\right)^2
    =1.
  \end{equation}
\item \emph{Avec l'énergie mécanique constante}~: Si \(X\) représente \(x\) ou
  \(s\) (si c'est \(\theta\), il faut multiplier par le rayon) alors
  \begin{equation}
    \frac{m\dot{X}^2}{2} + \frac{KX^2}{2} = \frac{m\dot{X}_0^2}{2} +
    \frac{KX_0^2}{2}.
  \end{equation}
  En utilisant \(\omega_0^2=\frac{K}{m}\), on obtient \(\dot{X}^2+\omega_0^2X^2
    = \dot{X}_0^2+\omega_0^2X_0^2\) et alors
  \begin{equation}
    \left( \frac{\dot{X}}{\omega_0 A} \right)^2 +  \left(\frac{X}{A}\right)^2
    =1.
  \end{equation}
\end{itemize}%

Cette relation entre \(X\) et \(\dot{X}\) est représentée par une ellipse de
demi-axes \(A\) sur \(X\) et \(\omega_0A\) sur \(\dot{X}\). \emph{Le portrait
de phase de l'oscillateur harmonique est un ensemble d'ellipses homothétiques
dans une homothétie de centre \(O\)}.
%
\subsection{Portrait de phase du pendule pesant simple}%
\label{chap5-subsec:portraitdephasepesantsimple}%
%
\subsubsection{Définition, équation différentielle}%
%
\emph{Un pendule pesant simple est un point matériel qui oscille sur un cercle
vertical sous l'action de son poids}. Il peut s'agir d'un point matériel
suspendu à un fil inextensible, de masse négligeable dont l'autre extrémité est
fixe, à condition que la vitesse initiale ait une direction convenable. Mais
dans certaines conditions, le fil peut se détendre et le point matériel quitte
sa trajectoire circulaire \ldots On évite ce problème de la tension du fil en
considérant un point lié à un cercle vertical (anneau circulaire). On
considérera ici le cas où le point matériel n'est soumis, en plus de son poids,
qu'à la réaction normale du cercle sur lequel il se déplace (on néglige les
frottements). L'équation différentielle du mouvement s'obtient avec la
conservation de l'énergie mécanique (ou avec le principe fondamental de la
dynamique, ou encore avec le théorème du moment cinétique appliqué au point
fixe \(O\))~:%

En notant \(L\) le rayon du cercle (longueur du fil, pour le pendule simple),
le moment cinétique en \(O\) est \(\vv{\sigma_0} = \vv{OM}\wedge m \vvv =
mL^2\dot{\theta}\uz\). Donc
\(\derived{\vv{\sigma_0}}{t}=mL^2\ddot{\theta}\uz\). La somme des moments en
\(O\) des forces que subit \(M\) est~:
\(\vv{\Gamma_0}=\vv{OM}\wedge\vv{R}+\vv{OM}\wedge\vv{P}\). Avec \(\vv{R} = R_r
\ur +R_z \uz\) (pas de frottements) et \(\vv{P}=mg(\cos\theta \ur -\sin\theta
\utheta)\). Alors
\begin{align}%
  \vv{\Gamma_0} &= L \ur \wedge \left( (mg\cos\theta +R_r)\ur -
  mg\sin\theta\utheta +R_z\uz\right) \\
  & = -mgL\sin\theta\uz - LR_z\utheta.
\end{align}%
Mais \(\derived{\vv{\sigma_0}}{t} = \vv{\Gamma}_0\), donc \(R_z=0\) et
\(mL^2\ddot{\theta}=-mgL\sin\theta\). En posant \(\omega_0 =
\sqrt{\frac{g}{L}}\), on obtient l'équation différentielle du mouvement du
pendule pesant simple~:
\begin{equation}%
  \ddot{\theta} + \omega_0^2\sin\theta =0.
\end{equation}%
%
\subsubsection{Cas des petites oscillations}%
%
Pour des mouvements de faible amplitude, on peut faire l'approximation
\(\sin\theta = \theta\) et il s'agit d'un oscillateur harmonique. En notant
\(\Theta\) l'amplitude du mouvement, on obtient la loi horaire \(\theta =
\Theta\cos(\omega_0 t +\varphi)\). La trajectoire de phase est une ellipse
d'équation
\begin{equation}%
  \left(\frac{\dot{\theta}}{\omega_0 \Theta}\right)^2 +
  \left(\frac{\theta}{\Theta}\right)^2=1,
\end{equation}%
ou
\begin{equation}%
  \left(\frac{\dot{\theta}}{\omega_0}\right)^2 = \Theta^2-\theta^2
\end{equation}%
avec \(\Theta = \sqrt{\theta_0^2+\frac{\dot{\theta_0}}{\omega_0}}\).
%
\subsubsection{Cas général}%
%
Dans le cas général, il n'y a pas de solution analytique de l'équation
différentielle, mais on peut obtenir l'équation du portrait de phase en
multipliant l'équation différentielle par \(2\dot{\theta} \D t\), on obtient
\begin{equation}%
  2\dot{\theta} \D \dot{\theta} + 2\omega_0^2\sin\theta \D \theta=0.
\end{equation}%
En intégrant de \(0\) à \(t\), on a
\begin{equation}%
  \dot{\theta}^2-\dot{\theta_0}^2+2\omega_0^2(\cos\theta_0 - \cos\theta)=0,
\end{equation}%
soit encore
\begin{equation}%
  \left(\frac{\dot{\theta}}{\omega_0}\right)^2 = 2\cos\theta - 2\cos\theta_0 +
  \left(\frac{\dot{\theta_0}}{\omega_0}\right)^2.
\end{equation}%
%
La condition \(\dot\theta^2\) (équivalente à \(E_c>0\)) s'écrit
\(A<\cos\theta\) avec \(A=\cos\theta_0 -
\frac{1}{2}\left(\frac{\dot{\theta_0}}{\omega_0}\right)^2\). D'une part on
remarque que \(A<1\) et d'autre part que la condition est vérifiée pour tout
\(\theta\) si et seulement si \(A<-1\). Il peut se présenter trois cas
différents~:
\begin{itemize}%
\item Si \(-1 < A < 1\), alors il existe \(\Theta=\arccos A\) et
  \(\cos\theta>\cos\Theta\), soit \(-\Theta < \theta < \Theta\). L'angle
    \(\Theta\) est l'amplitude des oscillations périodiques du pendule.
    \(\Theta = \arccos\left(\cos\theta_0 -
    \frac{1}{2}\left(\frac{\dot{\theta_0}}{\omega_0}\right)^2\right)\) et la
    trajectoire de phase a pour équation
  \begin{equation}
    \left(\frac{\dot{\theta}}{\omega_0}\right)^2 = 2(\cos\theta - \cos\Theta).
  \end{equation}
\item Si \(A < -1\), alors l'équation de la trajectoire de phase s'écrit
  toujours
  \begin{equation}
    \left(\frac{\dot{\theta}}{\omega_0}\right)^2 = 2(\cos\theta - A),
  \end{equation}
  mais toutes les valeurs de \(\theta\) sont possibles. Il y a rotation
    toujours dans le même sens et non oscillation. Le mouvement est périodique
    bien que la trajectoire de phase ne se referme pas.
\item Si \(A = 1\), alors le point de phase se déplace sur la trajectoire de
  phase critique. Si cela était réalisable, le point matériel s'arrêterait pour
    \(\theta=\pi\), c'est-à-dire sur une position d'équilibre instable. Si cela
    se produit, c'est à cause de légers frottements inévitables que l'on a
    négligé.
\end{itemize}%
%
\section{Oscillateur amorti par frottements fluides}%
\label{chap5-sec:oscillateuramorti}%
%
\subsection{Équation différentielle}%
\label{chap5-subsec:equadiffoscillateuramorti}%
%
On considère un pendule élastique horizontal. Le point matériel \(M\), de masse
\(m\), est soumis à son poids, à la force de rappel du ressort et à la réaction
du support (composée d'une réaction normale et d'une force de frottement
fluide). En appliquant le principe fondamental de la dynamique et en projetant
sur l'axe horizontal, on obtient l'équation différentielle du mouvement~:
\begin{equation}%
  m\ddot x = -f\dot x -kx,
\end{equation}%
soit
\begin{equation}%
  \ddot x + \frac{f}{m} \dot x +\frac{k}{m}x = 0.
\end{equation}%
En l'absence de frottements, \(M\) aurait un mouvement rectiligne sinusoïdal de
pulsation \(\omega_0 = \sqrt{\frac{k}{m}}\) appelée pulsation propre. D'autre
part, on posera le coefficient d'amortissement \(\lambda = \frac{f}{2m}\) (pour
faciliter l'écriture des solutions). On réécrit l'équation différentielle~:
\begin{equation}%
  \ddot x + 2\lambda \dot x + \omega_0^2 x =0.
\end{equation}%
%
\subsection{Loi horaire du mouvement}%
\label{chap5-subsec:loihorairedumouvement}%
%
On ne résoudra ici l'équation différentielle que pour des conditions initiales
simples~:
\begin{equation}%
  x(t=0) = x_0>0 \quad \dot x(t=0)=0.
\end{equation}%
%
L'équation caractéristique associée s'écrit \(u^2+2\lambda u + \omega_0^2=0\)
et son discriminant réduit \(\delta = \lambda^2-\omega_0^2\). Trois cas
différents peuvent se présenter selon la valeur de \(\delta\).
%
\subsubsection{Régime apériodique}%
\label{chap5-subsubsec:aperiodique}%
%
Si \(\lambda>\omega_0\) (\(\delta>0\)), l'équation caractéristique a deux
solutions réelles~: \(-\lambda \pm \sqrt{\lambda^2-\omega_0^2}\), toutes deux
négatives que l'on notera \(-\alpha\) et \(-\beta\). La solution de l'équation
différentielle s'écrit
\begin{equation}%
  x = A\e^{-\alpha t} +B\e^{-\beta t}
\end{equation}%
Les conditions initiales donnent \(A=\frac{-\beta x_0}{\alpha-\beta}\) et
\(B=\frac{\alpha x_0}{\alpha-\beta}\). Alors
\begin{equation}%
  x = \frac{x_0}{\alpha - \beta}(\alpha \e^{-\beta t} - \beta \e^{-\alpha t})
  \qquad \dot x = \frac{\alpha \beta x_0}{\alpha - \beta}(\e^{-\beta t} -
  \e^{-\alpha t}).
\end{equation}%
On remarque sur la figure~\ref{fig:osc_aperiodique} que \(x\) reste
toujours positif et que \(\dot x\) reste toujours négatif.
Il n'y a pas d'oscillations. Lorsque \(t\) tend vers l'infini, \(x\)
et \(\dot x\) tendent vers \(0\).

\begin{figure}%
  \centering
  \includegraphics[scale=0.7]{Equation_horaires_aperiodique.png}%
  \caption{Oscillations apériodiques}
  \label{fig:osc_aperiodique}
\end{figure}
%
\subsubsection{Régime critique}%
\label{chap5-subsubsec:critique}%
%
Si \(\lambda = \omega_0\) (\(\delta=0\)), l'équation caractéristique a une
racine double \(\lambda=\omega_0\). La solution de l'équation s'écrit
\begin{equation}%
  x = (A+Bt)\e^{-\lambda t} \qquad \dot x = (B-\lambda A-\lambda B
  t)\e^{-\lambda t}.
\end{equation}%
Les conditions initiales donnent \(x_0=A\) et \(B=\lambda A=\lambda x_0\) donc
\begin{equation}%
  x = x_0(1+\lambda t)\e^{-\lambda t} \qquad \dot x = -\lambda^2 x_0 t
  \e^{-\lambda t}.
\end{equation}%
%
Dans ce cas, sur la figure~\ref{fig:osc_critique}, \(x\) demeure
toujours positif et \(\dot x\) demeure toujours
négatif. Il n'y a pas d'oscillations. Lorsque \(t\) tend vers l'infini \(x\) et
\(\dot x\) tendent vers zéro. Ils tendent vers zéro plus rapidement que dans le
cas apériodique.
%
\begin{figure}%
  \centering
  \includegraphics[scale=0.7]{Equation_horaires_critique.png}%
  \caption{Oscillations critiques}
  \label{fig:osc_critique}
\end{figure}
\subsubsection{Régime pseudo-périodique}%
\label{chap5-subsubsec:pseudoperiodique}%

Si \(\lambda < \omega_0\) (\(\delta < 0\)), l'équation caractéristique a deux
solutions complexes \(-\lambda \pm i\sqrt{-\delta}\). On pose \(\omega =
\sqrt{-\delta}\), soit \(\omega = \sqrt{\omega_0^2-\lambda^2}\). C'est la
pseudo-pulsation. Les solutions réelles de l'équation différentielle s'écrivent
\begin{equation}%
  x = \e^{-\lambda t}(A\cos\omega t + B\sin\omega t) \qquad \dot x =
  \e^{-\lambda t}((B\omega -\lambda A)\cos\omega t - (\lambda B + \omega
  A)\sin\omega t).
\end{equation}%
Les conditions initiales donnent \(x_0=A\) et \(0 = -\lambda A + B \omega\)
d'où \(B = \frac{\lambda}{\omega}x_0\). La solution s'écrit donc
\begin{equation}%
  x = x_0 \e^{-\lambda t}\left(\cos \omega t + \frac{\lambda}{\omega}\sin\omega
  t\right) \qquad \dot x = -\frac{\omega_0^2}{\omega}x_0 \e^{-\lambda t}\sin
  \omega t.
\end{equation}%

On peut aussi poser \(\varphi = -\arctan\left(\frac{\lambda}{\omega}\right)\),
d'où \(\sin\varphi = -\frac{\lambda}{\sqrt{\lambda^2+\omega^2}}\) et
\(\cos\varphi = \frac{\omega}{\omega_0}\). Ainsi
\begin{equation}%
  x = \frac{\omega_0}{\omega}x_0 \e^{-\lambda t} \cos(\omega t + \varphi).
\end{equation}%
%
En analysant la figure \ref{fig:osc_pseudo_periodiques}, on peut dire que
l'amplitude est \(X = \frac{\omega_0}{\omega}x_0 \e^{-\lambda
t}\), elle décroît exponentiellement. Le graphe de \(x\) en fonction du temps
oscille entre deux exponentielles \(X\) et \(-X\). On remarquera d'autre part
que la pseudo-période \(T = \frac{2\pi}{\omega} =
\frac{2\pi}{\sqrt{\omega_0^2-\lambda^2}}\) est plus grande que la période
propre \(T_0 = \frac{2\pi}{\omega_0}\), celle de l'oscillateur non amorti.
L'écart entre \(T\) et \(T_0\) croît en fonction du coefficient d'amortissement
\(\lambda\). De plus
\begin{equation}%
  \ln\left(\frac{x(t)}{x(t+T)}\right) = \ln\left(\frac{\dot x(t)}{\dot
  x(t+T)}\right) = \ln\left(\frac{\e^{x(-\lambda t)}}{\e^{x(-\lambda
  (t+T))}}\right) = \lambda T.
\end{equation}%
Le nombre \(\delta = \lambda T\) est appelé le décrément logarithmique.

\begin{figure}%
  \centering
  \includegraphics[scale=0.7]{Equation_horaires_pseudo_periodique.png}%
  \caption{Oscillations pseudo périodiques}
  \label{fig:osc_pseudo_periodiques}
\end{figure}%

\subsection{Aspect énergétique, facteur de qualité d'un oscillateur}%
\label{chap5-subsec:aspecténergétique}%
%
L'énergie mécanique de l'oscillateur est \(E(t) = \frac{k x(t)^2}{2} + \frac{m
\dot x(t)^2}{2}\). Compte tenu du décrément logarithmique \(E(t+T) = E(t)
\e^{-2\lambda T}\). L'énergie perdue par frottements en une pseudo-période est
\(E(t)-E(t+T) = E(t)(1-\e^{-2\lambda T})\). Dans le cas d'un régime
pseudo-périodique d'amortissement très faible, \(2\lambda T << 1\), on peut
utiliser un développement limité \(\e^x = 1+x\). On a donc
\begin{equation}%
  E(t)-E(t+T) = 2\lambda T E(t).
\end{equation}%
\begin{defdef}[facteur de qualité]%
  Le facteur de qualité est défini par la valeur du rapport sans dimension
  \begin{equation}
    Q = 2\pi \frac{E(t)}{E(t)-E(t+T)}
  \end{equation}
  pour un régime pseudo-périodique très peu amorti.
\end{defdef}%
%
C'est-à-dire \(Q = \frac{\omega}{2\lambda}\) mais pour un régime très peu
amorti on a \(\lambda << \omega_0\). Donc \(Q=\frac{\omega_0}{2\lambda}\). Avec
\(\omega_0=\frac{k}{m}\) et \(\lambda =\frac{f}{2m}\) on a
\(Q=\frac{\sqrt{km}}{f}\). Plus le facteur de qualité est grand, plus on se
rapproche du régime périodique d'un oscillateur harmonique. La situation
critique est telle que \(\lambda = \omega_0\), donc le facteur de qualité vaut
\(Q_{\text{critique}} = \frac{1}{2}\). On peut utiliser \(Q\) comme un
paramètre, à la place de \(\lambda\), dans l'équation différentielle~:
\(2\lambda = \frac{\omega_0}{Q}\). L'équation différentielle s'écrit donc
\begin{equation}%
  \ddot x +\frac{\omega_0}{Q} \dot x +\omega_0^2 x = 0.
\end{equation}%
%
Les tracés ci-dessous donnent \(x(t)\) (pour les conditions initiales \(x=x_0\)
et \(\dot x=0\)) en coordonnées réduites \(\xi=\frac{x}{x_0}\) et \(\tau =
\omega_0 t\). Avec ces coordonnées réduites, \(\derived{x}{t} =
\derived{x}{\tau} \derived{\tau}{t}\), c'est-à-dire \(\dot x = \omega_0 x_0
\xi'\). De même \(\ddot x = \omega_0^2 x_0 \xi''\). Alors l'équation
différentielle s'écrit
\begin{equation}%
  \xi'' +\frac{1}{Q} \xi'+\xi=0.
\end{equation}%
Le facteur de qualité \(Q\) est donc le seul paramètre. On remarque sur ces
tracés (Figure \ref{fig:portrait_de_phase}) que l'oscillateur revient le plus
rapidement possible à sa position d'équilibre pour le régime critique.

\begin{figure}%
  \centering
  \includegraphics[scale=1]{portrait_de_phase_osc_amt.png}%
  \caption{Portraits de phase pour différents \(Q\)}%
  \label{fig:portrait_de_phase}
\end{figure}
%
\subsection{Portrait de phase de l'oscillateur amorti}%
\label{chap5-subsec:portraitdephasedeloscillateuramorti}%
%
Avec un logiciel de calcul, on obtient facilement le portrait de phase
représentant la relation entre \(\dot x\) et \(x\) (ou entre \(\xi'\) et
\(\xi\) sans dimension). On a tracé ici quelques trajectoires de phase
correspondant aux mêmes conditions initiales (pour \(t=0\) \(\xi=1\) et
\(\xi'=0\)). Si \(Q\) est très grand, on retrouve la trajectoire de phase
circulaire de l'oscillateur harmonique. Pour les autres valeurs de \(Q\), le
mouvement n'est pas périodique. La trajectoire de phase n'est donc pas une
courbe fermée, elle tend vers le point de phase limite \((0,0)\). L'apparition
d'oscillations pour \(Q>\frac{1}{2}\) se traduit par un changement de signe de
\(\xi\), \ldots
%
\clearpage
\section{Exercices}%
\label{chap5-sec:exercices}%
\begin{exercice}[Ressort comprimé]%
  Un objet de masse \(m\) assimilé à un point matériel M, glisse sans
  frottement sur une tige inclinéee de l'angle \(\alpha\) par rapport au plan
  horizontal.
  Arrivé au point \(O\), origine de l'axe \(Ox\), il vient comprimer un ressort
  de longueur à vide \(L_0\) et de raideur \(k\), dont l'autre extrémité \(A\)
  est fixe. Le contact est rompu dès que le ressort reprend sa longueur à vide
  \(L_0\). Pour \(x\) négatif, \(M\) est donc soumis à une force supplémentaire
  exercée par le ressort comprimé.
  À la date \(t=0\), \(M\) est lâché sans vitesse initiale du point d'abscisse
  \(x_0=\frac{L_0}{2}\). La longueur à vide du ressort est telle que \(\frac{mg
  \sin(\alpha)}{k} = \frac{L_0}{4}\).
  \begin{enumerate}
    \item Exprimer l'énergie potentielle de \(M\) avec \(j\), \(L_0\), et \(x\)
      pour \(x\) positif puis pour \(x\) négatif.
    \item Quelle est l'abscisse de la position d'équilibre? Est-ce une
      position d'équilibre stable?
    \item Représenter l'énergie potentielle \(E_p\) en fonction de \(x\) et
      montrer sur le tracé la valeur de l'énergie mécanique \(E\) avec les
      échelles suivantes : \(\SI{8}{\centi\meter}\) pour \(L_0\) et
      \(\SI{8}{\centi\meter}\) \(\frac{k L_0^2}{4}\).
    \item Établir les équations implicites de la trajectoire de phase (relation
      entre \(x\) et \(\dot{x}\)) avec les paramètres \(k\), \(m\) et \(L_0\).

    On remplace les variables \(x\) et \(t\) par des variables adimensionnées
      \(X\) et \(\tau\) et on pose \(X' = \derived{X}{\tau}\) \(X'' =
      \deriveds{X}{\tau}\). Ceci permet d'écrire les équations de la
      trajectoire de phase sans aucun paramètre.

    Exprimer les relations entre \(x\) et \(X\), entre \(t\) et \(\tau\), entre
      \(\dot{x}\) et \(X'\) et entre \(\ddot{x}\) et \(X''\).
    Écrire les équations implicites de la trajectoire de phase avec les
      nouvelles variables. Montrer comment on vérifie les résultats de la
      question 2) sur le tracé.
  \end{enumerate}
\end{exercice}%
