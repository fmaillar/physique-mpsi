\chapter{Conducteurs ohmiques, loi d'ohm, loi de joule}
\minitoc
\minilof
\minilot

\section{Conductivité, loi d'Ohm locale}
\label{chap10-sec:conductivite}

\subsection{Conducteur ohmique, mobilité des porteurs de charge}
\label{chap10-subsec:conducteurohmique}

On considère un conducteur homogène, électriquement isotrope, immobile et à température constante et uniforme. 
Dans un tel conducteur, le mouvement d'ensemble des porteurs de charge est dû à une seule cause, le champ électrique créé par le reste du circuit.
Le conducteur étant isotrope, il n'existe aucune direction privilégiée, mise à part celle du champ électrique. La vitesse moyenne (vitesse de leur mouvement d'ensemble) de chaque type de porteurs de charge est donc parallèle à \(\vv{E}\). Les lignes du champ électrique sont donc les lignes de courant.
La \emph{mobilité} \(\mu_i\) du \(i\)\ieme type de porteurs de charge est définie par \(\vvv_i= \mu_i \vv{E}\) lorsque le régime stationnaire (ou quasi-stationnaire) est atteint. La mobilité \(\mu_i > 0\) si \(q_i > 0\) et \(\mu_i < 0\) si \(q_i < 0\).

La limitation de la vitesse moyenne d'un porteur de charge est due aux interactions avec les obstacles. Dans un conducteur métallique, ces obstacles sont les défauts du réseau cristallin et les impuretés, c'est-à-dire tout ce qui rompt la périodicité spatiale du cristal. Dans un électrolyte ce sont les ions de charge opposée et éventuellement les molécules du solvant.
La mobilité d'un porteur de charge dépend de la température. Pour un métal elle décroît si la température \(T\) croît, car les défauts cristallins sont en nombre croissant avec \(T\). Par contre, pour un électrolyte, elle croît avec \(T\).
La densité volumique de courant est \(\vj = \sum_i \rho_i \vvv_i = \sum_i \rho_i \mu_i \vv{E}\).

\subsection{Loi d'Ohm locale}
\label{sec:loidOhmlocale}

Un conducteur suit la  loi d'Ohm locale si et seulement si \(\mu_i\)  est indépendant de \(\vv{E}\).

Les métaux suivent la loi d'Ohm locale, même pour des champs électriques intenses. Dans ce cas, le conducteur considéré est un conducteur ohmique. Sa conductivité est définie par \(\rho = \sum_i \rho_i \mu_i\) . Elle est indépendante de \(\vv{E}\) et uniforme dans tout le conducteur. Elle est de plus indépendante du temps en régime stationnaire ou elle en dépend très peu en ARQS. Son unité internationale est le siemens par mètre : \(\si{S/m}\).

La loi d'Ohm locale se traduit donc par la formule \(\vj = \sigma \vv{E}\) avec \(\sigma\) constant. L'inverse de la conductivité est la résistivité , son unité internationale est l'ohm-mètre : \(\si{\ohm.m}\). Le siemens est l'inverse de l'ohm. La loi d'Ohm locale s'écrit donc encore \(\vv{E} = \rho \vj\).

\section{Résistance électrique d'un conducteur ohmique, loi d'Ohm}
\label{chap10-sec:resistanceelectrique}

\subsection{Loi d'Ohm}
\label{chap10-subsec:loidohm}

Pour un conducteur ohmique, c'est-à-dire un conducteur homogène, isotrope, immobile, en régime stationnaire ou en ARQS (donc en particulier à \(T\) constante), la loi d'Ohm s'écrit : \(u = R i\) avec \(R\) constante, si l'on utilise la convention récepteur. 
%mettre une figure

$R$ est la résistance du conducteur ohmique, son unité SI est l'ohm : \(\si{\ohm}\). Son inverse est la conductance \(G=\frac{1}{R}\). Son unité SI est le siemens \(\si{S}\). La loi d'Ohm s'écrit donc aussi \(i = G u\). La loi d'Ohm est bien entendue une conséquence de la loi d'Ohm locale~: Soit un conducteur ohmique traversé par un courant en régime stationnaire, limité par deux équipotentielles A et B. Sa surface latérale est un tube de courant puisque aucun courant n'en sort. S étant une section quelconque du conducteur, l'intensité du courant est \( i = \iint_{S} \vj \cdot \vv{\D S} = \sigma \iint_S \vj \cdot \vec{\D S}\) et la tension est \(u \int_A^B \vv{E} \cdot \vv{\D M}\). Si en chaque point du conducteur, \(\vv{E}\) est multiplié par \(k\), alors \(u\) est multiplié par \(k\) et \(\sigma\) étant inchangé, \(i\) est aussi multiplié par \(k\). Le rapport \(R = \frac{u}{i}\) reste bien constant.

\subsection{Résistance d'un conducteur ohmique élémentaire et d'un conducteur ohmique cylindrique}

Soit un tube de courant élémentaire de longueur \(\D L\) et de section \(\D S\), limité par deux équipotentielles entre lesquelles la tension est \(\D u\) et parcouru par le courant d'intensité \(\D i\). 

Le champ électrique \(\vv{E}\) est comme les lignes de courant normal aux équipotentielles et il est donc parallèle au vecteur \(\D L\) et au vecteur \(\D S\). 

On a alors \(\D i = \sigma E \D S\) et \(\D u = E \D L\) d'où la conductance de cet élément de conducteur ohmique~: \(\D G = \sigma \derived{S}{L}\).
 (c'est un infiniment petit) et sa résistance (infiniment grande)~: \(\Delta R = \rho \derived{L}{S}\).

Par intégration, on obtient la résistance d'un conducteur ohmique cylindrique limité par deux équipotentielles, si la densité volumique de courant est bien parallèle à l'axe du cylindre : \(R = \rho \frac{L}{S}\).

Si un conducteur est filiforme, de section constante, chaque petite portion du fil est assimilable à un cylindre, la formule ci-dessus s'applique aussi.

Pour une forme quelconque du conducteur ohmique, on peut calculer sa résistance ou sa conductance en le décomposant en tubes de courant élémentaires et en appliquant les lois sur les associations de résistances ou de conductances.

\section{Étude physique de la conductivité}

\subsection{Cas des métaux, des alliages métalliques}

Pour un métal pur~:
\begin{itemize}
\item aux températures ordinaires : \(\rho = \rho_0 (1 + a \theta)\), avec \(\theta\) représentant la température en \(\degrees\)C et \(a\) est de l'ordre de \(\SI{3,7e-3}{K^{-1}}\);
\item à des températures basses ($\SI{20}{K}$ à \(\SI{100}{K}\) environ) la résistivité varie plus rapidement et non linéairement.
\end{itemize}

Les alliages métalliques ont souvent des coefficients a bien plus faibles, parfois même très faibles (constantan, manganine).

\subsection{Supraconducteurs}

Il existe une température critique en dessous de laquelle la résistivité s'annule pour certains matériaux. En dessous de cette température critique, le matériau est ``supraconducteur''.

Pour les métaux, la température critique est toujours très basse~: quelques kelvins pour la plupart, mais il n'y a pas de température critique pour les meilleurs conducteurs (cuivre, argent).

\subsection{Électrolytes}

La conductivité d'un électrolyte est une fonction croissante de la température, (la mobilité des ions croît avec la température).

Soit un électrolyte dans lequel les porteurs de charge sont les ions libres \(X_i^{z_i^+}\)($z_i > 0$ ou \(z_i < 0\) suivant si l'on a affaire à un cation ou à un anion).

La charge de l'ion est \(q_i = z_i e\). La concentration molaire volumique de cet ion étant \(c_i\) (ou \([X_i^{z_i^+}]\) ), la concentration volumique de ces ions est \(n_i = N  c_i\). La densité volumique de charge mobile pour ces ions est donc \(\rho_i = N  c_i z_i e\). En utilisant la constante de Faraday \(F = N e = \SI{96,5}{kC.mol^{-1}}\), on obtient : \(\rho_i = z_i F c_i\). On a donc~: .
\begin{equation}
  \vj = \sum_{i} \rho_i \vvv_i = \sum_i z_i \mu_i c_i \vv{E}
\end{equation}

Avec la loi d'Ohm locale \(\vj = \sigma \vv{E}\), on obtient \(\sigma = \sum_i z_i \mu_i c_i\)

La conductivité molaire des ions \(X_i^{z_i^+}\)  est définie par \(\lambda_i = z_i \mu_i F\), son unité SI est : \(\si{S.m^2.mol^{-1}}\), \(\lambda_i > 0\) car \(z_i\) et \(\rho_i\) sont de même signe. D'où l'expression de la conductivité de l'électrolyte : \(\sigma = \sum_i \lambda_i c_i\). Les conductivités molaires des ions, comme leurs mobilités, dépendent de la température et des concentrations.

Pour une solution suffisamment diluée, la conductivité molaire de chaque ion tend vers une limite appelée conductivité molaire limite, notée \(\lambda^0\), qui croît avec la température. Pour une solution très diluée : 
\begin{equation}
  \sigma = \sum_i \lambda_i^0 c_i
\end{equation}

\section{Associations de résistances}

\subsection{Association en série}

Des dipôles sont en série s'ils sont traversés par le même courant. Soit \(i\) l'intensité de ce courant, \(R\) la résistance électrique du \(k\)\ieme{} conducteur et \(u_k = R_k i\) la tension entre les équipotentielles qui le limitent. La tension entre les bornes de l'association des \(n\) conducteurs ohmiques est \(u = \sum_{k=1}^n u_k = \sum_{k=1}^n R_k i\). La résistance du conducteur ohmique équivalent au groupement en série est donc \(R = \sum_{k=1}^n R_k\).

\subsection{Association en parallèle}

Des dipôles sont en parallèle s'ils sont placés entre les deux mêmes équipotentielles, donc sous la même tension. Soit \(u\) la tension entre ces deux équipotentielles, \(G_k\) la conductance électrique du \(k\)\ieme{} conducteur et \(i_k = G_k u\) l'intensité du courant qui le traverse. L'intensité du courant qui traverse l'ensemble des \(n\) conducteurs ohmiques est \(i = \sum_{k=1}^n i_k = \sum_{k=1}^n G_k u\). La conductance du conducteur ohmique équivalent au groupement en parallèle est donc \(G = \sum_{k=1}^n G_k\).

\subsection{Cas de deux conducteurs ohmiques}

Pour deux conducteurs ohmiques en série \(R=R_1+R_2\) et \(G = \frac{G_1G_2}{G_1+G_2}\).

Pour deux conducteurs ohmiques en parallèle \(R = \frac{R_1R_2}{R_1+R_2}\) et \(G=G_1+G_2\).

\subsection{Théorème de Kennely (équivalence triangle, étoile)}

On admettra qu'il y a équivalence entre les deux réseaux ci-dessous et on cherchera les relations entre les résistances de l'étoile et celles du triangle.

Si \(i_{12} = 0\), on peut supprimer la branche correspondante, entre \(A_{31}\) et \(A_{23}\) on a, d'après les lois sur les associations de résistances : 
\begin{equation}
\label{eq:Kennely-1}
R_{31} + R_{23} = \frac{R_3(R_1+R_2)}{R_1+R_2+R_3}.
\end{equation}
Par le même raisonnement, on obtient
\begin{align}
\label{eq:Kennely-2} R_{12} + R_{31} & = \frac{R_1(R_2+R_3)}{R_1+R_2+R_3}, \\ 
\label{eq:Kennely-3} R_{23} + R_{12} &= \frac{R_2(R_3+R_1)}{R_1+R_2+R_3}.
\end{align}

L'opération sur les équations précédentes : (\ref{eq:Kennely-2}+\ref{eq:Kennely-3}-\ref{eq:Kennely-1})/2 donne
\begin{equation}\label{eq:Kennely-12}
R_{12} = \frac{R_1 R_2}{R_1+R_2+R_3},
\end{equation}
et de manière similaire
\begin{align}
\label{eq:Kennely-23} R_{23} &= \frac{R_2 R_3}{R_1+R_2+R_3}, \\
\label{eq:Kennely-31} R_{31} &= \frac{R_3 R_1}{R_1+R_2+R_3}.
\end{align}

Si \(V_{31} = V_{23}\), on peut réunir les noeuds \(A_{31}\) et \(A_{23}\) (on supprime \(R_3\) où ne passe aucun courant), on a alors
\begin{equation}\
G_2 + G_3 = \frac{(G_{12} + G_{31})G_{23}}{G_{31}+G_{23}+G_{12}}.
\end{equation}
De la même manière, on a
\begin{align}
G_3+G_1 &= \frac{(G_{23} + G_{12})G_{31}}{G_{31}+G_{23}+G_{12}}, \\
G_1+G_2 &= \frac{(G_{31} + G_{23})G_{12}}{G_{31}+G_{23}+G_{12}}.
\end{align}
En manipulant ces trois équations, il vient:
\begin{align}
G_1 &= \frac{G_{12} G_{31}}{G_{12}+G_{31}+G_{23}}, \\
G_2 &= \frac{G_{23} G_{12}}{G_{12}+G_{31}+G_{23}}, \\
G_3 &= \frac{G_{31} G_{23}}{G_{12}+G_{31}+G_{23}}.
\end{align}

\section{Théorème de Millman}
Soit un noeud au potentiel \(V\) ou aboutissent~:
\begin{itemize}
	\item \(n\) branches passives, par la branche de numéro \(k\) de conductance \(G_k\) arrive au noeud considéré le courant d'intensité \(i_k\) et son autre extrémité étant au potentiel \(V_k\),
	\item \(m\) branches comportant des sources de courant, \(\eta_j\) étant l'intensité du courant électromoteur de la branche numéro \(j\) fléché vers le noeud considéré.
\end{itemize}
La loi d'Ohm donne : \(i_k = G_k(V_k-V)\). La loi des noeuds donne \(\sum_{k=1}^{n} G_k(V_k-V) + \sum_{j=1}^{m} \eta_j = 0\). D'où, en notant \(G = \sum_{k=1}^{n} G_k\), le théorème de Millman :
\begin{equation}\label{eq:Millman}
V = \sum_{k=1}^{n} \frac{G_k}{G} V_k + \sum_{j=1}^{m} \frac{\eta_j}{G}.
\end{equation}
 
Le théorème de Millman, comme on le verra en exercices, est très pratique à utiliser dans les montages à amplificateurs opérationnels.

\section{Ponts diviseur de tension et diviseur de courant}
\subsection{Pont diviseur de tension}
%Mettre les schémas

Les deux schémas sont équivalents, le second présente le diviseur de tension sous forme d'un quadripôle, le quadripôle est le réseau de quatre bornes entouré en pointillés. \(R_C\) est la résistance de charge, ou résistance utile. En son absence, on dit que le diviseur de tension est en sortie ouverte Les formules du diviseur de tension s'obtiennent facilement avec la loi d'Ohm : pour une même intensité, les tensions sont proportionnelles aux résistances.

En sortie ouverte,
\begin{equation}\label{eq:pontdiviseuru}
\frac{u_s}{u_e} = \frac{R_2}{R_1+R_2} = \frac{G_1}{G_1+G_2}.
\end{equation}
Avec une résistance de charge, il faut remplacer \(R_2\) par la résistance équivalente \(R_2 \parallel R_c\), ce qui donne~:
\begin{equation}\label{eq:diviseuru_rc}
\frac{u_s}{u_e} = \frac{R_2R_c}{R_1R_2+R_1R_c+R_2R_c} = \frac{G_1}{G_1+G_2+G_c}.
\end{equation}

\subsection{Pont diviseur de courant}
Les deux schémas sont équivalents, le second présente le diviseur de courant sous forme d'un quadripôle, le quadripôle est le réseau de quatre bornes entouré en pointillés. \(G_C\) est la conductance de charge, ou conductance utile. En son absence, on dit que le diviseur de courant est en sortie court-circuitée. Les formules du diviseur de courant s'obtiennent facilement avec la loi d'Ohm : pour une même tension, les intensités sont proportionnelles aux conductances.

En sortie court-circuitée,
\begin{equation}\label{eq:pontdiviseuri}
\frac{i_s}{i_e} = \frac{G_2}{G_1+G_2} = \frac{R_1}{R_1+R_2}.
\end{equation}
Avec une conductance de charge, il faut remplacer \(G_2\) par la conductance équivalente \(G_2\) en série avec \(R_c\), qui vaut \(\frac{G_cG_2}{G_c+G_2}\) ce qui donne~:
\begin{equation}\label{eq:diviseuri_gc}
\frac{u_s}{u_e} = \frac{G_2G_c}{G_1G_2+G_1G_c+G_2G_c} = \frac{R_1}{R_1+R_2+R_c}.
\end{equation}

\section{Loi de Joule}
\subsection{Cas d'un conducteur ohmique}

\emph{Dans un conducteur ohmique toute l'énergie électrocinétique reçue est transformée en énergie thermique. En régime stationnaire, cette énergie thermique est intégralement cédée au milieu extérieur sous forme de chaleur}, puisque la température du conducteur est fixe, de même bie sûr que l'état physique et la composition chimique du conducteur.

Par contre, en régime variable, une partie de l'énergie thermique peut être conservée par le conducteur qui voit alors sa température varier. La résistance du conducteur est alors une variable de la température\ldots{} La puissance électrocinétique reçue par un conducteur ohmique est, avec la convention récepteur \(P = ui\), avec \(u=Ri\) ou \(u=Gi\), donc \(P=R i^2 = G u^2\).

\emph{La puissance thermique produite dans un conducteur ohmique est donc~:}
\begin{equation}\label{eq:loidejoule}
\P_{th} = Ri^2=Gu^2.
\end{equation}
L'énergie thermique produite pendant un temps infinitésimal \(\D t\) est donc \(\delta W_{th} = Ri^2 \D t = Gu^2 \D t\). Si le régime est stationnaire, la chaleur reçue par le conducteur ohmique vaut \(Q = -Ri^2 t = -Gu^2 t\).

\subsection{Cas d'un dipôle quelconque}

L'effet Joule est la production d'énergie thermique dans un dipôle. Il concerne tous les dipôles. Par définition, la résistance d'un dipôle dans lequel la puissance thermique produite est \(\P_{th}\) pour une intensité \(i\) traversant ce dipôle est \(R = \frac{\P_{th}}{i^2}\). La loi de Joule s'écrit donc pour un dipôle quelconque \(\P_{th} = Ri^2\), mais elle ne s'écrit \(\P_{th} = Gu^2\) que pour un conducteur ohmique. De même, en régime stationnaire \(Q = -Ri^2 t\), mais \(Q=-Gu^2t\) n'est valable que pour un conducteur ohmique.

\section{Exercices}
\begin{exercice}[Loi de joule locale]
	\begin{enumerate}
		\item On donne la masse de l'électron \(m=\SI{9.11e-31}{\kilogram}\), la charge élémentaire \(e = \SI{1.60e-19}{\coulomb}\), la masse molaire atomique de l'argent \(M=\SI{108}{\gram\per\mole}\) et la masse volumique de l'argent \(M_V = \SI{10.5}{\gram\per\cubed\centi\meter}\). En admettant qu'il y a un électron libre par atome d'argent, calculer la concentration \(n\) des électrons libres dans ce métal.
		\item La conductivité de l'argent étant \(\sigma = \SI{61.8}{\micro\siemens\per\meter}\), déterminer la norme de la vitesse moyenne des électrons libres pour une densité de courant de norme \(j=\SI{20.0}{\ampere\per\meter\squared}\), puis la mobilité \(\mu\) de ces électrons libres dans l'argent.
		\item Exprimer avec \(j\), \(\sigma\) et \(n\) la puissance de la force exercée par le champ électrique sur un porteur de charge, puis, avec \(j\) et \(\sigma\), le rapport \(\derived{P}{V}\), \(\D P\) étant la puissance électrocinétique que reçoivent les électrons libres d'un volume élémentaire \(\D V\). Cette dernière formule exprime la loi de Joule locale.
		\item Retrouver la loi de Joule macroscopique \(P=Ri^2\) à partir de la loi de Joule locale.
	\end{enumerate}
\end{exercice}

\begin{exercice}[Association de résistances]
	\begin{enumerate}
		\item Les douze arêtes d'un cube sont constituées par des résistance identiques de valeur \(r\) chacune. Le cube est relié au circuit extérieur par deux sommets opposés. Exprimer la résistance équivalente de ce cube avec \(r\) (On utilisera des symétries pour chercher comment on divise l'intensité totale entre les différentes arêtes).
		\item On considère un grillage formé de seize carrés dont les côtés ont chacun une résistance \(r\). Ce réseau est connecté au circuit extérieur par des sommets opposés. Exprimer la résistance équivalente de ce grillage avec \(r\) (On montrera d'abord à l'aide des symétries, l’équivalence entre les deux réseaux ci-dessous, puis on utilisera les lois sur les associations de résistance et le théorème de Kennelly). 
	\end{enumerate}
\end{exercice}

\begin{exercice}[Résistance d'une prise de terre]
	Deux sphères métalliques concentriques en un point \(O\), de rayons \(r_1\) et \(r_2\), supposées parfaitement conductrices, sont séparées par un milieu peu conducteur de résistivité \(\rho\). On établit entre ces deux sphères une différence de potentiel.
	
	Quelles sont les surfaces équipotentielles et quelles sont les lignes de ce champs électrique ?
	
	Calculer la résistance \(R\) du conducteur situé entre les deux sphères. On exprimera d'abord la résistance d'un tronc de cône élémentaire, de sommet \(O\), limité par les sphères de rayon \(r\) et \(r +\D r\), que l'on assimilera à un cylindre de section \(\D S\) ; puis on effectuera les intégrations nécessaires pour calculer \(R\).
	
	Que devient \(R\) lorsque \(r_2\) tend à l'infini ?
	
	Application numérique pour \(r_1 = \SI{10}{\centi\meter}\) et \(\rho=\SI{e3}{\ohm \meter}\) (sol argileux).
\end{exercice}

\begin{exercice}[Diviseur de courant en cascade]
	\begin{enumerate}
		\item Quelle est, exprimée en avec \(G\), la conductance \(G_k\) équivalente au groupement de conducteurs ohmiques à droite des points \(A_k\) et \(B_k\) pour \(k\) allant de 1 à 3 ?
		\item Montrer que pour tout naturel non nul \(k\) on a : \(G_k = \frac{x_{k+1}}{x_k} G\), où \((x_k)\) est la suite de Fibonacci définie par \(x_2=x_1=1\) et \(x_k = x_{k-1}+x_{k-2}\) pour \(k \geq 3\).
		\item Lorsque \(k\) tend vers l'infini, le rapport \(\frac{x_{k+1}}{x_k}\) converge vers une limite finie. Calculer cette limite et donner la valeur de \(G_\infty = \lim\limits_{k \to \infty} G_k\) en fonction de \(G\).
		\item On note \(i_k\) l'intensité entrant par \(A_k\) et sortant par \(B_k\). Calculer le rapport \(\frac{i_{k-1}}{i_k}\) avec \(G\) et \(G_{k-1}\), en utilisant la formule du diviseur de courant, puis avec les termes \(x_{2k+1}\) et \(x_{2k-1}\) de la suite de Fibonacci.
		\item En déduire que le rapport \(\frac{i_n}{i_0}\) de l'intensité du courant d'entrée à l'intensité du courant de sortie duquadripôle représenté ($A_n$, \(B_n\), \(A_0\), \(B_0\)) est égal à l'un des termes de la suite de Fibonacci, lequel ? 
	\end{enumerate}
\end{exercice}
