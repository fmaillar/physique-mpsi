\chapter{Dynamique du point matériel}%
\label{chap:dynamiquedupoint}%
\minitoc{}
\minilof{}
\minilot{}

\section{Masse pesante et masse inerte}%
\label{chap2-sec:massepesantemasseinerte}%

\begin{defdef}%
  La \emph{masse pesante} est définie expérimentalement à partir
  de l'utilisation de la balance. Elle est donc liée à la notion de poids. Si
  dans une balance à deux plateaux on remplace un corps \(A\) par un corps 
  \(B\) et
  que l'équilibre est conservé, on dira par définition que \(A\) et \(B\) ont 
  la
  même masse. Si la balance reste en équilibre en replaçant \(A\) par deux 
  corps
 \(B\) et \(C\) on dira que \(m_A=m_B+m_C\). La masse est donc par définition 
  une
  \emph{grandeur extensive}, elle s'ajoute.
\end{defdef}%

La validité de cette définition nécessite que l'équilibre d'une balance soit
conservé si on change la balance de lieu, c'est-à-dire si l'intensité de la
pesanteur est modifiée, ce qui est effectivement le cas.

L'égalité de la somme de deux masses pesantes étant ainsi définies, la masse
est une grandeur mesurable. Dans le système international son unité est
\emph{le kilogramme}, \(\si{kg}\). C'est la masse du cylindre de platine 
iridié%
nommé kilogramme étalon, déposé au Bureau International de Poids et Mesures.
Cette définition du kilogramme permet des mesures avec une précision relative
de l'ordre de \(10^{-9}\).

On admettra, comme le confirme l'expérience, que la masse pesante s'identifie à
la masse inerte. La masse inerte est la grandeur scalaire extensive,
indépendante de l'état de mouvement du corps considéré, qui caractérise la
difficulté que l'on a à modifier le mouvement de ce corps. La seule théorie
capable de rendre compte de cette identité entre masse inerte et masse pesante
est la théorie de la relativité générale.


\section{Quantité de mouvement et moment cinétique}%
\label{chap2-sec:quantitedemouvementetmomentcinetique}%

\begin{defdef}%
  Pour un point matériel de masse \(m\) dans un référentiel où sa
  vitesse est \(\vv{v}\), la quantité de mouvement est définie par
  \begin{equation}
    \vv{p}=m\vv{v}.
  \end{equation}
\end{defdef}%
%
\begin{defdef}%
  En un point \(P\), le moment cinétique du point matériel \(M\)
  dans ce même référentiel est par définition le moment en \(P\) du vecteur
  quantité de mouvement en \(M\). Il est noté \(\vv{L}_P\) et vaut
  \begin{equation}
    \vv{L}_P=m\vv{PM} \wedge \vv{v}.
  \end{equation}
\end{defdef}%

\section{Principe d'inertie, référentiel galiléen}%
\label{chap2-sec:principedinertie}%

Les interactions ont une propriété commune~: elles décroissent et tendent
vers zéro lorsque la distance entre deux systèmes en interaction tend vers
l'infini et elles tendent vers l'infini lorsque cette distance tend vers
zéro.

Un point matériel ou un système situé à une distance suffisamment grande de
tout autre point matériel peut être considéré comme \emph{isolé},
c'est-à-dire sans interactions. Si un point matériel a le même comportement
mécanique qu'un point matériel isolé, on dira que les interactions qu'il
subit se compensent, on dit encore qu'il est \emph{pseudo-isolé}. Le
principe d'inertie s'énonce ainsi~: \emph{Il existe des référentiels
privilégiés dans lesquels le mouvement d'un point matériel isolé est
rectiligne et uniforme. On les appelle référentiels d'inertie ou
référentiels galiléens.}

Le principe d'inertie définit donc les référentiels galiléens et postule
leur existence. L'expérimentation conduit à constater que le référentiel de
Copernic est un référentiel galiléen. Le référentiel de Copernic a, par
définition, son origine au centre d’inertie du système solaire, ses axes
étant définis par les directions de trois étoiles très éloignées. Le
caractère galiléen du référentiel de Copernic n'est en fait qu'une
approximation, mais celle-ci reste très valable tant que l'on ne
s'intéresse pas à des mouvements de systèmes n'appartenant pas au système
solaire. Ainsi, le mouvement d'une étoile de la galaxie nécessite
d'utiliser un référentiel moins grossièrement galiléen, lié au centre
d'inertie de la galaxie et dont les axes sont dirigés vers des galaxies
lointaines \ldots

La loi de composition des vitesses montre que si un référentiel est en
translation rectiligne et uniforme par rapport à un référentiel galiléen,
ce référentiel est aussi galiléen. En effet, si est \(\vvv\) est la vitesse
d'un point matériel dans un référentiel galiléen (R) et si \(\vvv_0\) est la
vitesse de translation constante, du référentiel (R) par rapport à un autre
référentiel (\(R_0\)), alors la vitesse du point matériel considéré est
\(\vvv'=\vvv+\vvv_0\) dans le référentiel (\(R_0\)). Si \(\vvv\) est constante
(mouvement rectiligne uniforme), alors \(\vvv'\) est aussi constante. Tout
référentiel en translation rectiligne uniforme par rapport au référentiel
de Copernic est donc galiléen.

Ce n'est pas le cas d'un référentiel lié à la Terre puisque la planète
tourne par rapport aux étoiles. Cependant, pour des applications pratiques
réclamant peu de précision et concernant des objets terrestres, on admettra
que si les effets de la rotation de la terre peuvent être négligés un
référentiel terrestre peut être assimilé à un référentiel galiléen.

\section{Notion de force}%
\label{chap2-sec:notiondeforce}%

On admettra que les interactions subies par un point matériel peuvent être
représentées par des forces, caractérisées par~:
\begin{itemize}%
  \item une intensité, en Newton \(\si{N}\) dans le système international;
  \item un sens;
  \item une direction (la relation d'équipollence);
  \item un point d'application.
\end{itemize}%

Ces différentes caractéristiques d'une force peuvent être définies par les
effets statiques de la force, par exemple sur un ressort parfait \ldots Une
force est donc mathématiquement définie par un vecteur associé à un point,
(``vecteur lié'' ou ``bipoint'' représentant ce vecteur avec pour origine le
point d'application de la force). On pourra si nécessaire adopter la notation :
\((M, \vF)\). La \emph{droite d'action} d'une force est la droite ayant la
direction de la force et passant par son point d'application. On admettra enfin
que le caractère additif des vecteurs forces~: si un point matériel est soumis
aux forces \((M,\vF_1), (M,\vF_2), \ldots\) il a le même mouvement que s'il 
était
soumis à la force résultante, la somme \((M, \vF)\) avec \(\vF =
\vF_1+\vF_2+\ldots\).

Plus généralement, dans le cas de systèmes matériels, on appellera force la
grandeur vectorielle liée à un point d'application, capable de modifier le
mouvement ou la forme du système.

On notera encore que la position du point d'application d'une force n'est une
notion évidente que pour une force agissant sur un point matériel. Il n'est pas
toujours facile ni indispensable de préciser le point d'application si l'on
considère la résultante des forces correspondant à un type d'interaction subie
par un système. Par contre la droite d'action est plus facile à préciser et
plus nécessaire.

\section{Relation fondamentale de la dynamique du point matériel (deuxième loi
de Newton)}
\label{chap2-sec:relationfondamentaledeladynamique}%

\emph{Dans un référentiel galiléen, la force totale qui s'exerce sur un point
matériel est la dérivée temporelle de sa quantité de mouvement}~:
\begin{equation}%
  \vF = \derived{\vv{p}}{t}.
\end{equation}%

Cette loi reste valable, sous cette forme, en mécanique relativiste. Elle
s'applique à des points matériels, donc à des objets contenant de la matière.
La masse étant une constante, on a donc aussi \begin{equation} \vF = m
\derived{\vv{v}}{t}=m\vv{a}. \end{equation}%

D'après cette relation, on voit que la dimension de la force est
\([m.L.t^{-2}]\) donc \(\SI{1}{N}=\SI{1}{kg.m.s^{-2}}\). Le Newton est donc
l'intensité de la force agissant sur un point matériel pesant \(\SI{1}{kg}\)
subissant une accélération de \(\SI{1}{m.s^{-2}}\).

\section{Théorème du moment cinétique pour un point matériel}%
\label{chap2-sec:theoremedumomentcinetique}%

Si \(P\) est un point fixe dans le référentiel galiléen considéré~:
\begin{equation}%
  \derived{\vv{L}_P}{t} = \derived{\vv{PM}}{t} \wedge \vv{p} +\vv{PM} \wedge
  \derived{\vv{p}}{t} = \vv{PM} \wedge \vF
\end{equation}%

\emph{Dans un référentiel galiléen, la dérivée temporelle du moment cinétique
d'un point matériel \(M\) en un point fixe \(P\) vaut le moment en \(P\) de la
force qui agit en \(M\).}

\section{Principe des interactions (troisième loi de Newton)}%
\label{chap2-sec:principedesinteractions}%

\emph{Si à un instant donné, un point matériel \(A\) exerce une force
\((B,\vF_1)\) sur un point matériel \(B\), alors \(B\) exerce la force \((A, 
\vF_2\)
telle que \(\vF_1 =-\vF_2\) et de même droite d'action \((AB)\).}

Ce principe ne s'applique que si l'interaction se propage à une vitesse
infinie. Si l'on tient compte de la propagation à la célérité de la lumière
\(c\), on ne peut plus appliquer en toute rigueur ce principe si les vitesses
considérées ne sont pas négligeable devant \(c\). D'autre part, ce principe ne
s'applique qu'à des points matériels.

\section{Interactions à distance}%
\label{chap2-sec:interactionsadistance}%

\subsection{Interaction gravitationnelle}%
\label{chap2-subsec:interactiongravitationnelle}%

Elle suit la loi de Newton~: Pour deux points matériels \((A, m_A)\) et \((B,
m_B)\), c'est la force de droite d'action \((AB)\) et d'intensité \(F=\Gb
\frac{m_A m_B}{AB^2}\) avec \(\Gb = \SI{6,672e-11}{N.m^2.kg^{-2}}\) la 
constante%
de gravitation universelle. Elle est toujours attractive~: \begin{equation}
\vF_{A/B} = -\Gb \frac{m_A m_B}{AB^3}\vv{AB}. \end{equation} En un point où
la masse \(m\) subit (ou subirait si elle n'est pas présente) une force de
gravitation \(\vF\), le champ de gravitation est \(\vv{G}=\frac{\vF}{m}\).

La théorie complète qui décrit la gravitation est la relativité. Cette
interaction serait véhiculée par des particules nommées gravitons, que l'on
n'a pas encore détectés.

\subsection{Interaction électromagnétique}%
\label{chap2-subsec:interactionelectromagnetique}%

Elle concerne les charges électriques. Une charge ponctuelle \(q\) de vitesse
\(\vvv\) dans un champ électromagnétique caractérisé dans le référentiel
considéré par les vecteurs \(\vv{E}\) (le champ électrique, en 
\(\si{V.m^{-1}}\))
et \(\vv{B}\) (le champ magnétique, en tesla noté \(\si{T}\)) est soumis à la
force de Lorentz \(\vF = q(\vv{E}+\vv \wedge \vv{B})\).

En plus de l'expression de la force de Lorentz, les lois de
l'électromagnétisme comprennent les équations de Maxwell qui seront étudiées
en seconde année. La théorie complète actuelle, relativiste et quantique, est
l'électrodynamique quantique. Les particules qui véhiculent l'interaction
électromagnétiques sont les photons. Si les objets qui créent le champ sont
immobile, on parle de champ statique.

En particulier, dans le cas de deux charges ponctuelles, immobiles,
l'interaction électrostatique est donnée par la loi de Coulomb. Pour deux
charges ponctuelles \((A, q_A)\) et \((B, q_B)\) la force est telle que
\begin{equation} \vF_{A/B} = \frac{1}{4\pi \epsilon_0} \frac{q_A q_B}{AB^3}%
\vv{AB}, \end{equation} avec \(\epsilon_0 = \SI{8,8537e-12}{F.m^{-1}}\) la%
permittivité électrique du vide. Cette force est attractive lorsque les
charges sont de signes opposées et répulsive sinon.

\subsection{Interaction forte}%
\label{chap2-subsec:interactionforte}%

Elle concerne les quarks qui sont les particules fondamentales qui forment
les baryons (neutron, proton, \ldots) et les mésons. Elles ne concernent pas
les leptons (électron, positron, muon, neutrino, \ldots). Elle est liée à la
couleur des quarks. Il y a trois couleurs~: vert, bleu, et rouge et les trois
anti-couleurs correspondantes magenta (anti-vert), jaune (anti-bleu) et cyan
(anti-rouge).

Les quarks \emph{up}, noté \(u\) et de charge \(\frac{2e}{3}\), et \emph{down},
noté \(d\) et de charge \(-\frac{e}{3}\) forment les constituants des 
nucléons~:
le neutre est composé d'un up et deux downs et le proton est constitué de
deux ups et d'un down.

L'interaction forte est régie par les lois de la chromo-dynamique quantique.
Elle est véhiculée par des gluons (il en existe 8).

Dans un nucléon, les trois quarks sont de couleurs différentes et se
neutralisent comme des charges électriques opposées. Mais une molécule, bien
que neutre, peut en attirer une autre du fait du décalage entre les charges
positives et négatives (dipôle électrique); de même un nucléon, bien que
neutre du point de vue de la chromo-dynamique, attire les autres nucléons.
C'est pourquoi les nucléons peuvent se lier entre eux pour former des noyaux
atomiques plus ou moins stables. Sans l'interaction forte, les noyaux
atomiques éclateraient sous l'effet des répulsions électriques des protons.

Entre les quarks, elle semble de portée infinie comme les interactions
gravitationnelle et électromagnétique et bien plus intense, avec une intensité
qui semble croître avec la distance (ce qui empêche d'isoler un quark). Mais
entre les nucléons elle n'est que cent fois plus forte que l'interaction
électromagnétique et sa portée est faible (de l'ordre du femtométre).

\subsection{Interaction faible}%
\label{chap2-subsec:interactionfaible}%

Elle est de l'ordre de \(10^5\) fois plus faible que l'interaction forte et de
portée très faible (inférieure au femtométre). Elle intervient dans les
radioactivités \(\beta^-\), \(\beta^+\) et la capture électronique (K-capture),
c'est-à-dire~:
\begin{align}%
  d \rightarrow u + e + \bar{\nu} \quad \beta^{-},\\
  u \rightarrow d + \bar{e} + \nu \quad \beta^{+}, \\
  u+e \rightarrow d + \nu \quad  K.
\end{align}%

Elle est véhiculée par les bosons intermédiaires \(W^+\), \(W^-\) et \(Z\).

\subsection{Unification des interactions}%
\label{chap2-subsec:unification}%

La théorie électrofaible (1967) de Weinberg, Salam et Glashow unifie les
interactions électromagnétique et faible. Dans cette théorie les interactions
électromagnétique et faible apparaissent comme deux aspects d'une même
interaction plus fondamentale.

On cherche à unifier l'interaction forte avec l'interaction électrofaible,
voire avec l'interaction gravitationnelle, mais les tentatives actuelles
(théorie des cordes, relativité d'échelle, \ldots) ne sont encore que des
ébauches.

\emph{Remarque}~: Pour l'étude de la mécanique classique à laquelle on se%
limitera, les seules interactions à prendre en compte seront les interactions
gravitationnelle et électromagnétique. En effet, on ne considérera que de la
matière formée de particules ne changeant pas de nature et les interactions
fortes et faibles ont alors des portées bien trop courtes pour qu'on ait à
les prendre en considération.

\section{Interactions de contact}%
\label{chap2-sec:interactiondecontac}%

Des objets \emph{en contact} sont formés d'atomes et sont en interaction
électromagnétique (essentiellement par l'intermédiaire de leurs nuages
électroniques). Sans entrer dans les détails, on les appellera interactions
de contact.

Pour s'en tenir aux actions subies par des objets assimilés à des points
matériels, on peut envisager les cas qui suivent.

\subsection{Point matériel lié à un fil tendu de masse nulle}%
\label{chap2-subsec:filtendu}%

\emph{Un fil tendu, de masse négligeable, exerce à ses deux extrémités des
forces opposées dont la droite d'action est matérialisée par le fil. Leur
sens est évident, un fil ne peut que tirer.} En fait il s'agit d'une force
élastique liée à un léger allongement du fil.

\subsection{Point matériel liée à une extrémité d'un ressort parfait}%
\label{chap2-subsec:ressort}%

\emph{Un ressort, de masse négligeable, parfaitement élastique, exerce à ses
deux extrémités des forces opposées dont la droite d'action est confondue
avec l'axe du ressort et dont l'intensité est proportionnelle à son
allongement. Un ressort tendu tire et un ressort comprimé pousse.}

\begin{equation}%
  F_{R/M} = k\abs{L-L_0},
\end{equation}%
avec \(k\) la raideur du ressort en \(\si{N.m^{-1}}\), \(L_0\) la longueur à 
vide du
ressort et \(L\) la longueur du ressort. Ainsi si le ressort est attaché en un
point \(A\) et la force appliqué en \(M\) on a
\begin{equation}%
  \vv{F_{R/M}} = -k(L-L_0) \frac{\vv{AM}}{L}.
\end{equation}%

\subsection{Mouvement d'un point lié à une courbe ou à une surface}%
\label{chap2-subsec:mouvementlieaunecourbe}%

Si un point mobile reste en contact avec une surface \(\Sigma\) sans la
quitter, la liaison est dite bilatérale. Tout se passe comme si \(M\) était
entre deux surface parallèles \(\Sigma\) et \(\Sigma'\). La force de liaison ou
\emph{réaction de la surface} a en général deux composantes~: une normale%
(à \(\Sigma\)) et une tangentielle (à \(\Sigma\))~:
\begin{equation}%
  \vv{R} = \vv{R_n} + \vv{R_t}.
\end{equation}%
Si le point mobile peut quitter la surface sans la traverser, le contact est
unilatéral. La réaction normale est alors toujours dans le sens de \(\Sigma\) 
vers
\(M\). Elle s'annule quand \(M\) quitte \(\Sigma\) et le problème change de 
nature. En
l'absence de frottement, \(\vv{R_t}=\vv{0}\). S'il y a frottement, la réaction
tangentielle est de même direction que la vitesse et de sens opposé.

Si un point mobile reste en contact avec une courbe \(\Gamma\), sa
trajectoire est imposée. En l'absence de frottement, \(\vv{R_t}=\vv{0}\).
S'il y a frottement, la réaction tangentielle est de même direction que la
vitesse et de sens opposé.

Pour un point lié à une surface, la direction de \(\vv{R}_n\) est imposée.
Alors que pour un point lié à une courbe, elle ce l'est pas car il y a une
infinité de normales en un point d'une courbe.

\clearpage
\section{Exercices}%
\label{chap2-sec:exercices}%

\begin{exercice}[Tir dans le vide]%
  Un canon éjecte des obus avec une vitesse de norme 
  \(v_0=\SI{200}{m.s^{-1}}\).
  On désire atteindre un objectif situé au sommet d'une montagne de hauteur
 \(h=\SI{1000}{m}\). On notera \(x_m\) la distance horizontale maximale entre 
  le
  canon et l'objectif pour que ce soit possible et \(g\) l'accélération de la
  pesanteur. On suppose que le champs de pesanteur est uniforme et on néglige 
  la
  résistance de l'air ainsi que les effets de rotation de la Terre.
  \begin{enumerate}
    \item Trouver la relation entre \(x_m\), \(v_0\), \(g\) et \(h\). Calculer 
      \(x_m\) si
 \(g=\SI{10}{m.s^{-2}}\).
    \item On se place à la distance horizontale \(\frac{x_m}{2}\) de l'objectif 
      et
    on appelle \(\alpha\) l'angle de tir. Trouver la relation entre \(x_m\), 
      \(v_0\),
 \(g\), \(h\) et \(\alpha\). Calculer numériquement les valeurs de \(\alpha\) 
      en
    degrés et minutes.
  \item Calculer la vitesse de l'obus au moment où il atteint l'objectif.
  \end{enumerate}
\end{exercice}%

\begin{exercice}[Accéléromètre]%
  Montrer que l'on peut mesurer l'accélération d'un ascenseur dans les phases
  uniformément variées de son mouvement avec un ressort, une masse et un double
  décimètre.
\end{exercice}%

\begin{exercice}[Mouvement sans frottement sur une surface plane]%
  On considère le mouvement d'un point matériel glissant sans frottement sur
  un plan incliné. On notera \(\vg\) l'accélération de la pesanteur,
 \(\vv{v_0}\) la vitesse initiale, \(O\) la position initiale, \(\alpha\) le
  rectiligne du dièdre formé par le plan incliné et le plan horizontal,
 \(\beta\) l'angle de \(\vv{v_0}\) avec une horizontale du plan incliné
  (compté positivement si \(\vv{v_0}\) est au-dessus de cette horizontale).
  On utilise comme repère orthogonal~: \((Ox)\) horizontale du plan incliné
  orienté dans le sens de \(\vv{v_0}\), \((Oy)\) vers le haut dans le plan
  incliné et \((Oz)\) perpendiculaire au plan vers le haut.
  \begin{enumerate}
    \item Établir les expressions vectorielles de \(\vv{a}\), \(\vv{v}\) et
 \(\vv{OM}\) en fonction du temps \(t\), puis donner les coordonnées \(x\)
      et \(y\) de \(M\) en fonction de \(t\). En déduire l'équation cartésienne
    de la trajectoire.
    \item Exprimer la portée horizontale et la flèche de la trajectoire avec
 \(g\), \(v_0\), \(\alpha\) et \(\beta\).
    \item Donner la loi horaire du mouvement \(s=f(t)\) (la trajectoire étant 
      dans
      le sens du mouvement et l'origine des abscisses curvilignes étant le 
      point
 \(O\)).
    \item Exprimer l'accélération tangentielle et l'accélération normale, puis
      en déduire l'expression du rayon de courbure en fonction de la position 
      de
 \(M\), puis en fonction du temps \(t\).
  \end{enumerate}
\end{exercice}%

\begin{exercice}[Frottement proportionnel à la vitesse]%
  Un point matériel lancé sur une droite avec une vitesse initiale de norme 
  \(v_0
  = \SI{2}{m.s^{-1}}\) n'est soumis qu'à la force de frottement fluide \(\vv{f} 
  =
  -\lambda m \vv{v}\).
  \begin{enumerate}
    \item Le point parcourt en tout la distance \(d = \SI{100}{m}\). Calculer
      numériquement \(\lambda\).
    \item Calculer les temps \(t_1\), \(t_2\), \(t_3\) et \(t_f\) nécessaires 
      pour
      parcourir les distances respectives de \(\SI{50}{m}\), \(\SI{99}{m}\),
 \(\SI{99.9}{m}\), \(\SI{100}{m}\).
    \item Calculer le travail de la force de frottement de l'instant initial, à
 \(t_1\), à \(t_2\), à \(t_3\) puis à \(t_f\) sachant que \(m=\SI{1}{g}\).
  \end{enumerate}
\end{exercice}%

\begin{exercice}[Frottement proportionnel au carré de la vitesse]%
  Un point matériel lancé sur une droite avec une vitesse initiale de norme 
  \(v_0
  = \SI{2}{m.s^{-1}}\) n'est soumis qu'à la force de frottement fluide \(\vv{f} 
  =
  -K m v \vv{v}\). Le point met \(T=\SI{50}{s}\) pour parcourir 
  \(d=\SI{50}{m}\).
\begin{enumerate} \item Calculer numériquement \(K\), en précisant son unité.%
    \item Quel temps met le point matériel pour parcourir la distance \(2d\)?
    \item Quel est le travail de la force de frottement quand le point matériel
      a parcouru la distance \(2d\) si \(m=\SI{1}{kg}\)?
\end{enumerate}%
\end{exercice}%

\begin{exercice}[Pendule simple]%
  Un anneau, assimilé à un point matériel de masse \(m\), glisse sans frotter 
  sur
  une tige circulaire, de rayon \(r\), placée dans un plan vertical. On repère 
  sa
  position par l'angle polaire \(\theta = (\vv{OA}, \vv{OM})\). Le point \(A\) 
  étant
  la position la plus basse. Établir l'équation différentielle vérifiée par
  \(\theta\)~:
\begin{enumerate}%
  \item avec la relation fondamentale de la dynamique;
  \item avec le théorème du moment cinétique.
\end{enumerate}%
\end{exercice}%
% Local Variables:
% mode: latex
% TeX-master: "physique"
% End:
