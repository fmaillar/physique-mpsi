\chapter{Oscillations mécaniques forcées}%
\minitoc{}
\minilof{}
\minilot{}

\section{Oscillateur harmonique soumis à des frottements fluides et à une force
excitatrice sinusoïdale}
On considèrera dans ce chapitre un point matériel \(M\), de masse \(m\), mobile
sur un axe \((Ox)\), soumis à~:
\begin{itemize}%
  \item une force de rappel, due à un ressort parfait, \(-kx \ux\),
  \item une force de frottement fluide, visqueux, du type \(-\alpha \dot{x}
    \ux\), c'est-à-dire proportionnelle à la vitesse du point \(M\),
  \item une force excitatrice \(f \ux\), on n'étudiera que le cas où la force
    excitatrice est une fonction sinusoïdale de la date soit \(f = F\cos(\omega
    t + \varphi)\).
\end{itemize}%
L'élongation de \(M\) est solution d'une équation différentielle~:
\begin{equation}%
  m\ddot{x}+\alpha\dot{x}+kx = f.
\end{equation}%
La solution de l'équation différentielle est la somme de la solution générale 
de
l'équation sans second membre (notée \(x_T\)) et d'une solution particulière de
l'équation complète (notée \(x_P\)). Les constantes d'intégration intervenant
dans \(x_T\) étant déterminées par les conditions initiales (en utilisant la
solution complète \(x = x_T+x_P\)).

La solution de l'équation sans second membre \(x_T\) peut prendre différentes
forme, comme on l'a vu dans la section~\ref{sec:regpropresRLC}, suivant la 
valeur
de l'amortissement. Dans tous les cas, la solution générale tend vers 0 du fait
de la présence d'exponentielles décroissantes \ldots

On a vu en électrocinétique qu'une solution particulière de ce type d'équation
est sinusoïdale, de même pulsation que le second membre de l'équation
différentielle, soit \(x_P = X\cos(\omega t + \Psi)\). Au bout d'un temps, on
peut faire l'approximation \(x = x_P\) et l'oscillateur est en \emph{régime
sinusoïdal permanent}.

On ne cherchera ici que la solution particulière sinusoïdale, c'est-à-dire
l'amplitude et le déphasage de \(x\) par rapport à \(f\), en régime sinusoïdal
permanent.

\section{Recherche de l'amplitude et du déphasage en régime sinusoïdal
permanent}
On peut mettre l'équation différentielle sous l'une de ses formes en posant
\(\lambda = \alpha/(2m)\) le coefficient d'amortissement et
\(\omega_0=\sqrt{k/m}\) la pulsation propre, alors \(\ddot{x}+2\lambda
\dot{x}+\omega_0^2x = f/m\).%
En définissant le facteur de qualité \(Q =
\frac{m\omega_0}{\alpha}=\frac{\sqrt{km}}{\alpha}=\frac{k}{\alpha\omega_0}\) 
on%
peut aussi écrire que \(\ddot{x}+\frac{\omega_0}{Q}\dot{x}+\omega_0^2
x=\frac{f}{m}\).

La résolution utilise la même méthode qu'en électrocinétique~: on associe à
chaque grandeur fonction sinusoïdale du temps, une \emph{grandeur complexe
associée}.
\(\xc = \Xc\exp(\ii\omega t)\) avec \(\Xc = X\exp(\ii\psi)\) et \(\fc =
\Fc\exp(\ii\omega t)\) avec \(\Fc = F\exp(\ii\phi)\).

\(\xc\) est donc solution de l'équation différentielle complexe associée~:
\(\ddot{\xc}+\frac{\omega_0}{Q}\dot{\xc}+\omega_0^2 \xc=\frac{\fc}{m}\), soit
\((\ii\omega)^2\xc + (\ii\omega)\frac{\omega_0}{Q}\xc+\omega_0^2\xc =
\frac{\fc}{m}\), soit en divisant par \(\exp(\ii\omega t)\) les deux membres~:%
\begin{equation}%
  \label{eq:elongationcomplexe}
  \Xc = \frac{\Fc/m}{\omega_0^2-\omega^2+\ii\frac{\omega\omega_0}{Q}}.
\end{equation}%

On déduit donc l'amplitude de l'élongation en passant au module~:
\begin{equation}%
  X =
  \frac{F/m}{\sqrt{(\omega_0^2-\omega^2)^2+\left(\frac{\omega\omega_0}{Q}\right)^2}},
\end{equation}%
et le déphasage de l'élongation par rapport à l'abscisse de la force
excitatrice~: \(\psi-\varphi = \Phi\)~: \begin{itemize}
  \item si \(\omega<\omega_0\)~: \(\Phi =
    \arctan(\frac{\omega\omega_0}{Q(\omega^2-\omega_0^2)})\);
  \item si \(\omega>\omega_0\)~: \(\Phi = -\pi +
    \arctan(\frac{\omega\omega_0}{Q(\omega^2-\omega_0^2)})\);
  \item si \(\omega=\omega_0\)~: \(\Phi = -\frac{\pi}{2}\)
\end{itemize}%
On constate que si \(\omega = 0\), alors \(\Phi=0\) et \(X_0 =
\frac{F}{m\omega_0^2}\). En l'infini, \(X\) tend vers 0 et \(\Phi\) tend vers%
\(-\pi\).

\section{Résonance d'élongation}%
\subsection{Pulsation à la résonance, amplitude de l'élongation à la 
résonance}%

L'amplitude de l'élongation est maximale lorsque
\((\omega_0^2-\omega^2)^2+\left(\frac{\omega\omega_0}{Q}\right)^2\) est 
minimal,
donc lorsque sa dérivée (par rapport à \(\omega\)) est nulle~:
\(2(-2\omega)(\omega_0^2-\omega^2)+2\omega\frac{\omega_0^2}{Q^2} = 0\),
c'est-à-dire pour \(\omega = 0\) et pour \(\omega =
\omega_0\sqrt{1-\frac{1}{2Q^2}}\).

\emph{Il n'y a donc résonance d'élongation que si \(Q>\frac{1}{\sqrt{2}}\)} et%
la pulsation à la résonance est inférieure à la pulsation propre \(\omega_r =
\omega_0\sqrt{1-\frac{1}{2Q^2}}\). L'amplitude à la résonance d'élongation est
donc~: \(X_r = \frac{QX_0}{\sqrt{1-\frac{1}{4Q^2}}}\).

Si \(Q>\frac{1}{\sqrt{2}}\), l'amplitude maximale vaut donc l'amplitude à la
résonance : \(X_m = X_r\) ; sinon \(X_m = X_0\).

On remarque que \(X_r\) est une fonction croissante de \(Q\) et que
\(\lim\limits_{Q\to\infty} \omega_r = \omega_0\) et
\(\lim\limits_{Q\to\infty} X_r = +\infty\). D'autre part, de l'expression de
\(\omega_r\), on obtient \(Q^2 =
\frac{1}{2\left(1-\left(\frac{\omega_R}{\omega_0}\right)^2\right)}\) et en%
reportant dans l'expression de \(X_m\), on obtient après simplification~: \(X_m
= \frac{X_0}{\sqrt{1-\left(\frac{\omega_r}{\omega_0}\right)^4}}\).

L'équation de la courbe sur laquelle se trouvent les maxima de \(X\) est donc
\(X = \frac{X_0}{\sqrt{1-\left(\frac{\omega}{\omega_0}\right)^4}}\).

\subsection{Variables réduites, tracés}%
\label{subsec:varred}%
Pour les tracés, on utilise les variables réduites~: \(\xi = \frac{X}{X_0} \
\Omega = \frac{\omega}{\omega_0}\). On a alors~: \(\xi =
\left((1-\Omega^2)^2+\frac{\Omega^2}{Q^2}\right)^{-\frac{1}{2}}\) et \(\xi_m =
\frac{Q}{\sqrt{1-\frac{1}{4Q^2}}} =%
\left(\frac{1}{Q^2}-\frac{1}{4Q^4}\right)^{-\frac{1}{2}}\).

Les maxima de \(\xi\) sont sur la courbe d'équation \(\xi =
(1-\Omega^4)^{-\frac{1}{2}}\).

Si \(\Omega <1\), alors \(\Phi =
\arctan\left(\frac{1}{Q\left(\Omega-\frac{1}{\Omega}\right)}\right)\). Si
\(\Omega >1\), alors \(\Phi =
\arctan\left(\frac{1}{Q\left(\Omega-\frac{1}{\Omega}\right)}\right) - \pi\). On
pourrait encore démontrer que dans tous les cas~: \(\Phi =
\arctan\left(Q\left(\frac{1}{\Omega}-\Omega\right)\right)-\frac{\pi}{2}\).

Les courbes sont tracées sur la figure~\ref{fig:oscforcees}.

\begin{figure}[!h]%
  \centering
  \includegraphics[scale=0.7]{./Fig17.png}
  \caption{\(\xi\) et \(\Phi\) à différentes valeurs valeurs du facteur de
  qualité}
  \label{fig:oscforcees}
\end{figure}%

\subsection{Bande passante à \SI{-3}{\deci\bel}}%

Pour \(Q \leq \frac{1}{\sqrt{2}}, \xi_m = 1\), donc à l'unique coupure, on a
\(\left(1-\Omega_c^2)^2+\frac{\Omega^2}{Q^2}\right)^{-0.5} = 2^{-0.5}\). On
obtient \(\Omega_C = \sqrt{1 - \frac{1}{2Q^2} + \sqrt{2 - \frac{1}{Q^2} +
\frac{1}{4Q^4}}}\).%

Pour \(Q \geqslant \frac{1}{\sqrt{2}}\), les coupures correspondent à \(\xi_C =
\frac{\xi_M}{\sqrt{2}} = \left(\frac{2}{Q^2}-\frac{1}{2Q^4}\right)^{-0.5}\), 
les%
pulsations réduites aux coupures sont donc solutions de l'équation
\(1-\Omega^2)^2 +\frac{\Omega^2}{Q^2} = \frac{2}{Q^2} - \frac{1}{2Q^4}\), dont
la résolution donne les résultats suivants~:
\begin{itemize}%
  \item Si \(Q \geqslant \sqrt{1 + \frac{1}{\sqrt{2}}}\) il y a deux coupures,
    pour \(\Omega_C = \sqrt{1-\frac{1}{2Q^2} \pm \sqrt{\frac{1}{Q^2} -
    \frac{1}{Q^4}}}\) ;
  \item Si \(\frac{1}{\sqrt{2}} \leqslant Q \leqslant \sqrt{1 +
    \frac{1}{\sqrt{2}}}\) il n'y a qu'une coupure,
    pour \(\Omega_C = \sqrt{1-\frac{1}{2Q^2} + \sqrt{\frac{1}{Q^2} -
    \frac{1}{Q^4}}}\).
\end{itemize}%
On remarque que pour \(Q = \frac{1}{\sqrt{2}}\), \(\Omega_C = 1\) (donc
\(\omega_C = \omega_0\). On remarque aussi que \emph{si \(\Q\) est très grand},
les pulsations de coupures peuvent être obtenues avec un développement limité~:
\(\Omega_1 \sim \sqrt{1-\frac{1}{Q}} \sim 1-\frac{1}{2Q}\) et \(\Omega_2 \sim
\sqrt{1+\frac{1}{Q}} \sim 1+\frac{1}{2Q}\). La bande passante est donc, en%
pulsation réduite, \(\Delta\Omega \approx \frac{1}{Q}\), ou en pulsation~:
\begin{equation}%
  \Delta\omega \approx \frac{\omega_0}{Q} = \frac{\alpha}{m} \quad Q \approx
  \frac{\omega_0}{\Delta\omega}=\frac{N_0}{\Delta N}.
\end{equation}%

\section{Résonance de vitesse}%

On rappelle l'équation \eqref{eq:elongationcomplexe}, \(\Xc =
\frac{\Fc/m}{\omega_0^2-\omega^2+\ii\frac{\omega\omega_0}{Q}}\). En posant \(v 
=%
\dot{x}\) : abscisse de la vitesse, on a \(\Vitc = \ii \xc\) et on en déduit 
que%
\(\Vitc = \ii\omega \Xc\), soit~:
\begin{equation}%
  \Vitc = \frac{\ii\omega 
  \Fc/m}{\omega_0^2-\omega^2+\ii\frac{\omega\omega_0}{Q}}
  = \frac{\frac{QF}{m\omega_0}}{1+\ii Q\left(\Omega-\frac{1}{\Omega}\right)}.
\end{equation}%
La deuxième égalité s'obtient en multipliant le numérateur et le dénominateur
par \(\frac{Q}{\ii\omega\omega_0}\).

L'amplitude de vitesse est donc~:
\begin{equation}%
  V =
  \frac{\frac{QF}{m\omega_0}}{\sqrt{1+Q^2\left(\Omega-\frac{1}{\Omega}\right)^2}}.
\end{equation}%
Elle est maximale pour \(\Omega = 1\), soit \(\omega = \omega_0\), il y a alors
résonance de vitesse en cette pulsation et ce maximum vaut \(V_M =
\frac{QF}{m\omega_0}\).%

Aux coupures, la vitesse vaut \(V_C = \frac{V_M}{\sqrt{2}}\), donc les
pulsations de coupures sont données par l'équation \(\Omega - \frac{1}{\Omega} 
=
\pm \frac{1}{Q}\), soient \(\Omega_1 = -\frac{1}{2Q} + 
\sqrt{\frac{1}{4Q^2}+1}\)
et \(\Omega_2 = +\frac{1}{2Q} + \sqrt{\frac{1}{4Q^2}+1}\). La bande passante,
pour l'amplitude de la vitesse, en pulsation réduite est donc \(\Delta\Omega =
\frac{1}{Q}\), ou en pulsation \(\Delta\omega = \frac{\omega_0}{Q}\), ou en%
fréquence \(\Delta N = \frac{N_0}{Q}\). \emph{Le facteur de qualité exprime 
donc
l'acuité de la résonance de vitesse}~: \(Q = \frac{\omega_0}{\Delta\omega} =
\frac{N_0}{\Delta N}\).%

La courbe de résonance de vitesse peut être tracée avec les variables réduites
\(\Omega\) et \(\xi = \frac{V}{V_M}\), son équation est~:
\begin{equation}%
  \xi = \frac{1}{\sqrt{Q^2\left(\Omega - \frac{1}{\Omega}\right)^2 + 1}}
\end{equation}%
et le déphasage de l'abscisse \(v\) de la vitesse par rapport à celle de la
force excitatrice \(f\) est~:
\begin{equation}%
  \Psi = \arctan\left(Q\left(\frac{1}{\Omega} - \Omega\right)\right).
\end{equation}%
On remarque que \(\Vitc = \ii\omega \xc\), donc que \(v\) est déphasée de
\(\frac{\pi}{2}\) par rapport à \(x\), ce qui justifie l'expression de \(\Phi\)
dans la section~\ref{subsec:varred}.

Aux coupures, on a \(\Psi_1 = \frac{\pi}{4}\) et \(\Psi_2 = -\frac{\pi}{4}\).

\begin{figure}[!h]%
  \centering
  \includegraphics[scale=0.7]{./Fig17-2.png}
  \caption{\(\xi\) et \(\Psi\) à différentes valeurs valeurs du facteur de
  qualité}
  \label{fig:oscforceesvit}
\end{figure}%

Les tracés pour \(\xi\) et \(\psi\) correspondent aux mêmes valeurs de \(Q\) 
que
pour les tracés de résonance d'élongation~\ref{subsec:varred}, affichés sur la
figure~\ref{fig:oscforceesvit}. Le tracé de \(\xi\) est trompeur, car \(V_M\)
est proportionnel à \(Q\), donc \(\xi\) ne donne pas une idée claire des maxima
de \(V\). Si l'on traçait \(V\) en fonction de \(\Omega\), on aurait des maxima
d'autant plus grand que \(Q\) est grand. L'avantage de notre tracé est que tous
les maxima sont ramenés à la valeur \(1\) et qu'on peut donc repérer le 
coupures
et les bandes passantes, quel que soit \(Q\), sur la même horizontale 
d'ordonnée
\(\frac{1}{\sqrt{2}}\).

\section{Résonance de puissance}%

La puissance développée par la force excitatrice est~:
\begin{equation}%
  p = \vec{f} \cdot \vv = F\cos(\omega t +\varphi) V\cos(\omega t + \varphi + 
  \Psi)
  = \frac{FV}{2}\left(\cos(2\omega t + \varphi + \Psi) + \cos \Psi\right)
\end{equation}%
Sa valeur moyenne sur une période est la \og{}puissance moyenne\fg{}~:\(P =
\frac{FV\cos\Psi}{2}\). Avec \(\Psi = \arctan(Q(1/\omega - \Omega))\), on%
obtient~: \(\cos \Psi =
\frac{1}{\sqrt{Q^2\left(\Omega-\frac{1}{\Omega}\right)^2+1}} = \xi\), et \(V =%
\frac{\frac{QF}{m\omega_0}}{\sqrt{Q^2\left(\Omega-\frac{1}{\Omega}\right)^2+1}}\),%
donc
\begin{equation}%
  P =
  \frac{QF^2}{2m\omega_0\left(Q^2\left(\Omega-\frac{1}{\Omega}\right)^2+1\right)}.
\end{equation}%
La puissance maximale est obtenue pour \(\Omega = 1, \omega = \omega_0\), il y 
a
alors \emph{résonance de puissance}, alors \(P_M = \frac{QF^2}{2m\omega_0} =
\frac{F^2}{2\alpha}\). On a donc \(\tau = \frac{P}{P_M} =%
\frac{1}{Q^2\left(\Omega-\frac{1}{\Omega}\right)^2+1} = \xi^2\).%

Le tracé de \(\tau\) en fonction de \(\Omega\) pour différents \(Q\) (les mêmes
que précédemment) est donné par la figure~\ref{fig:oscforceespuis}.

\begin{figure}[!h]%
  \centering
  \includegraphics[scale=0.7]{./Fig17-3.png}
  \caption{\(\tau\) à différentes valeurs valeurs du facteur de qualité}
  \label{fig:oscforceespuis}
\end{figure}%

Aux coupures en puissance ce maximum est divisé par 2. Les coupures
correspondent donc à \(Q^2\left(\Omega-\frac{1}{\Omega}\right)^2 = 1\) comme
pour la résonance de vitesse.

\emph{La pulsation de résonance de puissance, les pulsations de coupure en
puissance et la bande passante correspondante sont donc les mêmes que pour
l'amplitude de la vitesse}.

\section{Exemple d'oscillations sinusoïdales forcées}%

On considère un ressort parfait \((k, L_0)\), enfilé sur une tige horizontale,
dont une extrémité \(A\) est actionnée par un moteur qui lui communique un
mouvement sinusoïdal. L'autre extrémité du ressort est lié au point matériel
\(M(m)\) qui est soumis à une force de frottement fluide \(-\alpha \vec{v}\).

Soit \(\overline{A_0O} = L_0\), \(\overline{A_0A} = x_A = X_A\cos(\omega t +
\varphi)\) (\(\omega\) est ici la vitesse de rotation du moteur) et
\(\overline{OM} = x\), la longueur du ressort est \(L = L_0 + x - x_A\) et la
force qu'il exerce sur \(M\) est \(-k(L-L_0)\ux = -k(x-x_A)\ux\). La force de
frottement fluide est \(-\alpha \dot{x}\ux\).

La relation fondamentale de la dynamique donne \(m \ddot{x} = -\alpha \dot{x}
-k(x-x_A)\), soit \(m\ddot{x} +\alpha \dot{x} + kx = f\), avec \(f =
F\cos(\omega t +\varphi)\) si l'on pose \(F=kX_A\).

Tout se passe donc comme si \(M\) était soumis à une force supplémentaire
sinusoïdale et si \(A\) était fixe.

\section[Analogie mécanique/électrique]{Analogie entre les oscillations
sinusoïdales forcées en mécanique et en électricité}

\subsection{équivalences entre grandeurs mécaniques et électriques}%

Pour un dipôle RLC série branché sur une source de tension, l'équation
différentielle qui régit le système s'écrit~: \(L \ddot{q} + R\dot{q} + q/C =
e\) avec \(q\) la charge électrique du condensateur.

Pour un point matériel de masse \(m\) lié à un ressort et soumis à une force
excitatrice \(f\), avec frottement fluide \(-\alpha \vec{v}\), l'équation
différentielle qui régit le système s'écrit~:\(m\ddot{x}+\alpha \dot{x}+k x =
f\). On a les équivalence suivantes dans le tableau~\ref{tab:equiv}.

\begin{table}[h]%
  \centering
  \begin{tabular}{||c|c|c||}
    \hline
    Electrique & Mecanique & Interprétation\\
    \hline
    \(L\) & \(m\) & Facteur d'inertie\\
    \(R\) & \(\alpha\) & Facteur d'amortissement\\
    \(\frac{1}{C}\) & \(k\) & Raideur d'oscillateur\\
    \(e\) & \(f\) & Excitation\\
    \(\frac{1}{\sqrt{LC}}\) & \(\sqrt{\frac{k}{m}}\) & Pulsation propre
    \(\omega_0\)\\
    \(\frac{L\omega_0}{R} = \frac{1}{R}\sqrt{\frac{L}{C}}\) &
    \(\frac{m\omega_0}{\alpha} = \frac{1}{\alpha}\sqrt{km}\) & Facteur de
    qualité \(Q\)\\
    \(\frac{L}{2R}\) & \(\frac{m}{2\alpha}\) & Coefficient d'amortissement
    \(\lambda\)\\
    \(q\) & \(x\) & \\
    \(i=\dot{q}\) & \(v=\dot{x}\) & \\
    \(p=ei\) & \(p=fv\) & Puissance reçue\\
    \hline
  \end{tabular}
  \caption{Tableau d'analogies entre la mécanique et l'électricité}
  \label{tab:equiv}
\end{table}%

L'équation différentielle s'écrit suivant les cas~:
\[\ddot{q}+2\lambda\dot{q}+\omega_0^2 q = e/L \text{~ou~}
\ddot{q}+\frac{\omega_0}{Q}\dot{q}+\omega_0^2 q = e/L\]%
\[\ddot{x}+2\lambda\dot{x}+\omega_0^2 q = e/L \text{~ou~}
\ddot{x}+\frac{\omega_0}{Q}\dot{x}+\omega_0^2 x = e/L\]%

La résolution de l'équation sans second membre donne les différents types de
régimes transitoires. Les équations différentielles vérifiées par l'intensité 
et
par l'abscisse de la vitesse sont obtenues en dérivant par rapport au temps
celles qui sont vérifiées par \(q\) et \(x\)~:
\[L\ddot{i}+R\dot{i}+i/C = \dot{e} \text{~et~} m\ddot{v}+\alpha\dot{v}+kv =
\dot{f}\]%

\subsection{Notion d'impédance mécanique}%

En régime sinusoïdal permanent, de pulsation \(\omega\), \(\ic\) et \(\vitc\)
vérifient les mêmes équations, en électricité on a vu que le circuit RLC série
a pour impédance électrique \(\Zc = \frac{\Ec}{\Ic} = R + \ju\left(L\omega
-\frac{1}{C\omega}\right)\). De même on définit une impédance mécanique \(\Zc =
\frac{\Fc}{\Vitc} = \alpha +\ju\left(m\omega -\frac{k}{\omega}\right)\).%

\section{Exercices}%
\begin{exercice}[Étude mécanique d'un haut-parleur]%
  %TODO : Mettre la représentation du haut-parleur
  On modélise la partie mécanique d'un haut-parleur par une masse \(m\) 
  assimilée
  à un point matériel \(M\), se déplacant sur une ige horizontale d'axe 
  \((Ox)\),
  reliée à un ressort de longueur à vide \(L_0\) et de raideur \(k\) et liée à 
  un
  amortisseur fluide (piston solidaire de \(M\), se déplaçant dans un liquide) 
  qui
  exerce sur \(M\) la force \(-f \vvv\). La masse \(m\) est celle du noyau de 
  fer
  et du cône du haut-parleur, elle est soumise en plus à une force magnétique
  \(\vF\) proportionnelle au courant \(i(t)\) entrant dans la bobine (fixe)
  du haut-parleur.
  %TODO : Mettre le circuit électrique du haut-parleur

  On a \(\vF = Ki \ux\) avec \(i = I_m \cos(\omega t)\), \(m = \qty{10}{g}\),
  \(k = \qty{15e3}{N/m}\), \(K = \qty{200}{N/A}\), \(I_m = \qty{1.0}{A}\) et
  \(\omega = 2000\pi\unit{rad/s}\).
  \begin{enumerate}
    \item Écrire l'équation différentielle vérifiée par l'abscisse \(x =
      \bar{OM}\) de \(M\) (\(O\) étant la position de \(M\) lorsque le ressort 
      est
      au repos).
    \item La normaliser, avec la pulsation propre \(\omega_0\) et le facteur de
      qualité \(Q\). On veut \(Q = \frac{\sqrt{2}}{2}\), calculer le 
      coefficient
      \(f\).
    \item Déterminer l'expression de la réponse forcée \(x = X_m\cos(\omega t
      +\varphi)\) en calculant \(X_m\) et \(\varphi\).
    \item Tracer l'allure de la courbe représentant \(X_m\) en fonction de
      \(\omega\). Quelle est la bande passante du système pour l'amplitude
      \(X_m\).
  \end{enumerate}
\end{exercice}%
\begin{exercice}[Ressorts en série, ressorts en parallèle]%
  On dispose de deux ressorts parfaitement élastiques, de masses négligeables
  devant la masse \(m\) du point matériel \(M\), de raideurs \(k\) et \(k'\), 
  de
  longueurs à vide \(L_0\) et \(L'_0\).
  \begin{enumerate}
    \item Les ressorts et \(M\) sont enfilés sur une tige horizontale comme
      indiqué par le schéma. La distance entre les deux supports est fixe et
      vaut \(L_0+L'_0\). Le point \(O\) est la position de \(M\) lorsque les
      ressorts sont au repos et on note \(x = \bar{OM}\).
      La force de frottement exercée par la tige sur \(M\) est négligée mais
      l'ensemble est plongé dans un fluide de viscosité \(\eta\), de masse
      volumique \(\mu\) et \(M\) (qui est une sphère de rayon \(r\)) subit une
      frottement fluide qui vaut \(-6\pi\eta r \vvv\). On notera \(\alpha =
      6\pi\eta r\).
      %TODO : Mettre le schéma mécanique.

      Écrire l'équation différentielle du mouvement. Quelle est la raideur 
      \(K\)
      du ressort équivalent aux deux ressorts? Représenter le dispositif sous 
      la
      forme d'un circuit mécanique après avoir précisé si les ressorts sont
      analogues à des condensateurs en série ou en parallèle.
    \item Les deux ressorts sont maintenant attachés l'un à l'autre comme le
      montre le dessin ci-dessous.
      %TODO : Mettre le schéma mécanique.

      Exprimer la longueur \(L'\) du ressort \((L'_0, k')\) en fonction de la
      longueur \(L\) de l'autre ressort, avec \(k, k', L_0\) et \(L'_0\). En
      déduire la relation entre l'abscisse \(x\) de \(M\) et la longueur \(L\),
      avec \(k, k'\) et \(L_0\). Exprimer la force de rappel exercée sur \(M\)
      par le ressort auquel il est attaché. Quelle est la raideur du ressort
      équivalent, pour le point \(M\), aux deux ressorts ainsi groupés?
      Représenter le dispositif sous la forme d'un circuit mécanique après 
      avoir
      précisé si les ressorts sont analogues à des condensateurs en série ou en
      parallèle.
    \item Les deux ressorts sont enfilés l'un dans l'autre, accrochés au point
      \(A\) fixe et, à l'autre extrémité, à \(M\). Le tout est suspendu
      verticalement. On note \(L\) la longueur \(AM\) commune aux deux ressorts
      et \(L_{eq}\) sa valeur à l'équilibre.
      %TODO : Mettre le schéma mécanique.

      Démontrer que l'ensemble des deux ressorts exerce sur \(M\) la même force
      qu'un unique ressort \((K, \Lambda_0)\), en exprimant \(K\) avec \(k\) et
      \(k'\) et \(\Lambda_0\) avec \(L_0, L'_0, k, k'\).

      Écrire la relation vérifiée par la valeur \(L_{eq}\) de \(L\) à
      l'équilibre, avec \(K\), \(\Lambda_0\), \(L\), \(m\), \(g\), \(\mu\) et
      \(r\). La position \(O\) de \(M\) à l'équilibre est prise comme origine
      des abscisses \(x\). Le point \(M\) est soumis, en plus de son poids, à 
      la
      poussée d'Archimède, de la force de frottements visqueux et des forces
      exercées par les ressorts, à une force excitatrice \(\vv{f} = f\ux\). 
      Écrire
      l'équation différentielle vérifiée par \(x\) avec \(K, \alpha, m\) et
      \(f\). Représenter le dispositif sous la forme d'un circuit mécanique.
  \end{enumerate}
\end{exercice}%
