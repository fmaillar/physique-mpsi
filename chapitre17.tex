\chapter{Oscillations mécaniques forcées}
\minitoc
\minilof
\minilot

\section{Oscillateur harmonique soumis à des frottements fluides et à une force excitatrice sinusoïdale}
On considèrera dans ce chapitre un point matériel \(M\), de masse \(m\), mobile sur un axe \((Ox)\), soumis à~:
\begin{itemize}
    \item une force de rappel, due à un ressort parfait, \(-kx \ux\),
    \item une force de frottement fluide, visqueux, du type \(-\alpha \dot{x} \ux\), c'est-à-dire proportionnelle à la vitesse du point \(M\),
    \item une force excitatrice \(f \ux\), on nétudiera que le cas où la force excitatrice est une fonction sinusoïdale de la date soit \(f = F\cos(\omega t + \varphi)\).
\end{itemize}
L'élongation de \(M\) est solution d'une équation différentielle~:
\begin{equation}
    m\ddot{x}+\alpha\dot{x}+kx = f.
\end{equation}
La solution de l'équation différentielle est la somme de la solution générale de l'équation sans second membre (notée \(x_T\)) et d'une solution particulière de l'équation complète (notée \(x_P\)). Les constantes d'intégration intervenant dans \(x_T\) étant déterminées par les conditions initiales (en utilisant la solution complète \(x = x_T+x_P\)). 

La solution de l'équation sans second membre \(x_T\) peut prendre différentes forme, comme on l'a vu dans la section \ref{sec:regpropres}, suivant la valeur de l'amortissement. Dans tous les cas, la solution générale tend vers 0 du fait de la présence d'exponentielles décroissantes \ldots

On a vu en électrocinétique qu'une solution particulière de ce type d'équation est sinusoïdale, de même pulsation que le second membre de l'équation différentielle, soit \(x_P = X\cos(\omega t + \Psi)\). Au bout d'un temps, on peut faire l'approximation \(x = x_P\) et l'oscillateur est en \emph{régime sinusoïdal permanent}. 

On ne cherchera ici que la solution particulière sinusoïdale, c'est-à-dire l'amplitude et le déphasage de \(x\) par rapport à \(f\), en régime sinusoïdal permanent.

\section{Recherche de l'amplitude et du déphasage en régime sinusoïdal permanent}
On peut mettre l'équation différentielle sous l'une de ses formes en posant \(\lambda = \alpha/(2m)\) le coefficient d'amortissement et \(\omega_0=\sqrt{k/m}\) la pulsation propre, alors \(\ddot{x}+2\lambda \dot{x}+\omega_0^2x = f/m\).
En définissant le facteur de qualité \(Q = \frac{m\omega_0}{\alpha}=\frac{\sqrt{km}}{\alpha}=\frac{k}{\alpha\omega_0}\) on peut aussi écrire que \(\ddot{x}+\frac{\omega_0}{Q}\dot{x}+\omega_0^2 x=\frac{f}{m}\).

La résolution utilise la même méthode qu'en électrocinétique~: on associe à chaque grandeur fonction sinusoïdale du temps, une \emph{grandeur complexe associée}.
\(\xc = \Xc\exp(\ii\omega t)\) avec \(\Xc = X\exp(\ii\psi)\) et \(\fc = \Fc\exp(\ii\omega t)\) avec \(\Fc = F\exp(\ii\phi)\).

\(\xc\) est donc solution de l'équation différentielle complexe associée~: \(\ddot{\xc}+\frac{\omega_0}{Q}\dot{\xc}+\omega_0^2 \xc=\frac{\fc}{m}\), soit \((\ii\omega)^2\xc + (\ii\omega)\frac{\omega_0}{Q}\xc+\omega_0^2\xc = \frac{\fc}{m}\), soit en divisant par \(\exp(\ii\omega t)\) les deux membres~:
\begin{equation}
    \Xc = \frac{\Fc/m}{\omega_0^2-\omega^2+\ii\frac{\omega\omega_0}{Q}}.
\end{equation}

On déduit donc l'amplitude de l'élongation en passant au module~:
\begin{equation}
    X = \frac{F/m}{\sqrt{(\omega_0^2-\omega^2)^2+\left(\frac{\omega\omega_0}{Q}\right)^2}},
\end{equation}
et le déphasage de l'élongation par rapport à l'abscisse de la force excitatrice~: \(\psi-\varphi = \Phi\)~: 
\begin{itemize}
    \item si \(\omega<\omega_0\)~: \(\Phi = \arctan(\frac{\omega\omega_0}{Q(\omega^2-\omega_0^2)})\);
    \item si \(\omega>\omega_0\)~: \(\Phi = -\pi + \arctan(\frac{\omega\omega_0}{Q(\omega^2-\omega_0^2)})\);
    \item si \(\omega=\omega_0\)~: \(\Phi = -\frac{\pi}{2}\)
\end{itemize}
On constate que si \(\omega = 0\), alors \(\Phi=0\) et \(X_0 = \frac{F}{m\omega_0^2}\). En l'infini, \(X\) tend vers 0 et \(\Phi\) tend vers \(-\pi\).

\section{Résonance d'élongation}
\subsection{Pulsation à la résonance, amplitude de l'élongation à la résonance}

L'amplitude de l'élongation est maximale lorsque \((\omega_0^2-\omega^2)^2+\left(\frac{\omega\omega_0}{Q}\right)^2\) est minimal, donc lorsque sa dérivée (par rapport à \(\omega\)) est nulle~: \(2(-2\omega)(\omega_0^2-\omega^2)+2\omega\frac{\omega_0^2}{Q^2} = 0\), c'est-à-dire pour \(\omega = 0\) et pour \(\omega = \omega_0\sqrt{1-\frac{1}{2Q^2}}\).

\emph{Il n'y a donc résonance d'élongation que si \(Q>\frac{1}{\sqrt{2}}\)} et la pulsation à la résonance est inférieure à la pulsation propre \(\omega_r = \omega_0\sqrt{1-\frac{1}{2Q^2}}\). L'amplitude à la résonance d'élongation est donc~: \(X_r = \frac{QX_0}{\sqrt{1-\frac{1}{4Q^2}}}\). 

Si \(Q>\frac{1}{\sqrt{2}}\), l'amplitude maximale vaut donc l'amplitude à la résonance : \(X_m = X_r\) ; sinon \(X_m = X_0\).

On remarque que \(X_r\) est une fonction croissante de \(Q\) et que \(\lim\limits_{Q\to\infty} \omega_r = \omega_0\) et
\(\lim\limits_{Q\to\infty} X_r = +\infty\). D'autre part, de l'expression de \(\omega_r\), on obtient \(Q^2 = \frac{1}{2\left(1-\left(\frac{\omega_R}{\omega_0}\right)^2\right)}\) et en reportant dans l'expression de \(X_m\), on obtient après simplification~: \(X_m = \frac{X_0}{\sqrt{1-\left(\frac{\omega_r}{\omega_0}\right)^4}}\).

L'équation de la courbe sur laquelle se trouvent les maxima de \(X\) est donc \(X = \frac{X_0}{\sqrt{1-\left(\frac{\omega}{\omega_0}\right)^4}}\).

\subsection{Variables réduites, tracés}

Pour les tracés, on utilise les variables réduites~: \(\xi = \frac{X}{X_0} \ \Omega = \frac{\omega}{\omega_0}\). On a alors~: \(\xi = \left((1-\Omega^2)^2+\frac{\Omega^2}{Q^2}\right)^{-\frac{1}{2}}\) et \(\xi_m = \frac{Q}{\sqrt{1-\frac{1}{4Q^2}}} = \left(\frac{1}{Q^2}-\frac{1}{4Q^4}\right)^{-\frac{1}{2}}\).

Les maxima de \(\xi\) sont sur la courbe d'équation \(\xi = (1-\Omega^4)^{-\frac{1}{2}}\).

Si \(\Omega <1\), alors \(\Phi = \arctan\left(\frac{1}{Q\left(\Omega-\frac{1}{\Omega}\right)}\right)\). Si \(\Omega >1\), alors \(\Phi = \arctan\left(\frac{1}{Q\left(\Omega-\frac{1}{\Omega}\right)}\right) - \pi\). On pourrait encore démontrer que dans tous les cas~: \(\Phi = \arctan\left(Q\left(\frac{1}{\Omega}-\Omega\right)\right)-\frac{\pi}{2}\).

Les courbes sont tracées sur la figure \ref{fig:oscforcees}.

\begin{figure}[!h]
    \centering
    \includegraphics[scale=0.7]{./Fig17.png}
    \caption{\(\xi\) et \(\Phi\) à différentes valeurs valeurs du facteur de qualité}
    \label{fig:oscforcees}
\end{figure}

\section{Bande passante à \SI{-3}{\deci\bel}}

Pour \(Q \leq \frac{1}{\sqrt{2}}, \xi_m = 1\), donc à l'unique coupure, on a \(\left(1-\Omega_c^2)^2+\frac{\Omega^2}{Q^2}\right)^{-0.5} = 2^{-0.5}\).
