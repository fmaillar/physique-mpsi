\chapter{Lentilles sphériques minces}%
\label{chap:lentillesspheriques}%
\minitoc{}
\minilof{}
\minilot{}

\section{Différents types de lentilles sphériques}%
\label{chap8-sec:differentstypes}%

Une lentille sphérique est forme par un milieu transparent, homogène et 
optiquement isotrope, limité par deux calottes sphériques. Placée dans l'air, 
la lentille forme donc \emph{deux dioptres sphériques} traversés successivement 
par la lumière, c'est-à-dire deux surfaces réfringentes correspondant au 
passage de la lumière de l'air au milieu qui constitue la lentille, et au 
passage de ce milieu à l'air.
\emph{L'axe principal d'une lentille est 1a droite qui passe par les centres de 
courbure des deux faces.}%

Les deux réfractions subies par un rayon lumineux tendent à rabattre un rayon 
lumineux vers la base du prisme constitué par les plans tangents aux points 
d'incidence. On en déduit ainsi le caractère convergent ou divergent des 
différents types de lentilles, suivant leur action sur un faisceau lumineux 
parallèle à l'axe principal.

\section{Lentilles sphériques minces, leurs représentations}%
\label{chap8-sec:lentillesspheriquesminces}%

\emph{Une lentille sphérique mince est une lentille sphérique dont l'épaisseur, 
mesurée sur l'axe principal, est petite devant les rayons de courbure de ses 
deux faces.}%
La lentille mince est donc assimilée grossièrement à un plan, le plan de la 
lentille. L'intersection du plan de la lentille avec son axe principal est le 
centre optique de la lentille. Tout plan perpendiculaire à l'axe principal est 
un plan frontal.

Toute droite passant par le centre optique de la lentille est un axe secondaire 
de la lentille. Les lentilles minces sont représentées ainsi :
% Mettre une figure

\section{Conditions de stigmatisme approché pour une lentille mince, conditions 
de Gauss et aplanétisme}%
\label{chap8-sec:conditions}%

L'étude du dioptre sphérique n'étant pas au programme, on admettra les 
résultats suivants sans démonstration. Les conditions de stigmatisme approché 
sont les mêmes que pour le miroir sphérique; ce sont les conditions de Gauss :
\emph{Une lentille mince est approximativement stigmatique pour des rayons 
lumineux paraxiaux.}%

Utilisée dans ces conditions, la lentille mince est \emph{aplanétique; l'image 
d'un petit objet contenu dans un plan frontal et voisin de l'axe principal est 
elle-même contenue dans un plan frontal, conjugué de celui de l'objet.}

\section{Foyers d'une lentille mince}%
\label{chap8-sec:foyers}%

\subsection{Foyers principaux objet et image}%
\label{chap8-subsec:foyersprincipaux}%

Les définitions sont les mêmes que pour le miroir sphérique;
\emph{Le foyer principal objet, noté F, est le point de l'axe principal dont 
l'image est située à l'infini sur l'axe principal. Le foyer principal image, 
noté F', est le point de l'axe principal image du point objet à l'infini sur 
l'axe principal.}%

Pour une lentille sphérique mince, ces deux points sont distincts, \emph{les 
deux foyers sont symétriques l'un de l'autre par rapport au centre optique de 
la lentille}.

\subsection{Plans focaux, distance focale}%
\label{chap8-subsec:plansfocaux}%

\emph{Les plans frontaux contenant les foyers principaux objet et image sont 
respectivement nommés plan focal objet et plan focal image de la lentille}. Les 
distances sont mesurées algébriquement sur l'axe principal avec le sens de 
propagation de la lumière comme sens positif. \emph{La distance algébrique du 
centre optique au foyer principal image est la distance focale de la lentille, 
on la note \(f'\)}, \(f'= \overline{OF'}\). On la nomme encore ``distance 
focale image'' alors que \(f = \overline{OF}= – f'\) est la ``distance focale 
objet''.%

Pour une lentille convergente \(f' > 0\) et les deux foyers sont réels. Alors 
que pour une lentille divergente \(f' < 0\) et les deux foyers sont virtuels.

\subsection{Vergence}%
\label{chap8-subsec:vergence}%

Par définition, la vergence d'une lentille sphérique mince est l'inverse de sa 
distance focale \(C =\frac{1}{f}\). La vergence d'une lentille convergente est 
positive, celle d'une lentille divergente est négative. Son unité SI est la 
dioptrie. \(1 \delta = \SI{1}{m^{–1}}\).

On démontre à partir des formules du dioptre (hors programme) que la distance 
focale dépend des rayons de courbure des faces et de l'indice de la lentille 
par rapport à l'air par la formule
\begin{equation}%
  C = \frac{1}{f'} = (n-1) \left(\frac{1}{R_1} - \frac{1}{R_2}\right),
\end{equation}%
avec \(R_1 = \overline{OC_1}\) et \(R_2 = \overline{OC_2}\) avec \(C_1\) et 
\(C_2\) les centres de courbures des deux faces.

\subsection{Foyers secondaires}%
\label{chap8-subsec:foyerssecondaires}%

L'intersection d'un axe secondaire avec le plan focal objet est un foyer 
secondaire objet ; c'est le point objet conjugué du point image situé à 
l'infini sur cet axe secondaire. L'intersection d'un axe secondaire avec 1e 
plan focal image est un foyer secondaire image ; c'est 1e point image conjugué 
d'un point objet situé à l'infini sur cet axe secondaire.

\section{Tracé du rayon émergent correspondant à un rayon incident donné}%
\label{chap8-sec:trace}%

Bien entendu les propriétés énoncées ci-dessous ne sont valables que pour des 
rayons paraxiaux.

Tout rayon passant par le centre optique ne subit aucune déviation à la 
traversée de 1a lentille. À tout rayon incident parallèle à l'axe principal 
(resp.\ à un axe secondaire) correspond un rayon émergent dont le support passe 
par le foyer principal image (resp.\ le foyer secondaire image correspondant).
À tout rayon incident dont le support passe par le foyer principal objet 
(resp.\ par un foyer secondaire objet) correspond un rayon émergent parallèle
à l'axe principal (resp.\ à l'axe secondaire correspondant).

\section{Construction de l'image d'un objet frontal donné, dans les conditions 
de Gauss}%
\label{chap8-sec:constructionimagefrontal}%

On prendra d'abord l'exemple d'un objet réel situé avant le plan focal objet 
d'une lentille convergente. On obtient alors une image réelle, donc observable 
sur un écran diffusant. C'est le cas d'un projecteur de diapositives, d'un 
appareil photographique, d'un agrandisseur etc.
%mettre une image
%

Une lentille convergente donne d'un objet réel une image virtuelle, droite, 
plus grande que l'objet, si l'objet est situé entre le plan focal objet et la 
lentille. C'est le cas de la loupe ou d'un verre correcteur pour hypermétrope 
ou presbyte.
%mettre une image

Une lentille divergente donne d'un objet  réel une image virtuelle droite, plus 
petite que l'objet. C'est le cas d'un verre correcteur pour myope.
%mettre une image

On s'exercera en traitant les autres cas, en particulier avec les lentilles 
divergentes.

\section{Formules des lentilles minces}%
\label{chap8-sec:formulesdeslentillesminces}%
On note habituellement : \(p=\overline{OA}\) et \(p'=\overline{OA'}\) on a 
\(f'=\overline{OF'}=\overline{FO}=-f\).

Si \(p < 0\) l'objet est réel. Si \(p > 0\) l'objet est virtuel. Si \(p' < 0\) 
l'image est virtuelle. Si \(p' > 0\) l'image est réelle.

En observant les figures ci-dessus, on constate du fait des homothéties
\begin{equation}%
  \gamma = \frac{\overline{A'B'}}{\overline{AB}} = 
  \frac{\overline{A'B'}}{\overline{OI}} = \frac{\overline{OI'}}{\overline{AB}}
\end{equation}%
 avec \(  \frac{\overline{A'B'}}{\overline{OI}} =  
 \frac{\overline{F'A'}}{\overline{F'O}}\) et \( 
 \frac{\overline{OI'}}{\overline{AB}} = \frac{\overline{FO}}{\overline{FA}} \) 
 donc \(\gamma = \frac{\overline{FO}}{\overline{FA}} = 
 \frac{\overline{F'A'}}{\overline{F'O}}\).

Donc \(\frac{f'}{p+f'} = \frac{p'-f'}{f'}\) soit \(-f'^2 = -f'^2 + pp' - pf' + 
p'f'\)  et \(pf' - p'f' = pp'\). On obtient la formule de conjugaison en 
divisant les deux membres par \(pp'f'\).
Formule de conjugaison
\begin{equation}%
  \frac{1}{p'} - \frac{1}{p} = \frac{1}{f'} = C.
\end{equation}%

D'autre	 part, on obtient directement \(\gamma = 
\frac{\overline{OA'}}{\overline{OA}}\) avec l'homothétie des triangles \(OAB\) 
et \(OA'B'\).
Formule du grandissement transversal :
\begin{equation}%
  \gamma = \frac{p'}{p}.
\end{equation}%
Ce sont les formules de Descartes des lentilles sphériques minces (ou formules 
avec origine au centre optique).

Le grandissement \(\gamma > 0\) quand \(p\) et \(p'\) sont de même signe donc 
quand l'objet est réel et l'image virtuelle ou l'inverse. Le grandissement 
\(\gamma < 0\) quand \(p\) et \(p'\) sont de signes contraires donc quand 
l'objet et l'image sont tous deux réels ou tous deux virtuels. Si \(\gamma > 
0\) l'image est droite, de nature différente de l'objet. Si \(\gamma < 0\) 
l'image est renversée, de même nature que l'objet. Si \(|\gamma| > 1\)  l'image 
est plus grande que l'objet, sinon l'image est plus petite que l'objet.

De \(\gamma = \frac{\overline{FO}}{\overline{FA}} = 
\frac{\overline{F'A'}}{\overline{F'O}}\) on tire les formules de Newton ou 
formules avec origines aux foyers.
Formules du grandissement de Newton : \(\gamma = 
\frac{f'}{\overline{FA}}=-\frac{\overline{F'A'}}{f'}\)
Formule de conjugaison de Newton : \(\overline{FA}\overline{FA'} = ff' = 
-f'^2\).

\section{Image d'un objet à l'infini}%
\label{chap8-sec:imagealinfini}%

On prendra l'exemple d'un objet réel à l'infini dont une lentille divergente 
donne une image virtuelle droite.
% mettre image
%

\(B'\) se trouve au foyer secondaire correspondant à l'axe secondaire \(BO\). 
Si la lentille est convergente, l'image est réelle et renversée. L'image et 
l'objet sont vus de \(O\) sous le même angle \(\alpha\), toujours petit, si les 
conditions de Gauss sont respectées, donc \(\alpha = tan \alpha = 
\frac{\overline{A'B'}}{\overline{F'O}} = \frac{\overline{A'B'}}{-f'}\) donc 
\(\overline{A'B'} = -\alpha f'\)  (pour \(\overline{AB}>0\)).

\section{Étude analytique des différents cas}%
\label{chap8-sec:etudeanalytique}%

%\subsection{Notations utilisées}
%\label{chap8-subsec:notationsutilisees}

De la formule de conjugaison avec origine au centre optique, on tire
\begin{equation}%
  p' = \frac{pf'}{p+f'} \qquad \gamma = \frac{f'}{p+f'}.
\end{equation}%

Il est pratique d'utiliser des coordonnées réduites sans dimensions~:
\begin{itemize}%
\item Si la lentille est convergente, \(f' > 0\), alors on posera \(x = 
  \frac{p}{f'}\) et \(x' = \frac{p'}{f'}\) donc \(x' = \frac{x}{1+x}\) et 
    \(\gamma = \frac{1}{1+x}\). Ainsi
  \begin{equation}
    \forall x \ \derived{x'}{x} = \frac{1}{(1+x)^2} > 0 \quad 
    \derived{\gamma}{x} = - \frac{1}{(1+x)^2} < 0.
  \end{equation}
\item Si la lentille est divergente, \(f' < 0\), pour que \(x\) soit du même 
  signe que \(p\) et \(x'\) du même signe que \(p'\), il est préférable de 
    poser \(x = \frac{p}{f}\) et \(x' = \frac{p'}{f}\)  donc \(x' = 
    \frac{x}{1-x}\)  et \(\gamma = \frac{1}{1-x}\).
  \begin{equation}
    \forall x \ \derived{x'}{x} = \frac{1}{(1-x)^2} > 0 \quad 
    \derived{\gamma}{x} = - \frac{1}{(1-x)^2} < 0.
  \end{equation}
\end{itemize}%

Si l'objet avance dans le sens positif, l'image fait de même, mais avec une 
discontinuité; elle passe de \(+\infty\) à \(-\infty\) pour \(x = -l\) avec une 
lentille convergente, et de \(-\infty\) à \(+\infty\) pour \(x = 1\) avec une 
lentille divergente, c'est-à-dire dans les deux cas, quand l'objet atteint le 
plan focal objet.
On voit aussi que le grandissement décroît constamment pour une lentille 
convergente et croit constamment pour une lentille divergente, avec une 
discontinuité quand l'objet traverse 1e plan focal objet (\(x = -1\)pour une 
lentille convergente et \(x = 1\) pour une lentille divergente).

\section{Lentilles minces accolées}%
\label{chap8-sec:lentillesmincesaccolées}%

Si des 	lentilles minces sont accolées, leurs centres optiques co\"incident 
pratiquement. La première de distance focale \(f'_1\), donne de l'objet frontal 
\(AB\) une image \(A_1B_1\) qui pour la deuxième lentille est l'objet. La 
deuxième lentille, de distance focale \(f'_2\) donne de \(A_1B_1\) une image 
\(A'B'\) qui est aussi l'image de \(AB\) donnée par le système des deux 
lentilles accolées.

En posant \(p = \overline{OA}\), \(p_1 = \overline{OA_1}\) et \(p' = 
\overline{OA'}\), les formules des lentilles sphériques minces donnent
\begin{equation}%
  \frac{1}{p_1} - \frac{1}{p} = \frac{1}{f'_1} \qquad \frac{1}{p'} - 
  \frac{1}{p_1} = \frac{1}{f'_2}
\end{equation}%

d'où, par addition membre à membre : \(\frac{1}{p'} - \frac{1}{p} = 
\frac{1}{f'_1} + \frac{1}{f'_2} = C_1 + C_2\).

Le grandissement de la première lentille est \(\gamma_1 = \frac{p_1}{p}\), 
celui de la deuxième est \(\gamma_1 = \frac{p'}{p_1}\), celui du système est 
donc \(\gamma = \gamma_1 \gamma_2 = \frac{p'}{p}\).

\emph{L'ensemble de deux lentilles sphériques minces accolées équivaut à une 
lentille unique dont la vergence est la somme des vergences des deux 
lentilles.}%

Si les deux lentilles ont des vergences opposées, la vergence du système est 
nulle, sa distance focale est infinie, c'est-à-dire que ses foyers sont rejetés 
à l'infini. On a alors un \emph{système afocal}. Dans ce cas, \(p' = p\), 
l'image est identique à l'objet.

Une des possibilités pour mesurer la vergence d'une lentille est de rechercher 
la lentille de vergence connue qui, accolée à la première, donnera un système 
afocal. Quelques autres méthodes de \emph{focométrie} (mesure des distances 
focales) seront étudiées en travaux pratiques.

\section{Exercices}%
\label{chap8-sec:exercices}%

\begin{exercice}[Montage \(4f'\) d'une lentille convergente]%
Un objet AB est situé à une distance \(2f'\) en avant d'une lentille mince 
  convergente de distance focale image \(f'\). Où se trouve l'image \(A'B'\)? 
  Quel est le grandissement?
\end{exercice}%
%
\begin{exercice}[Lentille biconcave]%
Une lentille mince biconcave a des faces sphériques dont les rayons de courbure 
  sont \(\SI{0,10}{m}\) et \(\SI{0,15}{m}\). Le verre la constituant a un 
  indice valant \(1,5\). Où se trouve l'image d'un point objet situé à 
  \(\SI{20}{cm}\) en avant de la lentille?
\end{exercice}%
%
\begin{exercice}[Distance entre l'objet et l'image]%
On considère une lentille mince convergente de distance focale image \(f'\), un 
  point objet \(A\) situé sur l'axe et son image \(A'\). Étudier les variations 
  de la distance \(D = \overline{AA'}\) en fonction de la position de l'objet 
  \(A\) par rapport à la lentille.
\end{exercice}%
%
\begin{exercice}[Détermination d'une distance focale]%
Une lentille mince convergente donne d'un objet une image sur un écran, 
  agrandie deux fois. Lorsqu'on rapproche de \(\SI{0,36}{m}\) la lentille de 
  l'écran, la taille de l'image devient la moitié de celle de l'objet. 
  Déterminer la distance focale image de la lentille.
\end{exercice}%
%
\begin{exercice}[Mesure de la distance focale d'une lentille par la méthode de 
  Bessel]%
Un objet frontal \(AB\) et un écran (E) sont fixes et distants de \(D\). Entre 
  l'objet et l'écran, on déplace une lentille mince convergente de distance 
  focale image \(f'\). Montrer que si \(D \geq 4f'\), il existe deux positions 
  de la lentille distantes de \(d\) pour lesquelles il y a une image nette de 
  l'objet sur l'écran. Calculer \(f'\) en fonction de \(D\) et \(d\).
\end{exercice}%
%
\begin{exercice}[Méthode d'autocollimation]%
On accole une lentille mince convergente de distance focale \(f'\) et un miroir 
  plan. On éclaire ce dispositif au moyen d'un petit objet lumineux. Lorsque 
  celui-ci est à \(\SI{0,1}{m}\) du dispositif, l'image se forme dans le plan 
  de l'objet. Calculer la distance focale de la lentille.
\end{exercice}%
%
\begin{exercice}[Position de l'image donnée par un système catadioptrique]%
Un système optique est formé d'une lentille mince de distance focale 
  \(\SI{0,3}{m}\) et d'un miroir plan disposé à \(\SI{0,15}{m}\) de la 
  lentille. Déterminer la position de l'image que ce système donne d'un objet 
  situé à \(\SI{0,15}{m}\) en avant de la lentille.
\end{exercice}%
%
\begin{exercice}[Construction de l'image donnée par un système catadioptrique]%
Construire l'image d'un objet à travers un système optique formé d'une lentille 
  mince et d'un miroir plan disposé dans le plan focal image de la lentille. 
  L'objet est placé en avant de la lentille à une distance comprise entre la 
  distance focale et deux fois la distance focale.
\end{exercice}%
%
\begin{exercice}[Points doubles d'un système catadioptrique]%
Un système est formé par l'association d'une lentille mince convergente de 
  distance focale \(\SI{0,1}{m}\) et d'un miroir plan situé à \(\SI{0,2}{m}\) 
  de la lentille. Déterminer le ou les point(s) de l'axe qui est à lui-même, ou 
  qui sont à eux-mêmes, leur propre image. Ces points sont dits points doubles 
  ou points de Bravais.
\end{exercice}%
%
\begin{exercice}[Construction de l'image donnée par un système catadioptrique]%
Un système optique est formé d'un miroir sphérique concave de distance focale 
  \(\SI{0,1}{m}\) et d'une lentille mince convergente de distance focale 
  \(\SI{0,2}{m}\). La distance entre la lentille et le miroir est 
  \(\SI{0,3}{m}\). Un objet est placé à \(\SI{40}{cm}\) de la lentille. 
  Construire son image à travers le système.
\end{exercice}%
%
\begin{exercice}[Système catadioptrique afocal]%
On associe une lentille divergente et un miroir sphérique concave de façon à ce 
  que le foyer principal image de la lentille soit confondu avec le centre du 
  miroir et que le foyer principal objet de celle-ci soit confondu avec le 
  sommet du miroir. Construire l'image d'un objet dans les conditions de Gauss, 
  déterminer le grandissement transversal.
\end{exercice}%
%
\begin{exercice}[Foyers d'un doublet]%
Un doublet est formé d'une lentille convergente de distance focale 
  \(\SI{15}{cm}\) et d'une lentille convergente de distance focale 
  \(\SI{10}{cm}\), les centres optiques des deux lentilles étant distants de 
  \(\SI{5}{cm}\). Déterminer les positions des foyers du doublet.
\end{exercice}%
%
\begin{exercice}[Doublet afocal]%
Une lentille convergente de \(\SI{0,3}{m}\) de distance focale et une lentille 
  divergente de \(\SI{0,1}{m}\) de distance focale sont distantes de 
  \(\SI{0,2}{m}\). Où faut-il placer une source lumineuse pour que ce doublet 
  donne un faisceau de rayons parallèles.
\end{exercice}%
%
\begin{exercice}[Étude graphique d'un doublet]%
Étudier graphiquement le doublet de symbole (–1; 2; –1).
\end{exercice}%
%
\begin{exercice}[Points doubles d'un doublet]%
Deux lentilles convergentes (L1) et (L2) de distances focales images \(f'_1\), 
  et \(f'_2\) forment un système afocal. Soit un objet \(AB\) repéré par la 
  distance algébrique \(\overline{F_1A}= x1\) et soit son image \(A'B'\) à 
  travers le doublet repéré par la distance algébrique \(\overline{F'_2A'}= 
  x_2\). Déterminer les relations de conjugaison du doublet. Application 
  numérique \(f'_1  = \SI{20}{cm}\), \(f'_2 = \SI{2}{cm}\), déterminer le point 
  de l'axe qui est à lui-même sa propre image.
\end{exercice}%
%
\begin{exercice}[Lunette de Galilée]%
Une lunette de Galilée est constituée d'une première lentille mince convergente 
  (L1) de distance focale \(f'_1 = \SI{0,3}{m}\) (objectif) et d'une seconde 
  lentille mince divergente (L2) de distance focale \(f'_2 = -\SI{0,12}{m}\) 
  (oculaire). Ces lentilles sont distantes de \(\SI{0,12}{m}\). Au moyen de 
  cette lunette, on observe un objet très éloigné vu sous le diamètre apparent 
  \(10'\). Déterminer les caractéristiques de l'image donnée par la lunette.
\end{exercice}%
%
\begin{exercice}[Téléobjectif]%
Un téléobjectif est formé d'une lentille mince convergente de distance focal 
  image \(\SI{5}{cm}\) et d'une lentille mince divergente de distance focale 
  image \(-\SI{2}{cm}\) distantes de \(\SI{3,5}{cm}\).
À quelle distance de la lentille convergente, l'image d'un objet lointain se 
  forme-t-elle? Quelle en est la taille si l'objet est vu sous un angle de 
  \(5'\) de la première lentille?
\end{exercice}%
%
\begin{exercice}[Téléobjectif]%
Un téléobjectif d'appareil photographique est constitué d'une lentille 
  convergente (L1) de distance focale \(f'_1 = \SI{0,06}{m}\) et d'une lentille 
  divergente (L2) de distance focale \(f'_2 = -\SI{0,08}{m}\). Les centres 
  optiques des deux lentilles sont distants de \(d = \overline{O_1O_2} = 
  \SI{0,02}{m}\). La pellicule photographique est placée dans le plan focal 
  image du téléobjectif.
\begin{enumerate}%
  \item Où faut-il disposer cette pellicule?
  \item Construire l'image d'un objet très éloigné.
  \item L'objet très éloigné est vu depuis le téléobjectif sous un diamètre 
    apparent de \(1'\). Déterminer la grandeur de l'image.
\end{enumerate}%
\end{exercice}%
%
\begin{exercice}[Microscope]%
Un microscope est constitué d'une lentille mince convergente (L1) de distance 
  focale \(f'_1 = \SI{0,002}{m}\) (objectif) et d'une lentille mince 
  convergente (L2) de distance focale \(f'_2 = \SI{0,02}{m}\). La distance 
  entre les foyers \(F'_1\) et \(F_2\) est \(d = \SI{0,159}{m}\). Un objet de 
  longueur \(\SI{0,01}{mm}\) est placé à \(\SI{0,025}{mm}\) du foyer principal 
  objet \(F_1\) de (L1). Déterminer les caractéristiques de l'image donnée par 
  le microscope. Faire une construction géométrique. En déduire les conditions 
  nécessaires pour que le doublet puisse effectivement jouer le rôle de 
  microscope.
\end{exercice}%

% Local Variables:
% mode: latex
% TeX-master: "physique"
% End:
