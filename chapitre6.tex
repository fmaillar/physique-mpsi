\chapter{Lois de Descartes -- Prisme}
\label{chap:loisdedescartes}
\minitoc
\minilof
\minilot

\section{Nature de la lumière}
\label{chap6-sec:naturedelalumiere}

\subsection{Vitesse de propagation, indices de réfraction}
\label{chap6-subsec:vitessedepropagation}

Une onde progressive électromagnétique monochromatique est constituée d'un champ électrique $\vect{E}$ et d'un champ magnétique $\vect{B}$ vibrant en phase, perpendiculairement entre eux et perpendiculairement à la direction de propagation. 
Si la fréquence de la radiation électromagnétique est $\nu$, alors la période est $\tau = \frac{1}{\nu}$. Si la vitesse de propagation est $v$, la longueur d'onde (période spatiale) est la distance dont l'onde progresse en une période $\lambda = v \tau$. Dans le vide $\lambda_0 = \frac{c}{\nu}$.

Pour un milieu isotrope, homogène, où la vitesse de propagation est $v$, l'indice de réfraction absolu du milieu pour la fréquence $\nu$ est $N = \frac{c}{v}$, $\lambda =  \frac{\lambda_0}{N}$.

Pour deux milieux 1 et 2, homogènes et optiquement isotropes, l'indice de réfraction du milieu 2 par rapport au milieu 1 est le rapport $n_{2/1}= \frac{v_1}{v_2}=\frac{N_2}{N_1}$. Les indices de réfraction dépendent de la fréquence de la radiation considérée.

\subsection{Différents domaines des ondes électromagnétiques}
\label{chap6-subsec:domainesdesondes}

Suivant la longueur d'onde dans le vide, on distingue différents domaines de radiation. La lumière visible correspond à une longueur d'onde dans le vide $\lambda_0$ telle que
\begin{equation}
  \SI{400}{nm} < \lambda_0 < \SI{750}{nm}.
\end{equation}

On distingue différentes couleurs, dans l'ordre des longueurs d'onde croissante~: violet, indigo, bleu, vert, jaune, orangé, rouge. Aux longueurs d'ondes dans le vide inférieure à $\SI{400}{nm}$ jusqu'à environ $\SI{1}{nm}$, on trouve le domaine de l'ultraviolet, puis les rayons $X$, les rayons $\gamma$ et les rayons cosmiques. Aux longueurs d'ondes au-delà de $\SI{750}{nm}$ jusqu'à environ $\SI{1}{mm}$, on trouve le domaine de l'infrarouge, puis celui des ondes radioélectriques.

Dans le domaine de la lumière proprement dite, des ultraviolets et des infrarouges l'indice absolu d'un milieu est toujours supérieur à $1$ et la vitesse de la lumière dans l'air et voisine de $c$. On utilise couramment les indices par rapport à l'air ($n$) au lieu des indices absolus ($N$). 

\section{Propagation rectiligne}
\label{chap6-sec:propagationrectiligne}

\subsection{Hypothèse de propagation rectiligne}
\label{chap6-subsec:hypothesedepropagationrectiligne}

Toute l'optique géométrique repose sur l'hypothèse de la propagation rectiligne de la lumière issue d'une source ponctuelle dans le cas d'une propagation homogène. Cette hypothèse est remise en question par les phénomènes de diffraction qui deviennent appréciable chaque fois que la lumière doit passer dans des fentes dont la largeur est du même ordre (ou inférieure) que la longueur d'onde.

\subsection{Rayons lumineux}
\label{chap6-subsec:rayonslumineux}

\begin{defdef}
 Le trajet rectiligne de la lumière issue d'un point est un rayon lumineux.
\end{defdef}

Un faisceau isogène est formé par des rayons lumineux issu d'un point source. S'il est étroit, on parle de pinceau lumineux. Si la lumière est issue d'une source étendue, on a un faisceau complexe forme une infinité de faisceaux isogènes.

Les phénomènes de réflexion et de réfraction permettant de modifier la direction des rayons lumineux, on peut avoir des faisceaux lumineux divergents, convergents ou cylindriques.

\section{Lois de Descartes}
\label{chap6-sec:LoisdeSnellDescartes}

\subsection{Définitions}
\label{chap6-subsec:définitions}

Elles précisent le comportement d'un rayon lumineux arrivant sur une surface réfléchissante et réfringente $\Sigma$ séparant deux milieux de propagation homogènes et isotropes.

On appelle point d'incidence le point d'intersection du rayon incident avec la surface $\Sigma$. La normale à $\Sigma$ en ce point et le rayon d'incidence déterminent le plan d'incidence. L'angle entre la normale au point d'incidence et le rayon incident est l'angle d'incidence. Il y a deux rayons émergents. Le rayon émergent dans le même mileu que le rayon incident est le rayon réfléchi, et l'angle qu'il forme avec la normale au point d'incidence est l'angle de réflexion. Le rayon émergent dans le second milieu de propagation est le rayon réfracté, l'angle qu'il forme avec la normale au point d'incidence s'appelle l'angle de réfraction.

\begin{theo}[Lois de Descartes pour la réflexion]
  Le rayon réfléchi appartient au plan d'incidence. L'angle de réflexion est égal à l'angle d'incidence : $i'=i$.
\end{theo}
\begin{theo}[Lois de Descartes pour la réfraction]
  Le rayon réfracté appartient au plan d'incidence. L'angle de réfraction $i_2$ est tel que $n_1 \sin i_1 = n_2 \sin i_2$, en notant $i_1$ l'angle d'incidence et $n$ l'indice des milieux.
\end{theo}

On remarquera que si une lumière monochromatique donnée est plus freinée par le milieu 2 que par le milieu 1, on dit que le milieu 2 est plus réfringent que le milieu 1. Dans ce cas, $v_2 < v_1$ (ou $n_2 > n_1$), et il en résulte que $i_2 > i_1$.

Si le milieu 2 est plus réfringent que le milieu 1, le rayon lumineux se rapproche de la normale lors du passage de la lumière.

\begin{theo}[Loi du retour inverse]
  Le trajet suivi par la lumière n'est pas modifié si le sens de propagation est inversé.
\end{theo}

\subsection{Réflexion totale et réfraction limite}
\label{chap6-subsec:reftotale}
Soit un rayon lumineux passant du milieu 1 au milieu 2 plus réfringent ($n_{2/1}>1$). L'angle de réfraction est tel que $n_{2/1} \sin i_2 = \sin i_1$, donc
\begin{equation}
  i_2 < \arcsin \left(\frac{1}{n_{2/1}} \right) = i_m.
\end{equation}
L'angle limite de réfraction $i_m$ correspond à un rayon incident longeant la surface réfringente ($i_1 = \frac{\pi}{2}$).

Inversement, si la lumière passe du milieu 2 (plus réfringent) au milieu 1 (moins réfringent), l'angle d'incidence $i_2$ ne dépassera pas la valeur $i_m$ obtenue pour un angle de réfraction $i_1 = \frac{\pi}{2}$. Si l'angle d'incidence est supérieur à $i_m$, il y a réflexion totale.

Par exemple, si un verre a l'indice $n=\frac{3}{2}$ par rapport à l'air, pour une lumière monochromatique donnée, l'angle limite de réfraction pour le passage de la lumière de l'air au verre (ou l'angle d'incidence à partir duquel il y a réflexion totale pour le passage du verre à l'air) est $i_m = \arcsin\left(\frac{2}{3}\right) = 41,8°$

\section{Prisme}
\label{chap6-sec:prisme}

Un prisme est un mileu transparent, homogène et optiquement isotrope, limité par deux faces planes et une base. On s'intéressera uniquement ici à un prisme placé dans l'air et éclairé en lumière monochromatique.

Le rectiligne du dière formé par les deux faces planes est nommé angle du prisme, noté souvent $A$. L'indice du mileu transparent par rapport à l'air est l'indice du prisme, noté $n$. Il dépend de la longueur d'onde dans le vide de la lumière. On s'intéresse uniquement ici au cas d'un rayon incident contenu dans un plan de section droite du prisme, c'est-à-dire dans un plan perpendiculaire à l'arête du dièdre.

La première loi de Descartes pour la réfraction implique que le trajet d'un tel rayon lumineux se fera entièrement dans un plan de section principale constitué par le plan d'incidence sur la face d'entrée.

L'angle d'incidence sur la face d'entrée sera noté $i$ et l'angle de réfraction sur la face de sortie $i'$. L'angle $i$ est compté positivement lorsque le rayon incident est du même coté de la normale à la face d'entrée que la base du prisme et négativement s'il est du même côté que l'arête. De même pour l'angle $i'$, il est positif si le rayon émergeant est du même côté de la normale à la face de sortie que la base du prisme et négativement dans l'autre cas.

L'angle de réfraction sur la face d''entrée sera noté $r$ et l'angle d'incidence sur la face de sortie sera noté $r'$. Les angles $r$ et $r'$ sont comptés positivement pour des rayons correspondants situés du même côté de la normale correspondante que l'arête du prisme.

\subsection{Les formules du prisme}
\label{chap6-subsec:formulesprisme}

Les deux premières formules traduisent la deuxième loi de Descartes pour la réfraction : $\sin(i) = n\sin(r)$ et $\sin(i') = n\sin(r')$. Les deux autres sont géométriques : $r+r'=A$ et $D=i+i'-A$. $D$ est la déviation, c'est l'angle entre le rayon incident et le rayon émergeant. \emph{Un rayon lumineux est toujours dévié vers la base du prisme, $D \geq 0$}.

\subsection{Étude de la déviation en fonction de l'angle d'incidence}
\label{chap6-subsec:etudedeviation}
\paragraph{Existence d'un minimum de déviation}

Pour une lumière monochromatique donnée et un prisme donnée, $n$ et $A$ sont constants. Les formules du prisme donnent alors:
$r = \arcsin{\frac{\sin(i)}{n}}$ $r'=A - \arcsin{\frac{\sin(i)}{n}}$ $i' = \arcsin{n\sin\left(A-\arcsin{\frac{\sin(i)}{n}}\right)}$, donc :
\begin{equation}
  D = i + \arcsin{n\sin\left(A-\arcsin{\frac{\sin(i)}{n}}\right)} -A.
\end{equation}
Cette fonction, $D=f(i)$ est assez compliquée à étudier, mais on peut montrer qu'elle admet un seul extremum, qui est d'ailleurs un minimum. On peut le confirmer par expérience.

\paragraph{Symétrie au minimum de déviation}
La loi du retour inverse de la lumière implique que si $i$ prend la valeur $i_1$ et $i'$ la valeur $i_2$ pour un sens donné de la lumière, alors, pour le sens inverse, $i=i_2$ et $i'=i_1$. La déviation est la même pour les deux sens de la lumière. Donc, à chaque valeur de $D$, il correspond deux angles d'incidence possibles.
